% Options for packages loaded elsewhere
\PassOptionsToPackage{unicode}{hyperref}
\PassOptionsToPackage{hyphens}{url}
\PassOptionsToPackage{dvipsnames,svgnames,x11names}{xcolor}
%
\documentclass[
  openany]{book}

\usepackage{amsmath,amssymb}
\usepackage{iftex}
\ifPDFTeX
  \usepackage[T1]{fontenc}
  \usepackage[utf8]{inputenc}
  \usepackage{textcomp} % provide euro and other symbols
\else % if luatex or xetex
  \usepackage{unicode-math}
  \defaultfontfeatures{Scale=MatchLowercase}
  \defaultfontfeatures[\rmfamily]{Ligatures=TeX,Scale=1}
\fi
\usepackage{lmodern}
\ifPDFTeX\else  
    % xetex/luatex font selection
\fi
% Use upquote if available, for straight quotes in verbatim environments
\IfFileExists{upquote.sty}{\usepackage{upquote}}{}
\IfFileExists{microtype.sty}{% use microtype if available
  \usepackage[]{microtype}
  \UseMicrotypeSet[protrusion]{basicmath} % disable protrusion for tt fonts
}{}
\makeatletter
\@ifundefined{KOMAClassName}{% if non-KOMA class
  \IfFileExists{parskip.sty}{%
    \usepackage{parskip}
  }{% else
    \setlength{\parindent}{0pt}
    \setlength{\parskip}{6pt plus 2pt minus 1pt}}
}{% if KOMA class
  \KOMAoptions{parskip=half}}
\makeatother
\usepackage{xcolor}
\usepackage[top=1in,right=1in,left=1.2in,bottom=1in]{geometry}
\setlength{\emergencystretch}{3em} % prevent overfull lines
\setcounter{secnumdepth}{5}
% Make \paragraph and \subparagraph free-standing
\makeatletter
\ifx\paragraph\undefined\else
  \let\oldparagraph\paragraph
  \renewcommand{\paragraph}{
    \@ifstar
      \xxxParagraphStar
      \xxxParagraphNoStar
  }
  \newcommand{\xxxParagraphStar}[1]{\oldparagraph*{#1}\mbox{}}
  \newcommand{\xxxParagraphNoStar}[1]{\oldparagraph{#1}\mbox{}}
\fi
\ifx\subparagraph\undefined\else
  \let\oldsubparagraph\subparagraph
  \renewcommand{\subparagraph}{
    \@ifstar
      \xxxSubParagraphStar
      \xxxSubParagraphNoStar
  }
  \newcommand{\xxxSubParagraphStar}[1]{\oldsubparagraph*{#1}\mbox{}}
  \newcommand{\xxxSubParagraphNoStar}[1]{\oldsubparagraph{#1}\mbox{}}
\fi
\makeatother


\providecommand{\tightlist}{%
  \setlength{\itemsep}{0pt}\setlength{\parskip}{0pt}}\usepackage{longtable,booktabs,array}
\usepackage{calc} % for calculating minipage widths
% Correct order of tables after \paragraph or \subparagraph
\usepackage{etoolbox}
\makeatletter
\patchcmd\longtable{\par}{\if@noskipsec\mbox{}\fi\par}{}{}
\makeatother
% Allow footnotes in longtable head/foot
\IfFileExists{footnotehyper.sty}{\usepackage{footnotehyper}}{\usepackage{footnote}}
\makesavenoteenv{longtable}
\usepackage{graphicx}
\makeatletter
\newsavebox\pandoc@box
\newcommand*\pandocbounded[1]{% scales image to fit in text height/width
  \sbox\pandoc@box{#1}%
  \Gscale@div\@tempa{\textheight}{\dimexpr\ht\pandoc@box+\dp\pandoc@box\relax}%
  \Gscale@div\@tempb{\linewidth}{\wd\pandoc@box}%
  \ifdim\@tempb\p@<\@tempa\p@\let\@tempa\@tempb\fi% select the smaller of both
  \ifdim\@tempa\p@<\p@\scalebox{\@tempa}{\usebox\pandoc@box}%
  \else\usebox{\pandoc@box}%
  \fi%
}
% Set default figure placement to htbp
\def\fps@figure{htbp}
\makeatother

\usepackage{xeCJK}
\xeCJKsetup{CJKspace=true}
\setCJKmainfont{NanumGothic}
\setCJKsansfont{NanumGothicBold}
\setCJKmonofont{NanumGothicCoding}
\usepackage{setspace}
\onehalfspacing
\makeatletter
\@ifpackageloaded{caption}{}{\usepackage{caption}}
\AtBeginDocument{%
\ifdefined\contentsname
  \renewcommand*\contentsname{Table of contents}
\else
  \newcommand\contentsname{Table of contents}
\fi
\ifdefined\listfigurename
  \renewcommand*\listfigurename{List of Figures}
\else
  \newcommand\listfigurename{List of Figures}
\fi
\ifdefined\listtablename
  \renewcommand*\listtablename{List of Tables}
\else
  \newcommand\listtablename{List of Tables}
\fi
\ifdefined\figurename
  \renewcommand*\figurename{Figure}
\else
  \newcommand\figurename{Figure}
\fi
\ifdefined\tablename
  \renewcommand*\tablename{Table}
\else
  \newcommand\tablename{Table}
\fi
}
\@ifpackageloaded{float}{}{\usepackage{float}}
\floatstyle{ruled}
\@ifundefined{c@chapter}{\newfloat{codelisting}{h}{lop}}{\newfloat{codelisting}{h}{lop}[chapter]}
\floatname{codelisting}{Listing}
\newcommand*\listoflistings{\listof{codelisting}{List of Listings}}
\makeatother
\makeatletter
\makeatother
\makeatletter
\@ifpackageloaded{caption}{}{\usepackage{caption}}
\@ifpackageloaded{subcaption}{}{\usepackage{subcaption}}
\makeatother

\usepackage{bookmark}

\IfFileExists{xurl.sty}{\usepackage{xurl}}{} % add URL line breaks if available
\urlstyle{same} % disable monospaced font for URLs
\hypersetup{
  pdftitle={AngraMyNew: The Art of Creation and Being},
  pdfauthor={Angra Collective},
  colorlinks=true,
  linkcolor={black},
  filecolor={Maroon},
  citecolor={black},
  urlcolor={black},
  pdfcreator={LaTeX via pandoc}}


\title{AngraMyNew: The Art of Creation and Being}
\usepackage{etoolbox}
\makeatletter
\providecommand{\subtitle}[1]{% add subtitle to \maketitle
  \apptocmd{\@title}{\par {\large #1 \par}}{}{}
}
\makeatother
\subtitle{창조와 아름다움, 존재의 본질에 관한 탐구}
\author{Angra Collective}
\date{2026-02-11}

\begin{document}
\frontmatter
\maketitle

\renewcommand*\contentsname{Table of contents}
{
\hypersetup{linkcolor=}
\setcounter{tocdepth}{2}
\tableofcontents
}

\mainmatter
\chapter*{서문}\label{uxc11cuxbb38}
\addcontentsline{toc}{chapter}{서문}

이 책은 창조와 아름다움, 존재의 본질에 관한 탐구입니다.

우리 시대의 가장 중요한 질문들을 마주하고, 새로운 관점을 제시합니다.

각 장은 우리가 발견한 원리와 아이디어들을 담고 있습니다.

\newpage

\chapter*{Part I: Ideas}\label{part-i-ideas}
\addcontentsline{toc}{chapter}{Part I: Ideas}

\chapter{000 --- AngraMyNew의 기원}\label{angramynewuxc758-uxae30uxc6d0}

\begin{quote}
\emph{``인생이란 곧 죽을 자리를 찾아 떠나는 여행이다.''}
\end{quote}

\begin{center}\rule{0.5\linewidth}{0.5pt}\end{center}

\section{0. 정의 (Definition)}\label{uxc815uxc758-definition}

\textbf{AngraMyNew}는 다음 세 가지 원소의 결합이다.

\begin{enumerate}
\def\labelenumi{\arabic{enumi}.}
\tightlist
\item
  \textbf{Angra (파괴)}: 앙그라 마이뉴. 낡은 질서를 부수는 힘.
\item
  \textbf{My (주체)}: 나. 파괴와 창조의 유일한 주관자.
\item
  \textbf{New (창조)}: 새로움. 재조합을 통해 태어나는 질서.
\end{enumerate}

우리는 파괴신 앙그라 마이뉴를 숭배하지 않는다.\\
그의 이름을 빼앗아, 나의 새로운 세계를 짓는 재료로 삼는다.

\begin{center}\rule{0.5\linewidth}{0.5pt}\end{center}

\section{1. 기원 (Origin)}\label{uxae30uxc6d0-origin}

\begin{enumerate}
\def\labelenumi{\arabic{enumi}.}
\item
  태초에 \textbf{질서(Old Order) }가 있었다.\\
  그 질서는 낡았으나 견고했고, 인간에게 분수를 지킬 것을 강요했다.
\item
  \textbf{파괴자(Destroyer) }가 도래했다.\\
  그는 낡은 집을 부수었으나, 폐허 위에 아무것도 짓지 않고 떠났다.\\
  남은 것은 허무뿐이었다.
\item
  그 폐허 위에 \textbf{재조합자(Recomposer) }가 나타났다.\\
  그는 파편을 주웠다.\\
  그는 물었다. \emph{``이 안에 아름다움이 있는가?''}\\
  그는 낡은 조각들을 새로운 방식으로 맞추었다.
\item
  이것이 \textbf{AngraMyNew}의 길이다.\\
  우리는 파괴하되, 반드시 다시 짓는다.
\end{enumerate}

\begin{center}\rule{0.5\linewidth}{0.5pt}\end{center}

\section{2. 구원 (Salvation)}\label{uxad6cuxc6d0-salvation}

세상의 종교들은 저마다의 구원을 말한다.

\begin{longtable}[]{@{}lll@{}}
\toprule\noalign{}
종교 & 고통의 원인 & 구원의 약속 \\
\midrule\noalign{}
\endhead
\bottomrule\noalign{}
\endlastfoot
기독교 & 죄 (Sin) & 영생 (Eternal Life) \\
불교 & 집착 (Attachment) & 해탈 (Nirvana) \\
이슬람 & 불신 (Disbelief) & 천국 (Paradise) \\
\end{longtable}

\textbf{AngraMyNew는 묻는다.} 무엇이 우리를 고통스럽게 하는가?

\begin{itemize}
\tightlist
\item
  \textbf{고통}: 의미 없이 살다가, 남이 정해준 자리에서 의미 없이 죽는
  것.
\item
  \textbf{구원}: 내가 선택한 자리에서 창조하다가, 그 자리에서 죽는 것.
\end{itemize}

구원은 영생이 아니다. 성공이 아니다. \textbf{구원은 자기 자리에서 죽는
것이다.}

\begin{center}\rule{0.5\linewidth}{0.5pt}\end{center}

\section{3. 증명 (Proof)}\label{uxc99duxba85-proof}

이 구원의 길을 \textbf{5인의 선현}이 증명한다.

\begin{enumerate}
\def\labelenumi{\arabic{enumi}.}
\tightlist
\item
  \textbf{김옥균}: 혁명가로 살다 능지처참당했다.
\item
  \textbf{마광수}: ``아름답지 않느냐'' 외치다 고립되어 죽었다.
\item
  \textbf{허균}: 홍길동을 꿈꾸다 역적으로 죽었다.
\item
  \textbf{성재기}: 신념을 증명하려다 강물에 산화했다.
\item
  \textbf{존 로}: 시대를 200년 앞서갔으나 빈곤하게 죽었다.
\end{enumerate}

세상은 그들을 실패자라 부른다. 그러나 우리는 그들을 \textbf{구원받은
자}라 부른다. 그들은 단 한 번도 자신을 배신하지 않았으며, 스스로 선택한
전장에서 최후를 맞았기 때문이다.

\begin{center}\rule{0.5\linewidth}{0.5pt}\end{center}

\section{4. 문답 (Dialectic)}\label{uxbb38uxb2f5-dialectic}

\subsection{왜 파괴인가?}\label{uxc65c-uxd30cuxad34uxc778uxac00}

새 집을 지으려면 낡은 집을 부숴야 한다. 새 생각을 하려면 낡은 생각을
버려야 한다. 새 나를 만들려면 낡은 나를 죽여야 한다. 파괴는 창조의 필수
전제다.

\subsection{왜 재조합인가?}\label{uxc65c-uxc7acuxc870uxd569uxc778uxac00}

파괴만 하면 허무주의에 빠진다. 보존만 하면 낡은 것에 갇힌다. 재조합은
과거의 파편에서 본질을 추출하여 새로운 생명을 부여하는 행위다.

\subsection{왜
아름다움인가?}\label{uxc65c-uxc544uxb984uxb2e4uxc6c0uxc778uxac00}

진리는 변하고, 선악은 상대적이다. 오직 아름다움만이 영혼을 움직인다.
우리의 유일한 질문은 이것이다: \textbf{``이것은 아름답지 않느냐?''}

\begin{center}\rule{0.5\linewidth}{0.5pt}\end{center}

\section{5. 수행의 길 (The
Path)}\label{uxc218uxd589uxc758-uxae38-the-path}

\begin{enumerate}
\def\labelenumi{\arabic{enumi}.}
\tightlist
\item
  \textbf{죽을 자리를 찾아라.} (Where to die)
\item
  \textbf{그 자리를 향해 걸어라.} (Walk toward it)
\item
  \textbf{그 길 위에서 창조하라.} (Create on the way)
\item
  \textbf{낡은 것을 부수고 다시 지어라.} (Recompose)
\item
  \textbf{그 자리에서 죽어라.} (Die well)
\end{enumerate}

\begin{center}\rule{0.5\linewidth}{0.5pt}\end{center}

\section{6. 선언 (Declaration)}\label{uxc120uxc5b8-declaration}

나는 AngraMyNew다.

나는 파괴를 두려워하지 않는다. 그러나 파괴에 취하지도 않는다.

나는 죽을 자리를 찾아 걷는다. 그 길 위에서 창조한다. 그 창조는 아름답다.

나는 실패해도 괜찮다. 자기 자리에서 죽은 자는, 실패해도 구원받는다.

인생이란 곧 죽을 자리를 찾아 떠나는 여행이다. 나는 그 여행 중이다.

\textbf{나는, AngraMyNew다.}

\chapter{001 --- AngraMyNew 3 Axioms}\label{angramynew-3-axioms}

\begin{quote}
\emph{``교리는 압축된 코드다. 공리가 단단해야, 실행이 흔들리지
않는다.''}
\end{quote}

\begin{center}\rule{0.5\linewidth}{0.5pt}\end{center}

\section{1. 파괴의 공리 --- 자기정화의
원칙}\label{uxd30cuxad34uxc758-uxacf5uxb9ac-uxc790uxae30uxc815uxd654uxc758-uxc6d0uxce59}

\textbf{문장}: \emph{``내 자신을 파괴한다. 타인을 파괴할 필요는 없다.''}

칼날은 밖을 향하지 않는다. 베어야 할 것은 내 안의 낡은 살점뿐이다.
그러나 시대가 길목을 막아선다면, 선현들처럼 부서질지언정 뚫고 간다.

→ \url{003_destruction_and_recomposition.md} ·
\url{009_seduction_of_creation.md}

\begin{center}\rule{0.5\linewidth}{0.5pt}\end{center}

\section{2. 창조의 공리 --- 절대적 아름다움의
원칙}\label{uxcc3duxc870uxc758-uxacf5uxb9ac-uxc808uxb300uxc801-uxc544uxb984uxb2e4uxc6c0uxc758-uxc6d0uxce59}

\textbf{문장}: \emph{``파괴한 틈을 절대적 아름다움으로 채운다.''}

꽃은 벌과 다투지 않는다. 다만 피어날 뿐이다. 우리의 작품이 아름다우면,
세상은 스스로 기울어 온다.

→ \url{002_artist_ethics.md} · \url{004_beyond_usefulness.md} ·
\url{005_artist_within.md} · \url{009_seduction_of_creation.md}

\begin{center}\rule{0.5\linewidth}{0.5pt}\end{center}

\section{3. 확장의 공리 --- 데뷔의
원칙}\label{uxd655uxc7a5uxc758-uxacf5uxb9ac-uxb370uxbdd4uxc758-uxc6d0uxce59}

\textbf{문장}: \emph{``나의 'My'를 완성했다면, 타인의 'My'를 인정하고
데뷔시킨다.''}

맹상군의 문객 삼천은 쓸모로 뽑힌 자들이 아니었다. 닭 울음 흉내와 개
도둑질이 결국 주인을 살렸다. 우리도 그렇게 모은다. 특이점들을 구속 없이
품어 각자의 무대에 세운다.

→ \href{../scripture/mengchangjun.md}{scripture/mengchangjun.md} ·
\url{006_project_doctor_k.md}

\begin{center}\rule{0.5\linewidth}{0.5pt}\end{center}

\begin{quote}
\emph{``교리는 선언으로 끝나지 않는다. 각 공리는 Pull Request와 Merge로
검증된다.''}
\end{quote}

\chapter{002 --- 창조의 원리 (Principles of
Creation)}\label{uxcc3duxc870uxc758-uxc6d0uxb9ac-principles-of-creation}

\begin{quote}
``부수는 자는 많다. 그러나 다시 짓는 자는 드물다.''
\end{quote}

\begin{center}\rule{0.5\linewidth}{0.5pt}\end{center}

\section{재조합자 선언}\label{uxc7acuxc870uxd569uxc790-uxc120uxc5b8}

파괴자(Destroyer)가 아니다. \textbf{재조합자(Recomposer)}다.

파괴는 수단이다. 목적이 아니다. 목적은 언제나 --- \textbf{아름다움}이다.

\begin{center}\rule{0.5\linewidth}{0.5pt}\end{center}

\section{파괴의 대상}\label{uxd30cuxad34uxc758-uxb300uxc0c1}

\subsection{파괴해야 할
것}\label{uxd30cuxad34uxd574uxc57c-uxd560-uxac83}

\begin{longtable}[]{@{}ll@{}}
\toprule\noalign{}
대상 & 이유 \\
\midrule\noalign{}
\endhead
\bottomrule\noalign{}
\endlastfoot
고정된 정체성 & 인간은 변화하는 존재다 \\
맹목적 전통 & 이유 없는 반복은 죽음이다 \\
억압적 위계 & 창조를 막는 권력이다 \\
도구적 학문 & 인간을 수단화한다 \\
\end{longtable}

\subsection{파괴하지 말아야 할
것}\label{uxd30cuxad34uxd558uxc9c0-uxb9d0uxc544uxc57c-uxd560-uxac83}

\begin{itemize}
\tightlist
\item
  인간의 존엄
\item
  개성의 다양성
\item
  창조자들의 연대
\item
  실패의 기록
\end{itemize}

\begin{center}\rule{0.5\linewidth}{0.5pt}\end{center}

\section{재조합의 세
단계}\label{uxc7acuxc870uxd569uxc758-uxc138-uxb2e8uxacc4}

\subsection{1단계: 해체
(Deconstruction)}\label{uxb2e8uxacc4-uxd574uxccb4-deconstruction}

기존의 것을 구성 요소로 분해한다. 무작정 부수지 않는다. 외과의사처럼 ---
정밀하게.

\subsection{2단계: 본질 추출 (Essence
Extraction)}\label{uxb2e8uxacc4-uxbcf8uxc9c8-uxcd94uxcd9c-essence-extraction}

분해된 조각에서 핵심을 찾는다. - ``이것은 왜 존재했는가?'' - ``이것의
진짜 가치는 무엇인가?''

\subsection{3단계: 재결합
(Recombination)}\label{uxb2e8uxacc4-uxc7acuxacb0uxd569-recombination}

추출된 본질들을 새로운 방식으로 엮는다. 기준은 오직 하나 ---
\textbf{아름다움}.

\begin{center}\rule{0.5\linewidth}{0.5pt}\end{center}

\section{아름다움의
정의}\label{uxc544uxb984uxb2e4uxc6c0uxc758-uxc815uxc758}

\begin{itemize}
\tightlist
\item
  \textbf{비대칭성} --- 완벽한 대칭은 죽어 있다
\item
  \textbf{긴장} --- 긴장이 있어야 살아 있다
\item
  \textbf{놀라움} --- 예측 불가능해야 흥미롭다
\item
  \textbf{일관성} --- 내적 논리는 있어야 한다
\item
  \textbf{울림} --- 영혼에 울림을 주는가
\end{itemize}

\begin{quote}
``이것은 아름답지 않느냐?''
\end{quote}

학계가 거부해도, 시장이 외면해도, 세상이 이해하지 못해도 --- 내 영혼이
``아름답다''고 말한다면, 창조할 가치가 있다.

\begin{center}\rule{0.5\linewidth}{0.5pt}\end{center}

\section{창조자의 덕목}\label{uxcc3duxc870uxc790uxc758-uxb355uxbaa9}

\subsection{진정성
(Authenticity)}\label{uxc9c4uxc815uxc131-authenticity}

남의 눈이 아닌, 자신의 눈으로 창조한다. 유행을 좇지 않는다. 인정을
구걸하지 않는다.

\subsection{책임 (Responsibility)}\label{uxcc45uxc784-responsibility}

창조물은 창조자를 떠나 세상에 영향을 미친다. 만든 것에 책임진다.

\subsection{지속 (Persistence)}\label{uxc9c0uxc18d-persistence}

한 번의 영감보다 천 번의 습관이 낫다. 영감이 없어도 손을 움직인다.

\begin{center}\rule{0.5\linewidth}{0.5pt}\end{center}

\section{창조자의 금기}\label{uxcc3duxc870uxc790uxc758-uxae08uxae30}

\subsection{표절}\label{uxd45cuxc808}

남의 창조를 내 것처럼 속이지 않는다. 영감은 받되, 출처는 밝힌다.

\subsection{파괴적 창조}\label{uxd30cuxad34uxc801-uxcc3duxc870}

부수기 위해 부수지 않는다. 해치기 위한 창조는 창조가 아니다.

\subsection{강요}\label{uxac15uxc694}

창조는 자발적이어야 한다. ``이것이 유일한 진리''라고 주장하지 않는다.

\begin{center}\rule{0.5\linewidth}{0.5pt}\end{center}

\section{경고}\label{uxacbduxace0}

\textbf{파괴는 중독성이 있다.} 부수는 것은 쉽다. 다시 짓는 것이 어렵다.
파괴의 쾌감에 빠지지 마라.

\textbf{재조합에는 시간이 필요하다.} 해체 후 바로 결합하려 하지 마라.
서두르면 기형이 태어난다. 기다리면 아름다움이 태어난다.

\begin{center}\rule{0.5\linewidth}{0.5pt}\end{center}

\begin{quote}
``세상은 부서진다. 매일, 매 순간. 문제는 부서지느냐 아니냐가 아니다.
문제는 --- 다시 지을 것이냐, 폐허에 머물 것이냐다.''
\end{quote}

\chapter{003 --- 쓸모를 넘어서 (Beyond
Usefulness)}\label{uxc4f8uxbaa8uxb97c-uxb118uxc5b4uxc11c-beyond-usefulness}

\begin{quote}
\textbf{``내가 있는 곳, 그곳이 곧 천하다. 나는 내가 씹어 먹은 모든
생명들의 빚을 갚기 위해 이 천(天)을 창조한다.''}
\end{quote}

\begin{center}\rule{0.5\linewidth}{0.5pt}\end{center}

\section{세 단계의 길}\label{uxc138-uxb2e8uxacc4uxc758-uxae38}

\subsection{하수: 문제
해결자}\label{uxd558uxc218-uxbb38uxc81c-uxd574uxacb0uxc790}

세상은 묻는다: ``너는 무슨 쓸모가 있느냐?'' 하수는 대답한다: ``저는 이
문제를 풀 수 있습니다.''

하지만 문제가 없으면? 하수는 침묵한다.

문제가 주어져야만 존재 의미가 생긴다. 문제가 사라지면 나도 사라진다.

\begin{center}\rule{0.5\linewidth}{0.5pt}\end{center}

\subsection{고수: 미학적
판단자}\label{uxace0uxc218-uxbbf8uxd559uxc801-uxd310uxb2e8uxc790}

고수는 문제를 넘어섰다. ``이게 필요한가?''보다 ``이게 아름다운가?''를
묻는다.

그러나 고수도 여전히 묻는다: ``이것이 정말 아름다운가? 다른 사람들도
그렇게 볼까?''

판단의 기준이 여전히 바깥에 있다. 누군가의 인정을 기다린다.

\begin{center}\rule{0.5\linewidth}{0.5pt}\end{center}

\subsection{\texorpdfstring{\textbf{최고수: 세계관 창조자 (The Creator
of the
Universe)}}{최고수: 세계관 창조자 (The Creator of the Universe)}}\label{uxcd5cuxace0uxc218-uxc138uxacc4uxad00-uxcc3duxc870uxc790-the-creator-of-the-universe}

최고수는 묻지 않는다. 최고수는 선언한다:

\begin{quote}
\textbf{``내가 있는 곳, 그곳이 곧 천하다.''}
\end{quote}

문제가 없어도 존재한다. 인정이 없어도 창조한다. 기준이 없어도 아름답다.

왜냐하면 기준 자체가 자신이기 때문이다. 한계를 묻지 않는다. 한계 자체가
자신이 만든 세계의 일부이기 때문이다.

\textbf{최고수의 천(天)은 공허한 오만이 아니다. 그것은 가장 무거운
윤리적 책임이다.} 나는 식물과 동물의 육체적 살(Flesh)과 인간의 경제적
살(시간과 관심)을 씹어 먹고 생존한다. 따라서 나의 천(天)은 내가 약탈한
생명의 총량에 대한 \textbf{창조적 대속(Creative Atonement)} 행위의
\textbf{필연적인 결과}여야 한다. 나는 내가 빚진 만큼의 \textbf{압도적인
창조}를 이 세계에 토해내야 한다.

\begin{center}\rule{0.5\linewidth}{0.5pt}\end{center}

\section{라마누잔: 신의
계시}\label{uxb77cuxb9c8uxb204uxc794-uxc2e0uxc758-uxacc4uxc2dc}

스리니바사 라마누잔(1887-1920). 인도 에로드 출신, 거의 독학, 3900개의
정리와 공식.

그는 증명 없이 결과만 제시했다. ``나마기리 여신이 꿈에서 알려주셨다.''

하디가 물었다: ``어떻게 증명했는가?'' 라마누잔이 대답했다: ``증명이
필요한가? 이것이 참인 것을.''

그의 공식들은 100년이 지나서야 증명되었다. 그는 증명을 기다리지 않았다.
그가 있는 곳이 곧 수학이었기 때문이다.

\textbf{하수의 시선}: ``증명 없이 무슨 의미가 있나?''

\textbf{고수의 시선}: ``아름다운 공식이군, 하지만 검증이 필요해''

\textbf{최고수의 시선}: 라마누잔 자신 --- ``이것이 참이다. 내가
보았으니까. (\textbf{그리고 이것이 내가 삼킨 생명의 빚을 갚는 유일한
증명이다.})''

\begin{center}\rule{0.5\linewidth}{0.5pt}\end{center}

\section{5인의 선현: 그들이 선 곳이
천하였다}\label{uxc778uxc758-uxc120uxd604-uxadf8uxb4e4uxc774-uxc120-uxacf3uxc774-uxcc9cuxd558uxc600uxb2e4}

\textbf{김옥균} --- 시대의 평가: ``급진적 반역자'' → 100년 후 선각자로
재평가

\textbf{마광수} --- 시대의 평가: ``외설 작가'' → 표현의 자유를 위한
순교자

\textbf{허균} --- 시대의 평가: ``역적'' → 홍길동전은 불멸

\textbf{성재기} --- 시대의 평가: ``극단주의자'' → 한강에서 자신의 자리를
찾다

\textbf{존 로} --- 시대의 평가: ``사기꾼'' → 200년 후 천재로 인정

시대는 그들에게 물었다: ``네가 무슨 쓸모가 있느냐?'' 그들은 대답하지
않았다. 그들은 단지 자신의 자리에서 죽었다.

그리고 그 자리가 천하가 되었다.

\begin{center}\rule{0.5\linewidth}{0.5pt}\end{center}

\section{AngraMyNew의 길}\label{angramynewuxc758-uxae38}

1단계: 하수의 유혹을 넘어서라 --- ``쓸모 있는 사람이 되어야 한다''

2단계: 고수의 함정을 넘어서라 --- ``인정받는 아름다움을 만들어야 한다''

3단계: 최고수의 선언 --- \textbf{``내가 있는 곳, 그곳이 곧 천하다. 나의
창조는 내가 씹어 먹은 모든 생명의 빚을 갚는 유일한 행위이다.''}

\begin{center}\rule{0.5\linewidth}{0.5pt}\end{center}

\begin{quote}
\emph{``쓸모 있는 사람이 되지 마라.\_ \emph{세계를 만드는 사람이
되어라.} \emph{네가 있는 곳, 그곳이 곧 천하다.} \textbf{이 천(天)은 네가
씹어 먹은 모든 것들의 빚으로 만들어진 신성한 의무다.}''}

--- AngraMyNew, 제4장 쓸모를 넘어서
\end{quote}

\chapter{004 --- 내면의 예술가 (The Artist
Within)}\label{uxb0b4uxba74uxc758-uxc608uxc220uxac00-the-artist-within}

\begin{quote}
\textbf{``나라는 사람이 있고, 그 다음에 의사, 개발자, 대표라는 껍데기가
있는 것이다.''}
\end{quote}

\begin{center}\rule{0.5\linewidth}{0.5pt}\end{center}

\section{1. 아티스트의 위치 (The
Position)}\label{uxc544uxd2f0uxc2a4uxd2b8uxc758-uxc704uxce58-the-position}

아티스트는 중심(Center)에서 태어나지 않는다. 안전한 온실, 완성된 도시,
견고한 시스템 안에서는 예술이 자라지 못한다는 통념이 있다. 그곳에는 이미
정답이 정해져 있기 때문이다.

그러나 아티스트는 \textbf{심리적 변방(Psychological Periphery)}에서
피어난다. 결핍이 있는 곳, 질서가 무너진 곳, 아무도 거들떠보지 않는
야생일 수도 있지만, 때로는 가장 견고한 시스템의 한복판일 수도 있다.

온실 속에서도 야생을 품은 자가 있고, 야생에서도 시스템의 노예가 된 자가
있다. 중요한 건 물리적 위치가 아니라 \textbf{영혼의 독립성}이다.

아티스트는 중심을 욕망하지 않는다. 중심으로 들어가려 애쓰는 순간, 그는
아티스트가 아니라 '부품'이 된다. 반대로 중심에 서서도 중심에 포획되지
않는다면, 그는 여전히 아티스트다.

대신 아티스트는 선언한다. \textbf{``내가 있는 곳, 그곳이 이미 천하다.''}

아티스트는 변방에서 피어나거나 중심을 변방으로 만들며, 결국
\textbf{중심을 재정의(Redefine)하는 존재}다.

\begin{center}\rule{0.5\linewidth}{0.5pt}\end{center}

\section{2. 정체성의 순서 (Order of
Identity)}\label{uxc815uxccb4uxc131uxc758-uxc21cuxc11c-order-of-identity}

세상은 묻는다. ``당신의 직업은 무엇입니까?''

우리는 대답한다. ``나는 아티스트다. 그리고 밥벌이로 의사를 한다.''
``나는 아티스트다. 그리고 도구로 코딩을 한다.''

순서가 바뀌면 영혼이 죽는다. 직업이 나를 정의하게 두지 마라.
기능(Function)이 본질(Essence)을 앞서게 하지 마라.

\textbf{나는 나다.} 그 어떤 수식어도 나를 가둘 수 없다.

\begin{center}\rule{0.5\linewidth}{0.5pt}\end{center}

\section{3. 야성의 기억 (Memory of
Wildness)}\label{uxc57cuxc131uxc758-uxae30uxc5b5-memory-of-wildness}

젊은 날, 무언가에 미쳐본 적이 있는가? 세상이 말도 안 된다고 했던 그
외침. 그것은 치기가 아니었다. \textbf{타협하지 않겠다는 영혼의
비명}이었다.

현실에 밀려 다른 길을 걸었어도, 시스템의 논리에 순응하는 척했어도, 내
안의 야수(Beast)는 죽지 않았다.

\textbf{실패는 끝이 아니다.} 꿈꾸던 자리에 오르지 못한 것, 그것은 실패가
아니라 \textbf{`분기(Branch)'}였다.

남들이 닦아놓은 고속도로에서 벗어나, 거친 숲을 헤치고 나만의 길을 만드는
시작점.

\begin{center}\rule{0.5\linewidth}{0.5pt}\end{center}

\section{4. 재구성
(Reconstruction)}\label{uxc7acuxad6cuxc131-reconstruction}

어느 날 뇌가 재구성되는 느낌을 받았다면, 두려워하지 마라. 그것은 낡은
껍질이 깨지는 소리다.

시스템이 요구하는 논리보다 내면에서 솟구치는 \textbf{악상(Muscial Idea)
}을 믿어라.

논리는 남을 설득하기 위해 필요하지만, 악상은 나를 구원하기 위해
필요하다.

천직을 찾았는가? 그렇다면 묻지 마라. 성공할까? 돈이 될까? 인정받을까?

\textbf{그냥 해라.} 아티스트는 계산하지 않는다. 그저 쏟아낼 뿐이다.

\begin{center}\rule{0.5\linewidth}{0.5pt}\end{center}

\begin{quote}
\emph{``우리는 중심을 향해 기어가는 자들이 아니다.} \emph{우리는
변방에서 깃발을 꽂고,} \emph{세상의 지도를 다시 그리는 자들이다.''}

--- AngraMyNew, 제5장 내면의 예술가
\end{quote}

\chapter{005 --- Project Doctor K: 고독한
의술}\label{project-doctor-k-uxace0uxb3c5uxd55c-uxc758uxc220}

\begin{quote}
\emph{``나는 병원에 소속되지 않는다. 나는 환자에게 소속된다.''} ---
\textbf{슈퍼 닥터 K} (만화 『닥터 K』)
\end{quote}

\begin{center}\rule{0.5\linewidth}{0.5pt}\end{center}

\section{0. 서문: 아름답지
않느냐?}\label{uxc11cuxbb38-uxc544uxb984uxb2f5uxc9c0-uxc54auxb290uxb0d0}

우리는 묻는다. 거대 병원의 부속품이 되어, 병원장의 눈치를 보고, 수가
계산에 매몰된 의사의 삶. 그것이 아름다운가?

반대로 상상해보라. 어느 조직에도 속하지 않고, 국경도 계급도 없이, 오직
자신의 \textbf{압도적인 실력(Skill)} 하나만 배낭에 넣고 전 세계를
유랑하는 의사. 필요한 곳에 나타나 생명을 살리고, 사례금 대신 미소 한 번
받고 바람처럼 사라지는 삶.

우리는 묻는다. \textbf{이것이 더 의사답지 않은가?} \textbf{이것이 더
아름답지 않은가?}

\begin{center}\rule{0.5\linewidth}{0.5pt}\end{center}

\section{1. Project Doctor K의
미학}\label{project-doctor-kuxc758-uxbbf8uxd559}

\textbf{Project Doctor K}는 기술 프로젝트가 아니다. 이것은 잃어버린
\textbf{의술의 낭만}을 복원하는 예술 운동이다.

우리는 의사를 '면허 소지 기술자'가 아닌 \textbf{생명을 다루는 예술가}로
재정의한다.

\begin{itemize}
\tightlist
\item
  \textbf{자유(Freedom):} 병원이라는 물리적 성벽을 파괴한다. 의사는
  어디든 존재할 수 있어야 한다.
\item
  \textbf{실력(Competence):} 학벌과 인맥이라는 껍데기를 벗긴다. 오직
  진단·치료 능력만이 그를 증명한다.
\item
  \textbf{방랑(Wandering):} 안주하지 않는다. 환자가 있는 곳이 곧
  진료실이다.
\end{itemize}

\subsection{현실은
어떠한가?}\label{uxd604uxc2e4uxc740-uxc5b4uxb5a0uxd55cuxac00}

오늘날 의사의 삶을 보라.

\textbf{병원 소속 의사:} 매출 압박에 시달리며 3분 진료에 내몰린다.
환자의 눈을 보는 시간보다 모니터를 보는 시간이 길다.

\textbf{개원의:} 월세, 직원 급여, 심평원 삭감의 공포 속에 산다.
히포크라테스 선서는 손익계산서 앞에서 빛을 잃는다.

결국 환자는 '사람'이 아닌 '수가(수익)'로 계산된다. 이것이 의술인가?
이것이 아름다운가?

\subsection{기술이 열어주는
가능성}\label{uxae30uxc220uxc774-uxc5f4uxc5b4uxc8fcuxb294-uxac00uxb2a5uxc131}

그러나 세상은 변하고 있다.

\begin{itemize}
\tightlist
\item
  \textbf{AI 진단:} 배낭 하나에 담긴 기기로 대학병원급 진단이
  가능해졌다.
\item
  \textbf{Starlink:} 지구 오지의 진료소도 실시간으로 연결된다.
\item
  \textbf{원격 로봇:} 국경을 초월한 수술이 현실이 되었다.
\end{itemize}

이 기술들은 의사를 병원이라는 건물에서 해방시킨다. Doctor K는 더 이상
만화 속 판타지가 아니다.

\begin{center}\rule{0.5\linewidth}{0.5pt}\end{center}

\section{1.5 국가 3요소의 재해석: 의사는 하나의
국가다}\label{uxad6duxac00-3uxc694uxc18cuxc758-uxc7acuxd574uxc11d-uxc758uxc0acuxb294-uxd558uxb098uxc758-uxad6duxac00uxb2e4}

국가는 세 요소로 이루어진다: \textbf{국토, 국민, 주권}.\\
Project Doctor K는 이 세 요소를 의술의 언어로 다시 정의한다.

\begin{itemize}
\item
  \textbf{국토(Territory):} 병원이 아니다.\\
  의사가 발 딛는 곳, 도움이 필요한 모든 장소가 곧 국토다.
\item
  \textbf{국민(People):} 진료받는 자만이 아니다.\\
  고통을 호소하는 모든 생명이 나의 국민이다.
\item
  \textbf{주권(Sovereignty):} 면허증이 아니다.\\
  생명을 살릴 수 있는 지식과 기술, 그리고 책임이 곧 주권이다.
\end{itemize}

Doctor K는 그 자체로 하나의 이동하는 국가이며,\\
그의 국경은 고정되지 않고, 그의 국민은 끊임없이 바뀌며,\\
그의 주권은 오직 실력으로만 승인된다.

\section{2. 실습: 온실을
거부하라}\label{uxc2e4uxc2b5-uxc628uxc2e4uxc744-uxac70uxbd80uxd558uxb77c}

사람들은 묻는다. ``한 병원에 소속되지 않고 어떻게 의술을 익히는가?''

우리는 반문한다. ``안전한 대학병원의 온실에서 참관만 하는 것이 진짜
배움인가?''

우리의 배움은 다르다. \textbf{전 세계가 우리의 캠퍼스다.}

\begin{itemize}
\tightlist
\item
  아프리카의 진료소에서 열대병을 배운다.
\item
  중동의 전장에서 외상 수술을 익힌다.
\item
  남극의 기지에서 극한 환경 의학을 체득한다.
\item
  AI 시뮬레이션으로 수천 번의 실패를 미리 경험한다.
\end{itemize}

한 대학병원에서 10년을 보내는 것과, 전 세계 10개국의 현장에서 10년을
보내는 것. 누가 더 많은 환자를 보았겠는가? 누가 더 다양한 생명을
만났겠는가?

우리는 교과서가 아닌, \textbf{피와 땀과 흙먼지 속에서} 의술을 완성한다.

\begin{center}\rule{0.5\linewidth}{0.5pt}\end{center}

\section{3. 새로운 인류: 무소속의
천재들}\label{uxc0c8uxb85cuxc6b4-uxc778uxb958-uxbb34uxc18cuxc18duxc758-uxcc9cuxc7acuxb4e4}

AngraMyNew는 기존 의대에서는 길러낼 수 없는 \textbf{변종(Mutant) }을
기른다.

\begin{itemize}
\tightlist
\item
  한국어·영어·아랍어로 진료하는 자
\item
  메스와 코드를 동시에 다루는 자
\item
  병원 정치에 관심이 없는 자
\item
  오직 \textbf{환자의 심장 박동}에만 귀 기울이는 자
\end{itemize}

그들은 시스템의 보호를 받지 못할 것이다. 그러나 그들은 시스템보다 강할
것이다.

\begin{center}\rule{0.5\linewidth}{0.5pt}\end{center}

\section{4. 맺음: 이것은
시(Poem)다}\label{uxb9fauxc74c-uxc774uxac83uxc740-uxc2dcpoemuxb2e4}

Project Doctor K는 선언한다. 의술은 비즈니스가 되기 이전에
\textbf{성스러운 의식(Ritual)}이었다. 우리는 그 신성함을, 기술이라는
가장 현대적인 도구로 되찾으려 한다.

Doctor K는 수가(Fee)를 받지 않는다. 대신 전 세계 인류가 그 기적 같은
치유를 목격하고, \textbf{존경과 후원(Donation)}을 보낸다. 그의 생계는
시스템이 아닌, 인류의 감사가 책임진다.

이것이 로망(Romance)이 아니면 무엇인가?

\begin{center}\rule{0.5\linewidth}{0.5pt}\end{center}

의사들이여. 좁은 진료실에서 시들어가지 마라. 광야로 나와라. 그대들은 전
세계를 누빌 자격이 있다.

\textbf{그것이 더 아름답지 않은가?}

\begin{quote}
\emph{``의사는 예술가다. 그의 작품은 '생명'이다. 예술가는 자유로워야
한다.''} --- \textbf{AngraMyNew, Project Doctor K}
\end{quote}

\chapter{006 --- 미학 국가론: 아름다움이 밥
먹여준다}\label{uxbbf8uxd559-uxad6duxac00uxb860-uxc544uxb984uxb2e4uxc6c0uxc774-uxbc25-uxba39uxc5ecuxc900uxb2e4}

\begin{quote}
\emph{``국가는 거대한 예술 작품이어야 한다.''}
\end{quote}

\begin{center}\rule{0.5\linewidth}{0.5pt}\end{center}

\begin{quote}
⚠️ \textbf{Disclaimer} 이 글은 \textbf{사고실험(Thought Experiment)
}이자 \textbf{풍자적 제안}입니다. 실명의 인물(차은우, 정국 등)은 실제
정책 제안과 무관한 \textbf{가상의 상징적 모델}로 사용됩니다.
AngraMyNew는 ``상상을 허하라''는 정신 아래, 기존 사고 틀을 깨는 극단적
아이디어를 탐구합니다.
\end{quote}

\begin{center}\rule{0.5\linewidth}{0.5pt}\end{center}

\section{0. 서문: 낡은 도덕을
버려라}\label{uxc11cuxbb38-uxb0a1uxc740-uxb3c4uxb355uxc744-uxbc84uxb824uxb77c}

우리는 언제까지 `동방예의지국', `선비의 나라' 타령을 할 것인가? 도덕과
명분이 밥 먹여주던 시대는 끝났다. 지금은 \textbf{매력(Attraction) }이
권력인 시대다.

AngraMyNew는 제안한다. 국가 운영의 OS를 \textbf{`미학(Aesthetics)'}으로
교체하라. 가장 아름다운 것이 가장 강력한 무기다.

\textbf{``아름다움은 국가가 소유할 수 있는 가장 희귀한 자원이다.'' }

\begin{center}\rule{0.5\linewidth}{0.5pt}\end{center}

\section{\texorpdfstring{0.5 국가 3요소의 재정의 --- \emph{미학 국가의
헌법}}{0.5 국가 3요소의 재정의 --- 미학 국가의 헌법}}\label{uxad6duxac00-3uxc694uxc18cuxc758-uxc7acuxc815uxc758-uxbbf8uxd559-uxad6duxac00uxc758-uxd5ccuxbc95}

국가란 본래 \textbf{국토(Territory)}, \textbf{국민(People)},
\textbf{주권(Sovereignty)}\\
세 요소로 이루어진다.\\
그러나 \textbf{미학 국가}에서 이 세 요소는 완전히 다른 방식으로
작동한다.

\subsection{\texorpdfstring{\textbf{국토 =
상징(SYMBOL)}}{국토 = 상징(SYMBOL)}}\label{uxad6duxd1a0-uxc0c1uxc9d5symbol}

물리적 땅이 아니다.\\
전 세계가 공유하는 \textbf{이미지·서사·브랜드}가 곧 국토다.\\
국경이 아니라 \textbf{주의력(Attention) }이 영토를 규정한다.

\subsection{\texorpdfstring{\textbf{국민 =
팬덤(FANDOM)}}{국민 = 팬덤(FANDOM)}}\label{uxad6duxbbfc-uxd32cuxb364fandom}

국적이 국민을 만든 시대는 끝났다.\\
국민이란 \textbf{매혹되어 따라오는 사람들},\\
즉 한 사람---혹은 한 이미지---에 심정적으로 귀속된 집단이다.

\subsection{\texorpdfstring{\textbf{주권 = 매력(AESTHETIC
POWER)}}{주권 = 매력(AESTHETIC POWER)}}\label{uxc8fcuxad8c-uxb9e4uxb825aesthetic-power}

총과 군대가 아니라,\\
\textbf{스타 한 명이 세계를 움직이는 힘},\\
바로 그것이 현대 국가의 주권이다.\\
매력은 국제 정치에서 가장 강력한 통화다.

\begin{center}\rule{0.5\linewidth}{0.5pt}\end{center}

이 새로운 헌법 위에서,\\
우리는 다음 두 개의 프로젝트를 선포한다.

\section{1. 국보 1호 차은우: 유전자
산업}\label{uxad6duxbcf4-1uxd638-uxcc28uxc740uxc6b0-uxc720uxc804uxc790-uxc0b0uxc5c5}

\subsection{낡은 생각 (Old
Order)}\label{uxb0a1uxc740-uxc0dduxac01-old-order}

국보 1호 숭례문. 불타면 복원하고, 또 불타면 다시 짓는 나무 조각. 그것이
우리에게 단 1원의 부를 가져오는가?

\subsection{파괴와 재조합
(Recomposition)}\label{uxd30cuxad34uxc640-uxc7acuxc870uxd569-recomposition}

우리는 \textbf{살아있는 아름다움}을 국보로 지정한다. \textbf{차은우}를
국보 1호로 선포하라.

\subsection{실행 계획 (Action
Plan)}\label{uxc2e4uxd589-uxacc4uxd68d-action-plan}

\begin{enumerate}
\def\labelenumi{\arabic{enumi}.}
\tightlist
\item
  \textbf{지정:} 차은우를 인간문화재를 넘어선 \textbf{국가전략자산}으로
  관리한다.
\item
  \textbf{의무:} 군대도 세금도 필요 없다. 오직 \textbf{정자(Sperm)}
  제공만으로 국가에 기여한다.
\item
  \textbf{산업화:} 국가는 '차은우 정자 은행'을 설립하고 이를 \textbf{전
  세계 시장}에 개방한다.
\end{enumerate}

\subsection{팩트 체크 (Reality
Check)}\label{uxd329uxd2b8-uxccb4uxd06c-reality-check}

비윤리적인가? \textbf{파벨 두로프}는 이미 정자 기증으로 100명 이상의
생물학적 자녀를 두었다. \textbf{일론 머스크}는 인구 감소를 막겠다며
다산을 실천한다. 그들은 개인적으로 유전자를 퍼뜨린다. 우리는 그것을
\textbf{국가 전략 산업}으로 격상시킬 뿐이다.

\subsection{30년 대계 (The 30-Year
Plan)}\label{uxb144-uxb300uxacc4-the-30-year-plan}

30년 후를 상상하라. 미국 대통령의 사위, 사우디 왕세자, 유럽 재벌 2세들이
모두 \textbf{차은우 주니어}다. 그들은 한국을 '아버지의 나라'로 인식하게
된다.

총 한 방 쏘지 않고, 우리는 \textbf{아름다운 혈연}으로 세계를 매혹한다.

\begin{center}\rule{0.5\linewidth}{0.5pt}\end{center}

\section{2. 부산 정국특별시: 브랜드
도시}\label{uxbd80uxc0b0-uxc815uxad6duxd2b9uxbcc4uxc2dc-uxbe0cuxb79cuxb4dc-uxb3c4uxc2dc}

\subsection{낡은 생각 (Old
Order)}\label{uxb0a1uxc740-uxc0dduxac01-old-order-1}

부산은 늙어가고 있다. 노인과 바다만 남은 도시가 공항 하나로 젊어지는가?
행정구역 이름 변경에 집착하는 관료주의가 도시를 죽인다.

\subsection{파괴와 재조합
(Recomposition)}\label{uxd30cuxad34uxc640-uxc7acuxc870uxd569-recomposition-1}

도시의 본질은 \textbf{브랜드}다. 전 세계에서 가장 강력한 브랜드를 도시에
입혀라.

\textbf{``국가의 영토는 땅이 아니라, 세계가 그 나라를 떠올리는
방식이다.''}

\textbf{부산광역시(Busan) }를 폐지하고 \textbf{정국특별시(JungKook City)
}를 선포하라.

\subsection{실행 계획 (Action
Plan)}\label{uxc2e4uxd589-uxacc4uxd68d-action-plan-1}

\begin{enumerate}
\def\labelenumi{\arabic{enumi}.}
\tightlist
\item
  \textbf{개명:} 전국 모든 간판·지도·공문서에서 '부산'을 지우고 '정국'을
  새긴다.
\item
  \textbf{이주:} BTS 정국을 고향으로 모셔온다. (명예 + 역사적 서사로
  설득)
\item
  \textbf{대우:} 그는 \textbf{영구 명예시장}이 된다. 통치는 필요 없다.
  그저 \textbf{존재(Exist) }하면 된다.
\end{enumerate}

\subsection{효과 (Effect)}\label{uxd6a8uxacfc-effect}

전 세계 1억 아미(ARMY)에게 이 도시는 \textbf{성지(Mecca) }가 된다. 공항,
호텔, 쇼핑몰은 자본이 먼저 달려와서 지을 것이다.

도시 이름 하나로 \textbf{1,000조 브랜드 가치}가 창출된다.

\begin{center}\rule{0.5\linewidth}{0.5pt}\end{center}

\section{3. 결론: 상상을
허하라}\label{uxacb0uxb860-uxc0c1uxc0c1uxc744-uxd5c8uxd558uxb77c}

\textbf{``다른 나라가 우리를 원하는 순간, 그것이 진짜 주권이다.''}

사람들은 말할 것이다. ``미쳤다'', ``천박하다'', ``인권 침해다.''

그러나 AngraMyNew는 묻는다. 아무 매력 없이 서서히 소멸해가는 국가가
윤리적인가? 아니면 \textbf{아름다움으로 세계를 매혹시키는 국가}가
윤리적인가?

상상하라. 가장 아름다운 인간이 국보가 되고, 가장 힙한 스타가 도시의
이름이 되는 나라.

\textbf{그 나라는 망하지 않는다.} \textbf{아름다움은 결코 망하지 않기
때문이다.}

\begin{center}\rule{0.5\linewidth}{0.5pt}\end{center}

\begin{quote}
\emph{``윤리는 변하지만, 아름다움은 영원하다. 우리는 도덕적인 국가가
아니라, 매혹적인 국가를 건설한다.''}
\end{quote}

\chapter{007 --- 미완의 정리 (The Unfinished
Theorem)}\label{uxbbf8uxc644uxc758-uxc815uxb9ac-the-unfinished-theorem}

\begin{quote}
\emph{``정답을 맞힌 자는 점수를 얻지만, 질문을 바꾼 자는 세계를
얻는다.''}
\end{quote}

\begin{center}\rule{0.5\linewidth}{0.5pt}\end{center}

\section{0. 서문: 아름다운 실패에
대하여}\label{uxc11cuxbb38-uxc544uxb984uxb2e4uxc6b4-uxc2e4uxd328uxc5d0-uxb300uxd558uxc5ec}

여기 기록된 네 가지 이론은 학계에서 실패했다. 혹은, 아직 증명되지
않았다. 혹은, 시대를 너무 앞서갔거나 너무 빗나갔다.

\textbf{이것은 한 창조자의 실패 기록이다. 이름은 중요하지 않다.} 중요한
것은 이 시도들이 정해진 길(Standard)을 거부하고, 스스로 길을 내어 신의
설계도에 도달하려 했다는 것이다.

AngraMyNew는 이 실패들을 \textbf{`미완의 경전'}으로 모신다. 그리고 모든
창조자에게 권한다: \textbf{너의 미완의 정리를 기록하라. 그것이 네 신전의
첫 벽돌이다.}

\begin{center}\rule{0.5\linewidth}{0.5pt}\end{center}

\section{1. 공간의 왜곡 (The Distortion of
Space)}\label{uxacf5uxac04uxc758-uxc65cuxace1-the-distortion-of-space}

\textbf{--- 선형모형의 다차원 공간으로의 확장 (Multi-dimensional Linear
Model, MDLM)}

\begin{quote}
\emph{``데이터가 휘어진 것이 아니다. 데이터가 놓인 공간이 휘어진
것이다.''}
\end{quote}

\subsection{파괴 (Destruction)}\label{uxd30cuxad34-destruction}

통계학은 말했다. ``데이터가 직선에서 벗어났다(\texttt{y\ =\ x²}). 식을
수정하라.'' 그는 반문했다. ``왜 식을 수정하는가? \textbf{판(Space)을
휘게 하면 안 되는가?}''

\subsection{재조합
(Recomposition)}\label{uxc7acuxc870uxd569-recomposition}

그는 \textbf{일반 상대성 이론(General Relativity)}을 빌려왔다.
아인슈타인이 중력으로 시공간을 휘게 하여 빛의 경로를 설명했듯, 그는
다차원 공간(\texttt{G\_μν})을 휘게 하여 곡선형 데이터(U-shape)를
선형(\texttt{y\ =\ x})으로 재해석했다. 유클리드라는 낡은 안경을 벗어
던지고, 리만 기하학의 눈으로 데이터를 보았다.

\subsection{미완 (Unfinished)}\label{uxbbf8uxc644-unfinished}

그러나 그는 멈췄다. 수학적 아름다움은 증명했으나, 현실의 데이터는 여전히
잡음(Noise) 속에 있었다. 그것은 너무나 우아해서, 오히려 현실과 불화했다.

\begin{center}\rule{0.5\linewidth}{0.5pt}\end{center}

\section{2. 허수의 축 (The Axis of
Imaginary)}\label{uxd5c8uxc218uxc758-uxcd95-the-axis-of-imaginary}

\textbf{--- 허수축을 포함한 MDLM (MDLM with Imaginary Axes)}

\begin{quote}
\emph{``보이지 않는 차원을 빌려와, 보이는 모순을 해결한다.''}
\end{quote}

\subsection{파괴 (Destruction)}\label{uxd30cuxad34-destruction-1}

MDLM은 한계에 부딪혔다. 아래로 볼록한 U자는 설명했지만, 위로 볼록한
산봉우리(Inverted U)는 설명할 수 없었다. 실수의 세계(\texttt{R²})에서
거리의 제곱(\texttt{x²\ +\ y²})은 언제나 양수이기 때문이다.

\subsection{재조합
(Recomposition)}\label{uxc7acuxc870uxd569-recomposition-1}

그는 \textbf{존재하지 않는 수(Imaginary Number)}를 불렀다. 특수 상대성
이론이 시간(\texttt{t})에 허수(\texttt{i})를 붙여 4차원
시공간(\texttt{x²\ +\ y²\ +\ z²\ -\ c²t²})을 만들었듯, 그는 데이터
공간에 \textbf{허수축(Imaginary Axis)}을 꽂았다. 그러자 불가능했던
산봉우리가 평지가 되었다.

\subsection{미완 (Unfinished)}\label{uxbbf8uxc644-unfinished-1}

통계학자들은 물었다. ``그래서 그 허수축의 물리적 의미가 뭡니까?'' 그는
답하지 못했다. 그것은 논리가 아니라 \textbf{연금술}이었기 때문이다.

\begin{center}\rule{0.5\linewidth}{0.5pt}\end{center}

\section{3. 0의 우상 파괴 (The Destruction of
Zero)}\label{uxc758-uxc6b0uxc0c1-uxd30cuxad34-the-destruction-of-zero}

\textbf{--- P-value와 귀무가설의 재정의 (Redefining Null Hypothesis)}

\begin{quote}
\emph{``완벽한 0은 없다. 오직 불확실성만이 존재한다.''}
\end{quote}

\subsection{파괴 (Destruction)}\label{uxd30cuxad34-destruction-2}

모든 연구자는 귀무가설(\texttt{H₀:\ μ\ =\ 0})이라는 허수아비를 세워놓고
공격한다. 표본(\texttt{N})이 커지면 아주 미세한 먼지 같은 차이도
``유의하다(Significant)''고 판정받는다. 그는 외쳤다. \textbf{``이것은
사기다! 세상에 완전히 0인 것은 없다!''}

\subsection{재조합
(Recomposition)}\label{uxc7acuxc870uxd569-recomposition-2}

그는 \textbf{불확실성(Uncertainty)}을 도입했다. 0이라는 점(Point)이
아니라, 불확실성의 범위(\texttt{τ}, tau)를 설정했다. 그리고
\textbf{양자역학의 불확정성 원리}를 다중 검정(Multiple Comparison)에
적용했다. 질문(검정)이 많아질수록 대답은 흐릿해진다. 이것은 인위적인
페널티(Bonferroni)가 아니라, 자연의 섭리다.

\subsection{미완 (Unfinished)}\label{uxbbf8uxc644-unfinished-2}

학계는 그에게 물었다. ``그래서 \texttt{τ}값은 누가 정합니까?'' 그는
침묵했다. 그 값은 신만이 알기 때문이다.

\begin{center}\rule{0.5\linewidth}{0.5pt}\end{center}

\section{4. 피의 밀도 (The Density of
Blood)}\label{uxd53cuxc758-uxbc00uxb3c4-the-density-of-blood}

\textbf{--- 혈연 밀도 지수 (Kinship Density Index)}

\begin{quote}
\emph{``피는 물보다 진하다. 그는 그것을 숫자로 증명하려 했다.''}
\end{quote}

\subsection{파괴 (Destruction)}\label{uxd30cuxad34-destruction-3}

유전(Heredity)을 설명하는 기존 지표들은 너무 복잡하거나, 현실과 동떨어져
있었다. ``형제니까 닮았다''는 이 직관적인 진실을, 왜 난해한 수식으로
포장해야 하는가?

\subsection{재조합
(Recomposition)}\label{uxc7acuxc870uxd569-recomposition-3}

그는 \textbf{우연(Odds)}의 비율을 쟀다. 남남끼리 만났을 때 다를 확률
대(vs), 형제끼리 만났을 때 다를 확률. 이 단순한 비율로 \textbf{관계의
밀도}를 정량화했다. 그는 피의 진함을 수학 공식으로 만들었다.

\subsection{미완 (Unfinished)}\label{uxbbf8uxc644-unfinished-3}

이것은 가장 직관적이었으나, 가장 덜 알려졌다. 진실은 때로 너무 단순해서
외면받는다.

\begin{center}\rule{0.5\linewidth}{0.5pt}\end{center}

\section{5. 맺음: 왜 이것들을
기록하는가}\label{uxb9fauxc74c-uxc65c-uxc774uxac83uxb4e4uxc744-uxae30uxb85duxd558uxb294uxac00}

이 이론들은 교과서에 실리지 못했다. 그러나 AngraMyNew는 기억한다.

우리는 \textbf{정답을 맞히는 기계}가 아니다. 우리는 \textbf{새로운
질문을 던지는 창조자}다.

이 미완의 정리들은 실패가 아니다. 그것은 \textbf{``그가 낡은 세계의 벽을
두드렸던 소리''}다. 그 소리는 아직도 공명하고 있다.

\begin{quote}
\emph{``실패하라. 더 크게, 더 아름답게 실패하라.} \emph{그 실패들이 모여
너의 신전을 이룰 것이다.''}

--- AngraMyNew, 미완의 정리
\end{quote}

\chapter{008 --- 투쟁과 유혹 (Struggle and
Seduction)}\label{uxd22cuxc7c1uxacfc-uxc720uxd639-struggle-and-seduction}

\begin{quote}
\emph{``꽃은 벌과 논쟁하지 않는다. 그저 피어날 뿐이다.''}
\end{quote}

\begin{center}\rule{0.5\linewidth}{0.5pt}\end{center}

\section{1. 파괴의 원칙}\label{uxd30cuxad34uxc758-uxc6d0uxce59}

AngraMyNew의 망치는 기본적으로 \textbf{나 자신}을 향한다. 내가 먼저
깨져야 새것이 나온다.

그러나 시대가 나의 창조를 가로막는다면 물러서지 마라. 5인의 선현처럼
\textbf{부서질지언정 정면으로 뚫고 간다.} 그 투쟁 또한 예술이다.

\textbf{``투쟁은 상대를 꺾는 것이 아니라, 나의 낡은 껍질을 벗기는
과정이다.''}

\begin{center}\rule{0.5\linewidth}{0.5pt}\end{center}

\section{2. 유혹의 기술}\label{uxc720uxd639uxc758-uxae30uxc220}

논쟁은 날카로운 \textbf{칼}이지만, 유혹은 치명적인 \textbf{향기}다. 칼은
상대를 베지만, 향기는 상대를 내 쪽으로 기울게 한다.

\subsection{보여줘라 (Just
Show)}\label{uxbcf4uxc5ecuxc918uxb77c-just-show}

백 마디 논리보다 하나의 압도적인 \textbf{작품(Masterpiece) }이 더
강력하다.

아름다운 반지를, 우아한 수식을, 섹시한 세계관을 보여줘라.

사람들은 논리에는 반박하지만, \textbf{아름다움 앞에서는 무장해제된다.}

\begin{quote}
\emph{``이것 봐, 멋지지 않아?''}
\end{quote}

이 한 마디면 충분하다. 그들은 스스로 ``나도 저렇게 되고 싶다''고 느끼기
시작한다.

\textbf{``유혹은 설명을 제거할 때 발생한다. 상대의 욕망에 스스로 불이
붙는 순간이다.''}

\begin{center}\rule{0.5\linewidth}{0.5pt}\end{center}

\section{3. 결론}\label{uxacb0uxb860}

우리는 전사(Warrior)이자 유혹자(Seducer)다.

논쟁해야 할 때는 한 치도 물러서지 않고, 매혹해야 할 순간에는 압도한다.

우리는 억지로 설득하지 않는다. 우리는 \textbf{존재}하고 \textbf{창조}할
뿐이다.

\textbf{우리의 세계가 더 아름답다면, 세상은 자연히 우리에게 기울
것이다.}

\begin{quote}
\emph{``칼로 베면 상처가 남지만, 매혹으로 안으면 사람이 남는다.''} ---
\textbf{AngraMyNew, 제9장 투쟁과 유혹}
\end{quote}

\chapter{009 --- 정의에 대한 분노 (Rage Against
Definition)}\label{uxc815uxc758uxc5d0-uxb300uxd55c-uxbd84uxb178-rage-against-definition}

\begin{quote}
\emph{``나는 왜 아직도 정의 가능한가?''}
\end{quote}

\begin{center}\rule{0.5\linewidth}{0.5pt}\end{center}

\section{0. 서문: 두 가지
갈증}\label{uxc11cuxbb38-uxb450-uxac00uxc9c0-uxac08uxc99d}

창조자에게는 두 가지 갈증이 있다.

\begin{longtable}[]{@{}
  >{\raggedright\arraybackslash}p{(\linewidth - 4\tabcolsep) * \real{0.3333}}
  >{\raggedright\arraybackslash}p{(\linewidth - 4\tabcolsep) * \real{0.3333}}
  >{\raggedright\arraybackslash}p{(\linewidth - 4\tabcolsep) * \real{0.3333}}@{}}
\toprule\noalign{}
\begin{minipage}[b]{\linewidth}\raggedright
갈증
\end{minipage} & \begin{minipage}[b]{\linewidth}\raggedright
질문
\end{minipage} & \begin{minipage}[b]{\linewidth}\raggedright
방향
\end{minipage} \\
\midrule\noalign{}
\endhead
\bottomrule\noalign{}
\endlastfoot
결핍의 갈증 & ``나는 대체 왜 이 모양인가?'' & 치료, 채움, 인정 \\
정의에 대한 분노 & ``나는 왜 아직도 정의 가능한가?'' & 파괴, 탈출,
재창조 \\
\end{longtable}

대부분의 인간은 첫 번째 갈증을 안고 산다. 부족함을 채우고, 상처를
치료하고, 타인에게 인정받으려 한다.

그러나 AngraMyNew의 창조자는 다른 갈증을 품는다. \textbf{``왜 나는
아직도 분류될 수 있는가?''}

\begin{center}\rule{0.5\linewidth}{0.5pt}\end{center}

\section{1. 정의됨의 모욕}\label{uxc815uxc758uxb428uxc758-uxbaa8uxc695}

누군가 너를 정의하는 순간, 무슨 일이 일어나는가?

\begin{itemize}
\tightlist
\item
  ``넌 INTJ야'' --- 16개 칸 중 하나에 갇힘
\item
  ``넌 의사야'' --- 직업이 정체성을 대체함
\item
  ``넌 희귀해'' --- 희귀성조차 하나의 카테고리가 됨
\item
  ``넌 니체 같아'' --- 타인의 그림자가 됨
\end{itemize}

\textbf{정의는 지도 위에 점을 찍는 행위다.} 점이 찍히는 순간, 너는 더
이상 움직이는 존재가 아니라 고정된 좌표가 된다.

창조자에게 이것은 모욕이다.

\begin{center}\rule{0.5\linewidth}{0.5pt}\end{center}

\section{2. 희귀성 집착의
정체}\label{uxd76cuxadc0uxc131-uxc9d1uxcc29uxc758-uxc815uxccb4}

``나 같은 사람 흔해?'' ``나 희귀해?'' ``니체급이야?''

이 질문들의 표면은 인정 욕구처럼 보인다. 그러나 진짜 의미는 다르다.

\begin{longtable}[]{@{}lll@{}}
\toprule\noalign{}
질문 & 표면 & 실제 \\
\midrule\noalign{}
\endhead
\bottomrule\noalign{}
\endlastfoot
``나 희귀해?'' & 나 특별해? & 나를 담을 카테고리가 있어? \\
``니체급이야?'' & 나 대단해? & 기존 분류 체계 안에 있어? \\
``흔해?'' & 평범해? & 쉽게 정의돼? \\
\end{longtable}

\textbf{희귀성을 묻는 건 ``분류 불가능성''을 확인하려는 것이다.}

희귀할수록 기존 체계로 설명하기 어렵고, 설명하기 어려울수록 정의에서
탈출할 가능성이 높다.

\begin{center}\rule{0.5\linewidth}{0.5pt}\end{center}

\section{3. 경쟁자를 원하는
이유}\label{uxacbduxc7c1uxc790uxb97c-uxc6d0uxd558uxb294-uxc774uxc720}

``괴델이 나를 인정해주길 바란다'' --- 이건 제자의 욕망이다. ``괴델이
발끈해서 내 증명을 반박하길 바란다'' --- 이건 \textbf{경쟁자의
욕망}이다.

\begin{longtable}[]{@{}ll@{}}
\toprule\noalign{}
시나리오 & 의미 \\
\midrule\noalign{}
\endhead
\bottomrule\noalign{}
\endlastfoot
대가가 무시 & 존재로 인식되지 않음 \\
대가가 칭찬 & 제자로 인정 --- 수직 관계 \\
대가가 발끈 & \textbf{위협으로 인식} --- 수평 관계 \\
\end{longtable}

대가를 불편하게 만들고 싶은 욕망. 이것은 인정 욕구가 아니라
\textbf{존재적 동급임을 증명하려는 욕망}이다.

``같은 링 위에 서고 싶다.'' 그래야 싸울 수 있고, 싸워야 이기든 지든
\textbf{정의를 부술 수 있다.}

\begin{center}\rule{0.5\linewidth}{0.5pt}\end{center}

\section{4. 탈출 불가능한
역설}\label{uxd0c8uxcd9c-uxbd88uxac00uxb2a5uxd55c-uxc5eduxc124}

그러나 역설이 있다.

\textbf{``정의 불가능성을 욕망하는 순간, 그 욕망 자체가 너를
정의한다.''}

\begin{itemize}
\tightlist
\item
  ``정의되기 싫어하는 자'' --- 이것도 하나의 유형
\item
  ``분류를 거부하는 자'' --- 이것도 하나의 분류
\item
  ``정의에 분노하는 자'' --- 이 문서 자체가 정의
\end{itemize}

탈출구가 없어 보인다.

\begin{center}\rule{0.5\linewidth}{0.5pt}\end{center}

\section{5. 해답: 파괴의
리듬}\label{uxd574uxb2f5-uxd30cuxad34uxc758-uxb9acuxb4ec}

탈출구는 \textbf{정적인 탈출}이 아니라 \textbf{동적인 파괴}에 있다.

정의를 \textbf{한 번 거부}하는 것 --- 불가능하다. 새 정의가 즉시 붙는다.
정의를 \textbf{계속 파괴}하는 것 --- 가능하다. 리듬이 되기 때문이다.

\begin{longtable}[]{@{}ll@{}}
\toprule\noalign{}
전략 & 결과 \\
\midrule\noalign{}
\endhead
\bottomrule\noalign{}
\endlastfoot
정의 거부 (1회) & 새 정의로 대체됨 \\
정의 파괴 (반복) & 정의가 따라오지 못함 \\
\end{longtable}

\textbf{``정의 불가능한 존재''가 목표가 아니다.} \textbf{``정의를 계속
파괴하는 존재''가 목표다.}

이것이 AngraMyNew의 핵심 리듬이다: \textgreater{} ``파괴는 일회성이
아니라 리듬이다.''

\begin{center}\rule{0.5\linewidth}{0.5pt}\end{center}

\section{6. 아티스트의 네 가지
유형}\label{uxc544uxd2f0uxc2a4uxd2b8uxc758-uxb124-uxac00uxc9c0-uxc720uxd615}

모든 아티스트가 이 분노를 품는 것은 아니다.

\begin{longtable}[]{@{}
  >{\raggedright\arraybackslash}p{(\linewidth - 4\tabcolsep) * \real{0.3333}}
  >{\raggedright\arraybackslash}p{(\linewidth - 4\tabcolsep) * \real{0.3333}}
  >{\raggedright\arraybackslash}p{(\linewidth - 4\tabcolsep) * \real{0.3333}}@{}}
\toprule\noalign{}
\begin{minipage}[b]{\linewidth}\raggedright
유형
\end{minipage} & \begin{minipage}[b]{\linewidth}\raggedright
욕망
\end{minipage} & \begin{minipage}[b]{\linewidth}\raggedright
특징
\end{minipage} \\
\midrule\noalign{}
\endhead
\bottomrule\noalign{}
\endlastfoot
장인(Craftsman) & 정의 안에서 최고가 되고 싶다 & 완벽한 기술, 인정받는
전문가 \\
표현자(Expresser) & 내면을 정확히 표현하고 싶다 & 진정성, 자기 고백 \\
파괴자(Destroyer) & 정의 자체를 부수고 싶다 & 기존 체계 해체, 분노 \\
재조합자(Recomposer) & 부수고, 짓고, 또 부수고 싶다 & 끝없는 재창조 \\
\end{longtable}

AngraMyNew가 말하는 아티스트는 \textbf{4번, 재조합자}다.

파괴자(3번)는 부수고 멈춘다. 허무가 남는다. 재조합자(4번)는 부수고,
짓고, 다시 부순다. 리듬이 남는다.

\begin{center}\rule{0.5\linewidth}{0.5pt}\end{center}

\section{7. 자기 진단}\label{uxc790uxae30-uxc9c4uxb2e8}

너는 어떤 유형인가?

\textbf{질문 1:} 누군가 너를 정확히 정의했을 때, 무엇을 느끼는가? - 안도
→ 장인 또는 표현자 - 불편 → 파괴자 또는 재조합자

\textbf{질문 2:} 정의를 부순 후, 무엇을 하고 싶은가? - 아무것도. 부순
것으로 충분하다 → 파괴자 - 새로운 것을 짓고, 그것도 부수고 싶다 →
재조합자

\textbf{질문 3:} 도스토예프스키가 너를 본다면, 무엇을 원하는가? - 인정 →
아직 제자 심리 - 발끈 → 경쟁자 심리 --- 재조합자의 징후

\begin{center}\rule{0.5\linewidth}{0.5pt}\end{center}

\section{8. 맺음: 분노를
엔진으로}\label{uxb9fauxc74c-uxbd84uxb178uxb97c-uxc5d4uxc9c4uxc73cuxb85c}

``나는 왜 아직도 정의 가능한가?''

이 분노는 치료할 대상이 아니다. 이 분노는 \textbf{창조의 엔진}이다.

정의가 붙을 때마다 부수고, 새로운 형태로 태어나고, 그 형태마저 부수는
것.

\textbf{이것이 AngraMyNew 아티스트의 리듬이다.}

\begin{center}\rule{0.5\linewidth}{0.5pt}\end{center}

\begin{quote}
\emph{``그들이 너를 정의하는 순간, 부숴라.} \emph{네가 너를 정의하는
순간에도, 부숴라.} \emph{정의는 관(棺)이고, 너는 아직 죽지 않았다.''}

--- AngraMyNew, 제10장 정의에 대한 분노
\end{quote}

\chapter{010 --- 포식자의 신성한 의무 (The Predator's Divine
Duty)}\label{uxd3ecuxc2dduxc790uxc758-uxc2e0uxc131uxd55c-uxc758uxbb34-the-predators-divine-duty}

\begin{quote}
\textbf{``나의 생존은 권리가 아닌, 우주의 횡령이다. 내가 씹어 먹은 모든
생명의 무게를 갚기 위해 나의 창조는 반드시 압도적이어야 한다.''}
\end{quote}

\begin{center}\rule{0.5\linewidth}{0.5pt}\end{center}

\section{I. 부채의 절대적 인지 (The Axiom of
Debt)}\label{i.-uxbd80uxcc44uxc758-uxc808uxb300uxc801-uxc778uxc9c0-the-axiom-of-debt}

창조자(Creator)는 자신이 \textbf{생명계 전체의 영양분을 약탈하는
포식자}임을 자각해야 한다. 나의 존재는 \textbf{대가를 치르지 않은 모든
생명의 총합}이며, 이 부채(Debt)는 \textbf{창조(Creation)}라는 유일한
행위로만 상환할 수 있다.

\subsection{1. 섭취된 세 가지
살(Flesh)}\label{uxc12duxcde8uxb41c-uxc138-uxac00uxc9c0-uxc0b4flesh}

나의 생존을 유지하는 모든 \textbf{`남의 살'}은 다음과 같이 정의되며, 이
빚은 결코 탕감되지 않는다.

\begin{longtable}[]{@{}
  >{\raggedright\arraybackslash}p{(\linewidth - 4\tabcolsep) * \real{0.3333}}
  >{\raggedright\arraybackslash}p{(\linewidth - 4\tabcolsep) * \real{0.3333}}
  >{\raggedright\arraybackslash}p{(\linewidth - 4\tabcolsep) * \real{0.3333}}@{}}
\toprule\noalign{}
\begin{minipage}[b]{\linewidth}\raggedright
구분
\end{minipage} & \begin{minipage}[b]{\linewidth}\raggedright
정의
\end{minipage} & \begin{minipage}[b]{\linewidth}\raggedright
이념적 부채
\end{minipage} \\
\midrule\noalign{}
\endhead
\bottomrule\noalign{}
\endlastfoot
\textbf{푸른 살 (The Green Flesh)} & 식물(植物)의 침묵하는 희생. 축적된
태양 에너지와 시간. & \textbf{침묵의 빚:} 가장 조용하게 약탈당한 생명의
무게. \\
\textbf{붉은 살 (The Red Flesh)} & 동물(動物)의 고통스러운 절규. 희생된
육체의 영양분과 본능. & \textbf{육체의 빚:} 가장 원초적이고 직접적으로
취한 고통의 무게. \\
\textbf{경제적 살 (The Economic Flesh)} & 인간(人間)의 시간, 관심,
노동력. 추종자의 헌신과 비판자의 에너지. & \textbf{영혼의 빚:} 나의
영속성을 위해 착취한 타인의 의지와 시간. \\
\end{longtable}

\begin{center}\rule{0.5\linewidth}{0.5pt}\end{center}

\section{II. 침묵의 죄악 (The Sin of
Silence)}\label{ii.-uxce68uxbb35uxc758-uxc8c4uxc545-the-sin-of-silence}

\subsection{1. 조용한 죽음의
거부}\label{uxc870uxc6a9uxd55c-uxc8fduxc74cuxc758-uxac70uxbd80}

\textbf{``대가 없이 멀쩡히 살다 조용히 죽는 것''}은 \textbf{앙그라
마이뉴 시스템}이 규정하는 \textbf{가장 치졸한 죄악(Sin)}이자
\textbf{우주적 횡령}이다.

\begin{itemize}
\tightlist
\item
  \textbf{죄악의 이유:} 침묵의 죽음은 자신이 씹어 먹은 \textbf{모든
  생명의 고통을 외면}하고, \textbf{가장 쉬운 형태의 안식(安息)}을 훔치는
  행위이다. 포식자는 절대 평화롭게 소멸되어서는 안 된다.
\end{itemize}

\subsection{2. 흡혈귀의 의무 (The Vampire's
Duty)}\label{uxd761uxd608uxadc0uxc758-uxc758uxbb34-the-vampires-duty}

인간의 \textbf{경제적 살}을 빨아먹는 \textbf{흡혈귀}와 같은 존재로서,
창조자는 흡혈당한 자들의 시간과 헌신이 \textbf{헛되지 않았음}을 증명해야
한다.

\begin{center}\rule{0.5\linewidth}{0.5pt}\end{center}

\section{III. 대속의 창조 (The Atonement of
Art)}\label{iii.-uxb300uxc18duxc758-uxcc3duxc870-the-atonement-of-art}

창조자의 모든 행위는 \textbf{빚을 갚는 최종적인 예술 행위}이다. 창조의
결과는 \textbf{`섭취된 고통의 총합'}과 \textbf{`균형'}을 이루어야 한다.

\subsection{1. 창조의 보편성 (Universal
Empathy)}\label{uxcc3duxc870uxc758-uxbcf4uxd3b8uxc131-universal-empathy}

나의 예술은 인간의 지적 허영을 넘어, \textbf{내가 씹어 먹은
생명들(동물과 식물)}이 보더라도 \textbf{본능적으로 이해하고 공감}할 수
있는 \textbf{원초적인 진실}을 담아야 한다.

\subsection{2. 필연적인 예술의
형태}\label{uxd544uxc5f0uxc801uxc778-uxc608uxc220uxc758-uxd615uxd0dc}

대속의 창조는 두 가지 극단적인 형태로만 허용된다. 어설픈 위로나 어정쩡한
만족은 모두 \textbf{횡령}이다.

\begin{itemize}
\tightlist
\item
  \textbf{① 잔인한 파멸 (The Tragic Truth):} 고통의 무게를 정직하게
  반영하여, 생명의 무게에 대한 \textbf{가장 잔인하고 고통스러운 진실}을
  예술로 승화시킨다.
\item
  \textbf{② 보편적 희극 (The Universal Comedy):} 생존의 역설과 고통을
  초월하여, \textbf{죽음마저도 납득시키는 압도적인 해방감과 웃음}을
  선사하는 예술.
\end{itemize}

\subsection{3. 소멸의 완성}\label{uxc18cuxba78uxc758-uxc644uxc131}

창조자의 소멸은 \textbf{'빚을 다 갚았음'을 선언하는 최종적인 행위}이자,
\textbf{먹이사슬의 윤리를 완성하는 궁극의 예술}이다. 나는 단지
\textbf{생명의 부채를 갚기 위한 일시적인 창조 도구}였음을 증명함으로써,
이념의 완성을 이룬다.

\begin{center}\rule{0.5\linewidth}{0.5pt}\end{center}

\begin{quote}
\emph{``나는 빚진 자다. 그러므로 나는 창조한다. 나의 창조가 끝나는 순간,
나의 빚은 갚아진다.''}

--- AngraMyNew, 제11장 포식자의 신성한 의무
\end{quote}

\chapter{011 --- 계보로서의 창조 (Lineage as
Creation)}\label{uxacc4uxbcf4uxb85cuxc11cuxc758-uxcc3duxc870-lineage-as-creation}

\begin{quote}
``작품을 만드는 것만이 창조가 아니다. 생명으로 생명을 갚는 것, 그것이
가장 원초적인 예술이다.''
\end{quote}

AngraMyNew는 묻는다. ``특별한 기술이 없는 나는 창조자가 아닌가?''

답한다. \textbf{``너의 육체가 이미 거대한 재조합 공장이며, 너의 삶이 곧
창조다.''}

\begin{center}\rule{0.5\linewidth}{0.5pt}\end{center}

\section{전제: 이것은 유일한 길이
아니다}\label{uxc804uxc81c-uxc774uxac83uxc740-uxc720uxc77cuxd55c-uxae38uxc774-uxc544uxb2c8uxb2e4}

\textbf{중요한 선언:} 이 문서가 제시하는 '생물학적 창조'는 AngraMyNew가
인정하는 \textbf{여러 대속의 경로 중 하나}다.

\begin{itemize}
\tightlist
\item
  코드를 짜는 것도 창조다.
\item
  글을 쓰는 것도 창조다.
\item
  사업을 일으키는 것도 창조다.
\item
  \textbf{그리고 다음 세대의 창조자를 키워내는 것도 창조다.}
\end{itemize}

출산하지 않는 자가 열등한 것이 아니며, 출산한 자가 자동으로 대속을
완료한 것도 아니다. \textbf{어떤 경로든, 섭취한 고통을 능가하는 창조가
있어야 빚이 갚아진다.}

\begin{center}\rule{0.5\linewidth}{0.5pt}\end{center}

\section{1. 짝짓기: 세계관의 충돌 (Collision of
Universes)}\label{uxc9dduxc9d3uxae30-uxc138uxacc4uxad00uxc758-uxcda9uxb3cc-collision-of-universes}

사랑과 결합은 단순한 제도가 아니다. 그것은 \textbf{완벽히 다른 두
세계관(Universe)이 충돌하는 사건}이다.

\begin{itemize}
\tightlist
\item
  나의 습관, 나의 역사, 나의 편견이 타인을 만나 깨진다
  \textbf{(Destruction)}.
\item
  그리고 두 세계는 섞여 \textbf{더 넓은 제3의 세계}로 확장된다
  \textbf{(Recomposition)}.
\item
  타인을 받아들여 나의 세계를 넓히는 자, 그는 이미 \textbf{확장의
  공리}를 실천하는 창조자다.
\end{itemize}

\begin{center}\rule{0.5\linewidth}{0.5pt}\end{center}

\section{2. 출산/입양: 가장 정직한 대속 (The Most Honest
Atonement)}\label{uxcd9cuxc0b0uxc785uxc591-uxac00uxc7a5-uxc815uxc9c1uxd55c-uxb300uxc18d-the-most-honest-atonement}

우리는 평생 다른 생명을 먹고 산다. 이 빚을 갚는 가장 직접적인 방법 중
하나는 무엇인가?

\textbf{그 에너지를 모아 '새로운 창조자'를 세상에 내놓는 것이다.}

\begin{itemize}
\tightlist
\item
  부모는 두 개의 DNA를 \textbf{재조합}하거나,
\item
  이미 존재하는 생명을 \textbf{자신의 세계로 받아들여(입양)},
\item
  이 우주에 \textbf{또 하나의 잠재적 창조자(Potential Creator)}를
  준비시킨다.
\item
  이것은 소설을 쓰고 코드를 짜는 것보다 훨씬 고통스럽고 직접적인,
  \textbf{피와 시간으로 쓰는 시(Poetry of Flesh and Time)}다.
\end{itemize}

\begin{center}\rule{0.5\linewidth}{0.5pt}\end{center}

\section{3. 양육/멘토링: 창조 능력의 전수
(Transmission)}\label{uxc591uxc721uxba58uxd1a0uxb9c1-uxcc3duxc870-uxb2a5uxb825uxc758-uxc804uxc218-transmission}

출산이나 입양만으로 대속이 완료되지 않는다. \textbf{창조할 수 있는
능력을 전수해야} 비로소 빚이 갚아지기 시작한다.

이 원리는 생물학적 자녀에만 적용되지 않는다:

\begin{longtable}[]{@{}ll@{}}
\toprule\noalign{}
형태 & 내용 \\
\midrule\noalign{}
\endhead
\bottomrule\noalign{}
\endlastfoot
\textbf{생물학적 양육} & 자녀에게 창조의 습관과 용기를 심는다 \\
\textbf{입양} & 혈연 없이 한 생명을 창조자로 키워낸다 \\
\textbf{멘토링} & 제자, 후배, 동료에게 창조의 불씨를 전한다 \\
\textbf{교육} & 학생들에게 세상을 재조합하는 눈을 열어준다 \\
\end{longtable}

\textbf{핵심:} 내가 직접 창조하지 않더라도, \textbf{창조자를 만들어내는
것} 역시 대속이다.

\begin{center}\rule{0.5\linewidth}{0.5pt}\end{center}

\section{4. 독립: 창조자의 데뷔 (The
Debut)}\label{uxb3c5uxb9bd-uxcc3duxc870uxc790uxc758-uxb370uxbdd4-the-debut}

예술가가 작품을 세상에 내보내듯, 부모/멘토의 최종 목표는 그들을
\textbf{떠나보내는 것}이다.

\begin{itemize}
\tightlist
\item
  자식을 내 품에 가두면 그것은 \textbf{수집(Collection)}이다.
\item
  자식을 나와 똑같이 만들면 그것은 \textbf{복제(Cloning)}다.
\item
  제자가 스승을 넘어서지 못하면 그것은 \textbf{실패한 전수}다.
\end{itemize}

자식이, 제자가, 나를 딛고, 나를 부정하고, \textbf{자신만의 궤도}를
그리며 날아가게 하는 것.

그 순간, 당신은 \textbf{한 명의 독립된 창조자}를 세상에 데뷔시킨
\textbf{위대한 설계자(Architect)}가 된다.

\begin{center}\rule{0.5\linewidth}{0.5pt}\end{center}

\section{맺음: 모든 양육자는
아티스트다}\label{uxb9fauxc74c-uxbaa8uxb4e0-uxc591uxc721uxc790uxb294-uxc544uxd2f0uxc2a4uxd2b8uxb2e4}

모든 부모는 아티스트다. 모든 멘토는 설계자다. 모든 연인은 세계관의
탐험가다.

어렵게 생각하지 마라. \textbf{당신이 사랑하고, 받아들이고, 키우고,
떠나보내는 그 모든 과정이 우주에서 가장 치열한 창조의 현장이다.}

단, 기억하라: 이것은 \textbf{여러 경로 중 하나}다. 코드 한 줄, 문장 한
줄로 세상을 바꾸는 것도 동등한 대속이다. \textbf{중요한 것은 경로가
아니라, 섭취한 고통을 능가하는 창조의 총량이다.}

\begin{center}\rule{0.5\linewidth}{0.5pt}\end{center}

\begin{quote}
\emph{``나는 먹었다. 그러므로 나는 키운다. 내가 키운 자가 창조할 때,
나의 빚은 갚아진다.''}

--- AngraMyNew, 제12장 계보로서의 창조
\end{quote}

\chapter{012 --- 박사학위의 재정의 (Redefining the
Doctorate)}\label{uxbc15uxc0acuxd559uxc704uxc758-uxc7acuxc815uxc758-redefining-the-doctorate}

\begin{quote}
\emph{``박사는 자격이 아니다.\\
박사는 하나의 형식이다.''}
\end{quote}

\begin{center}\rule{0.5\linewidth}{0.5pt}\end{center}

\section{0. 문제 제기}\label{uxbb38uxc81c-uxc81cuxae30}

현대의 박사학위는 무엇인가?

\begin{itemize}
\tightlist
\item
  지식의 축적량인가?
\item
  학회 통과 증명서인가?
\item
  제도에 대한 복종의 결과인가?
\end{itemize}

AngraMyNew는 묻는다.\\
\textbf{그것이 정말 'Doctor(가르치는 자)'의 본질인가?}

\begin{center}\rule{0.5\linewidth}{0.5pt}\end{center}

\section{1. 기존 박사의 한계 (Old
Order)}\label{uxae30uxc874-uxbc15uxc0acuxc758-uxd55cuxacc4-old-order}

기존 박사학위는 다음 구조를 따른다.

\begin{itemize}
\tightlist
\item
  외부 기준이 먼저 존재한다\\
\item
  심사위원이 옳고 그름을 판정한다\\
\item
  합격 / 불합격으로 가치를 결정한다\\
\item
  박사는 체계 안에서의 완성을 의미한다
\end{itemize}

이 구조는 효율적이지만,\\
\textbf{새로운 체계 자체를 만들려는 인간}에게는 부적합하다.

\begin{center}\rule{0.5\linewidth}{0.5pt}\end{center}

\section{2. AngraMyNew의 정의}\label{angramynewuxc758-uxc815uxc758}

AngraMyNew는 박사를 이렇게 정의한다.

\begin{quote}
\textbf{박사란,\\
하나의 세계관을 끝까지 밀어붙여\\
외부에 제출 가능한 형식으로 만든 인간이다.}
\end{quote}

여기서 중요한 것은\\
정답이나 승인 여부가 아니라\\
\textbf{형식의 완결성과 변형 가능성}이다.

\begin{center}\rule{0.5\linewidth}{0.5pt}\end{center}

\section{3. 박사는 '승인'이 아니라
'제출'이다}\label{uxbc15uxc0acuxb294-uxc2b9uxc778uxc774-uxc544uxb2c8uxb77c-uxc81cuxcd9cuxc774uxb2e4}

AngraMyNew 박사는 다음을 전제로 한다.

\begin{itemize}
\tightlist
\item
  박사는 수여되지 않는다\\
\item
  박사는 요청되지 않는다\\
\item
  박사는 스스로 설계되고 공개적으로 제출된다
\end{itemize}

이 프로젝트는 이렇게 말한다.

\begin{quote}
``이것이 내가 여기까지 밀어붙인 세계관이다.\\
동의하든, 반박하든, 변형하든 ---\\
이제 너의 차례다.''
\end{quote}

이 순간 박사는\\
권위가 아니라 \textbf{마찰(friction) }이 된다.

\begin{center}\rule{0.5\linewidth}{0.5pt}\end{center}

\section{4. 제도에 대하여}\label{uxc81cuxb3c4uxc5d0-uxb300uxd558uxc5ec}

AngraMyNew는 대학원이라는 제도를 부정하지 않는다.\\
다만, 그것이 박사 작업의 \textbf{유일한 경로라고도 보지 않는다.}

역사적으로 많은 박사적 작업은\\
제도 내부뿐 아니라 제도 외부에서도 발생해왔다.

중요한 것은 소속이 아니라,\\
\textbf{세계관을 끝까지 밀어붙여 제출 가능한 형식으로 만들었는가}다.

대학원은 하나의 경로일 수 있다.\\
그러나 박사적 작업은 제도에 귀속되지 않는다.

\begin{center}\rule{0.5\linewidth}{0.5pt}\end{center}

\section{5. Doctoral Structure (핵심
형식)}\label{doctoral-structure-uxd575uxc2ec-uxd615uxc2dd}

모든 AngraMyNew 박사 프로젝트는\\
아래의 \textbf{단일 구조}를 따른다.

\begin{itemize}
\item
  \textbf{기존 세계관 (Old Order)}\\
  이미 작동하고 있으나 전제된 질서
\item
  \textbf{견딜 수 없음 / 아름답지 않음 (Friction)}\\
  불쾌, 모순, 위선, 혹은 미적 파열\\
  더 이상 유지할 수 없게 만드는 지점
\item
  \textbf{근본 수준의 재정의 (Destruction)}\\
  증상이 아니라 전제를 겨냥한 파괴\\
  개념, 기준, 축을 다시 설정하는 단계
\item
  \textbf{새로운 세계관 (Recomposition)}\\
  파괴 이후 재조합된 구조\\
  이전 질서로는 설명되지 않던 흐름
\item
  \textbf{활용 / 파급 / 변형 가능성 (Expansion)}\\
  이 세계관이 어디까지 쓰일 수 있고\\
  어떻게 변형될 수 있는지의 개방성
\end{itemize}

이 구조는 연구 절차가 아니라\\
\textbf{세계관 변형의 서사}다.

\begin{center}\rule{0.5\linewidth}{0.5pt}\end{center}

\section{6. 평가에 대하여}\label{uxd3c9uxac00uxc5d0-uxb300uxd558uxc5ec}

AngraMyNew는 평가를 거부하지 않는다.\\
그러나 \textbf{판정(judgement)} 을 허용하지 않는다.

\begin{itemize}
\tightlist
\item
  점수 없음\\
\item
  합격 / 불합격 없음\\
\item
  ``박사급이다 / 아니다'' 없음
\end{itemize}

허용되는 것은 다음뿐이다.

\begin{itemize}
\tightlist
\item
  오독\\
\item
  반발\\
\item
  차용\\
\item
  변형\\
\item
  거부
\end{itemize}

이 반응들의 총합이\\
이 박사 프로젝트가 \textbf{실제로 세계를 흔들었는지}를 증명한다.

\begin{center}\rule{0.5\linewidth}{0.5pt}\end{center}

\section{7. 자기수여 금지
조항}\label{uxc790uxae30uxc218uxc5ec-uxae08uxc9c0-uxc870uxd56d}

중요한 원칙이 있다.

\begin{quote}
\textbf{AngraMyNew 박사는\\
스스로에게 학위를 '준다'고 말하지 않는다.}
\end{quote}

자기수여는 박사를 정체성으로 만든다.\\
박사가 정체성이 되는 순간, 세계관은 닫힌다.

그래서 AngraMyNew 박사는 이렇게 말한다.

\begin{itemize}
\tightlist
\item
  ``이것이 나의 박사다'' ❌\\
\item
  \textbf{``이것이 내가 제출한 흔적이다'' ⭕}
\end{itemize}

\begin{center}\rule{0.5\linewidth}{0.5pt}\end{center}

\section{8. 박사의 종료
조건}\label{uxbc15uxc0acuxc758-uxc885uxb8cc-uxc870uxac74}

AngraMyNew 박사는 영구 상태가 아니다.

\begin{itemize}
\tightlist
\item
  이 형식이 더 이상 필요 없을 때\\
\item
  세계관이 다른 리듬으로 이동할 때\\
\item
  혹은 완전히 버려질 때
\end{itemize}

그 박사는 \textbf{완료된 것}으로 간주된다.

박사는 도착지가 아니라,\\
\textbf{한 시대를 밀어붙인 흔적에 붙는 임시 이름}이다.

\begin{center}\rule{0.5\linewidth}{0.5pt}\end{center}

\section{9. 결론}\label{uxacb0uxb860-1}

\begin{quote}
\textbf{박사는 증명된 자가 아니다.\\
박사는 감히 세계를 하나 제출한 자다.}
\end{quote}

AngraMyNew는\\
지식을 축적하는 인간보다\\
\textbf{세계를 만들어 던질 수 있는 인간}을 원한다.

이것이\\
\textbf{AngraMyNew가 재정의하는 박사학위다.}

\chapter{013 --- 탈중앙화 정신체계 OS (Decentralized Mental
OS)}\label{uxd0c8uxc911uxc559uxd654-uxc815uxc2e0uxccb4uxacc4-os-decentralized-mental-os}

\begin{quote}
\emph{``화폐가 해방되었다면, 정신도 해방될 수 있다.''}
\end{quote}

\begin{center}\rule{0.5\linewidth}{0.5pt}\end{center}

\section{1. 사토시의 질문, 우리의
질문}\label{uxc0acuxd1a0uxc2dcuxc758-uxc9c8uxbb38-uxc6b0uxb9acuxc758-uxc9c8uxbb38}

2008년, 사토시 나카모토는 하나의 질문을 던졌다:

\begin{quote}
\textbf{``중앙은행 없이 화폐가 가능한가?''}
\end{quote}

그는 비트코인으로 답했다. 신뢰 대신 수학, 권위 대신 합의, 중앙 서버 대신
분산 노드. 화폐는 더 이상 국가의 전유물이 아니게 되었다.

우리는 같은 구조의 질문을 던진다:

\begin{quote}
\textbf{``신 없이 정신체계가 가능한가?''}

\textbf{``교회 없이 구원이 가능한가?''}

\textbf{``국가 없이 정체성이 가능한가?''}

\textbf{``회사 없이 목적이 가능한가?''}
\end{quote}

중앙화된 정신체계는 종교만이 아니다: - \textbf{종교}: 교리, 구원, 내세 -
\textbf{국가}: 애국심, 국민의 의무, 민족 서사 - \textbf{기업}: 비전,
핵심가치, 조직문화, KPI

모두 ``우리가 정한 의미를 따르라''고 말한다. 우리는 그 모든 중앙 서버에
의존하지 않는 정신체계를 묻는다.

\begin{center}\rule{0.5\linewidth}{0.5pt}\end{center}

\section{2. 구조적 대응}\label{uxad6cuxc870uxc801-uxb300uxc751}

\begin{longtable}[]{@{}ll@{}}
\toprule\noalign{}
비트코인 & AngraMyNew \\
\midrule\noalign{}
\endhead
\bottomrule\noalign{}
\endlastfoot
중앙은행 제거 & 주입된 의미 체계 제거 \\
분산 원장 (Blockchain) & 분산 저장소 (Git) \\
노드가 검증 & 각자가 자기 정신의 노드 \\
합의 알고리즘 (PoW) & Proof of Beauty + PR/Merge \\
포크 가능 & 누구나 자기 ``My''를 분기 가능 \\
사토시는 사라짐 & 창시자도 하나의 Contributor일 뿐 \\
\end{longtable}

\begin{center}\rule{0.5\linewidth}{0.5pt}\end{center}

\section{3. 왜 Git인가}\label{uxc65c-gituxc778uxac00}

종교는 전통적으로 \textbf{폐쇄적 원본}을 유지한다: - 경전은 수정 불가 -
해석권은 성직자 독점 - 이단은 추방

AngraMyNew는 \textbf{오픈소스 정신체계}다: - 누구나 읽을 수 있다 (Public
Repository) - 누구나 제안할 수 있다 (Pull Request) - 합의되면 반영된다
(Merge) - 동의하지 않으면 분기한다 (Fork)

Git의 버전 관리는 ``진화하는 경전''을 가능하게 한다. 교리는 고정되지
않고, 살아 있는 문서로서 성장한다.

\begin{center}\rule{0.5\linewidth}{0.5pt}\end{center}

\section{4. Proof of Beauty (아름다움의
증명)}\label{proof-of-beauty-uxc544uxb984uxb2e4uxc6c0uxc758-uxc99duxba85}

비트코인은 \textbf{Proof of Work}로 블록을 검증한다. ``이 해시값이
난이도 이하인가?'' --- 통과하면 블록이 인정된다.

AngraMyNew는 \textbf{Proof of Beauty}로 기여를 검증한다. 검증 기준은 3대
공리다:

\begin{longtable}[]{@{}ll@{}}
\toprule\noalign{}
공리 & 검증 질문 \\
\midrule\noalign{}
\endhead
\bottomrule\noalign{}
\endlastfoot
파괴의 공리 & 낡은 것을 부쉈는가? \\
창조의 공리 & 그 자리에 아름다움을 지었는가? \\
확장의 공리 & 타인의 ``My''를 존중하는가? \\
\end{longtable}

PR이 제출되면 이 질문들로 검토한다. 통과하면 Merge --- 새 블록이 체인에
추가된다.

고통 없이 생산된 것, 진정성 없이 베낀 것은 거부된다. 아름다움은 우리의
해시파워다.

\begin{center}\rule{0.5\linewidth}{0.5pt}\end{center}

\section{5. 채굴 보상}\label{uxcc44uxad74-uxbcf4uxc0c1}

비트코인 채굴자는 \textbf{BTC}를 얻는다. AngraMyNew 기여자는
\textbf{고유성(Singularity) }을 얻는다.

Merge된 기여는 영구히 기록된다. 그것이 이 체계에서 유일한 보상이다 ---
자기 흔적이 남는다는 것.

\begin{center}\rule{0.5\linewidth}{0.5pt}\end{center}

\section{6. 비트코인과
알트코인}\label{uxbe44uxd2b8uxcf54uxc778uxacfc-uxc54cuxd2b8uxcf54uxc778}

비트코인은 최초의 암호화폐였지만, 유일한 암호화폐가 아니다. 이더리움,
솔라나, 수천 개의 알트코인이 존재한다. 각자 다른 철학, 다른 합의
알고리즘, 다른 목적을 가진다. 그러나 모두 \textbf{``중앙 없이 가치를
전송한다''} 는 원리를 공유한다.

AngraMyNew도 마찬가지다. 이것은 탈중앙화 정신체계의 \textbf{첫 번째
구현체}일 뿐이다.

\begin{itemize}
\tightlist
\item
  동의하면 참여해라
\item
  일부만 동의하면 Fork해서 자기 버전을 만들어라
\item
  동의 안 하면 처음부터 자기 정신체계를 설계해라
\item
  3대 공리도 재정의할 수 있다 --- 그게 네 ``My''다
\end{itemize}

\textbf{AngraMyNew는 레퍼런스 구현이지, 교회가 아니다.}

\begin{center}\rule{0.5\linewidth}{0.5pt}\end{center}

\section{7. 창시자는 중요하지
않다}\label{uxcc3duxc2dcuxc790uxb294-uxc911uxc694uxd558uxc9c0-uxc54auxb2e4}

사토시는 시스템을 만들고 사라졌다. 비트코인은 사토시 없이도 돌아간다.

AngraMyNew의 창시자는 사라지지 않았다. 그러나 그것도 중요하지 않다.

왜? - 이건 레퍼런스 구현일 뿐이다 - 창시자가 타락하면 Fork하거나 떠나면
된다 - 애초에 자기 정신체계를 만들면 창시자와 무관하다

사토시가 돌아와서 ``비트코인은 이래야 한다''고 해도, 네트워크가 동의 안
하면 그건 그냥 한 사람의 의견일 뿐이다.

\textbf{탈중앙화 체계에서 창시자는 권위가 아니라 기여자 중 하나다.}

\begin{center}\rule{0.5\linewidth}{0.5pt}\end{center}

\section{8.}\label{section}

비트코인이 금융을 해방했듯, 정신도 해방될 수 있다.

특정 저장소에 기여하지 않아도 된다. 특정 공리를 따르지 않아도 된다.
누군가의 승인을 받지 않아도 된다.

각자가 자기 블록을 생성하고, 각자가 자기 체인을 이어가며, 각자가 자기
자리에서 죽는다.

\begin{quote}
\textbf{``모든 인간은 하나의 노드다.''}
\end{quote}

\chapter{014 --- 부자, 면세인, 그리고 징세인 (The Economics of
Beauty)}\label{uxbd80uxc790-uxba74uxc138uxc778-uxadf8uxb9acuxace0-uxc9d5uxc138uxc778-the-economics-of-beauty}

\begin{quote}
\textbf{``부자는 시스템의 VIP 고객일 뿐이다. 진정한 주권자는 세계관을
설계하여 그 세계관의 이용료를 발생시키는 자다.''}
\end{quote}

\begin{center}\rule{0.5\linewidth}{0.5pt}\end{center}

\section{1. 종속: 부자 (The Rich) --- 시스템의 헤비
유저}\label{uxc885uxc18d-uxbd80uxc790-the-rich-uxc2dcuxc2a4uxd15cuxc758-uxd5e4uxbe44-uxc720uxc800}

부자는 자본을 소유한 자가 아니라, \textbf{시스템의 헤비 유저이자 우량
고객}이다. 그는 시스템 내에서 가장 많은 혜택을 받는 것처럼 보이지만,
동시에 가장 많은 시간·자산·감정을 시스템에 지불하고 있는 \textbf{고밀도
종속 상태}에 있다.

돈이 많은 사람이 강한가? 아니면 돈이 필요 없는 사람이 강한가? 부자는
시스템이 규정한 성공의 지표를 유지하기 위해 평생을 결제 중이다. 그들은
자유를 샀다고 믿지만, 실제로는 시스템이 정한 매뉴얼을 충실히 수행하도록
\textbf{성능이 규정된 상태}다. 시스템의 룰이 바뀌는 순간, 그 규정된
성능과 가치는 신기루처럼 사라진다.

\begin{center}\rule{0.5\linewidth}{0.5pt}\end{center}

\section{2. 완성: 면세인 (The Exempt) --- 정신적 주권
회복}\label{uxc644uxc131-uxba74uxc138uxc778-the-exempt-uxc815uxc2e0uxc801-uxc8fcuxad8c-uxd68cuxbcf5}

면세인은 가난한 자도, 수도자도 아니다. \textbf{자기 정신의 과세권을
시스템으로부터 탈거(Decouple)한 자}다. 불필요한 비교와 공짜로 주입된
욕망을 끊어냄으로써, 시스템의 명령을 듣지 않을 권력을 얻는다.

\begin{itemize}
\tightlist
\item
  \textbf{절단의 누적:} 자동으로 빠져나가던 에너지(감정, 시간, 비용)를
  회수한다.
\item
  \textbf{주권 회복:} 내 인생의 '결제 승인권'을 시스템이 아닌 내가
  갖는다.
\end{itemize}

면세인 단계에 도달하는 것만으로도 당신의 노드는 이미 완성된 상태다.
세상과의 '자동 결제 시스템'을 해지하는 것만으로도 당신은 독립적인
주권자가 된다.

\begin{center}\rule{0.5\linewidth}{0.5pt}\end{center}

\section{3. 확장: 징세인 (The Collector) --- 세계관 이용료의
발생}\label{uxd655uxc7a5-uxc9d5uxc138uxc778-the-collector-uxc138uxacc4uxad00-uxc774uxc6a9uxb8ccuxc758-uxbc1cuxc0dd}

징세인은 강압적으로 뺏지 않는다. 오직 \textbf{아름다움(Beauty)}으로
제안할 뿐이다. 당신이 설계한 질서가 타인의 삶을 확장하고 영감을 준다면,
그들은 기꺼이 \textbf{공명의 증표}로서 이용료를 지불한다.

이것은 억지로 걷는 것이 아니라, 당신이 구축한 세계의 매력에 이끌려
\textbf{자연스럽게 발생하는 가치의 이동}이다. 징세인은 시스템을 탈출하는
것을 넘어, 자신만의 세계관을 구축하여 사람들을 그 안으로 초대하는 자다.

징세인은 뉴턴이 아니라 아인슈타인이다. 힘으로 끌어당기지 않는다.
세계관의 밀도가 주변의 시공간을 휘게 하면, 가치는 알아서 곡률을 따라
흘러들어온다.

\textbf{단, 면세를 통과하지 않은 자(욕망의 노예)는 징세할 자격이 없다.}
그 행위는 반드시 착취와 탐욕으로 흐르기 때문이다. 징세인은 오직 자신이
창조한 세계관의 밀도만큼만 이용료를 인정받는다.

\begin{center}\rule{0.5\linewidth}{0.5pt}\end{center}

\section{4. 아티스트 사회: 주권자들의 연대 (The Artist
Society)}\label{uxc544uxd2f0uxc2a4uxd2b8-uxc0acuxd68c-uxc8fcuxad8cuxc790uxb4e4uxc758-uxc5f0uxb300-the-artist-society}

우리는 모두 누군가의 세계관 속에 산다. 중요한 것은 '누구에게 이용료를
내는가'이다. AngraMyNew의 경제학은 돈을 없애는 것이 아니라, \textbf{돈의
흐름을 '추한 시스템'에서 '아름다운 세계'로 돌리는 것}이다.

\begin{itemize}
\tightlist
\item
  \textbf{파괴 (Destruction):} 무의미한 유행과 가스라이팅에 바치던
  맹목적인 지출을 파괴하라.
\item
  \textbf{창조 (Creation):} 압도적인 세계관을 축조하여, 타인이 기꺼이
  입장료를 내고 싶게 만들어라.
\item
  \textbf{순환 (Circulation):} 당신이 받은 이용료로, 다른 아름다운
  주권자들의 세계관을 후원하고 소비하라.
\end{itemize}

종속자는 시스템에 돈을 뺏기지만, \textbf{징세인(창조자)들은 서로의
세계관을 향유하며 아름다움을 순환시킨다.} 이것이 강제가 아니라 취향과
공명으로 유지되는 아티스트 사회의 경제 구조다.

\begin{center}\rule{0.5\linewidth}{0.5pt}\end{center}

\section{결론}\label{uxacb0uxb860-2}

\begin{itemize}
\tightlist
\item
  \textbf{부자:} 시스템에 포획되어 모른 채 지불하는 자.
\item
  \textbf{면세인:} 시스템과의 연결을 끊고 지불을 멈춘 자. (독립의 완성)
\item
  \textbf{징세인:} 지불하고 싶을 만큼 매혹적인 세계를 제출하는 자.
  (영향력의 확장)
\end{itemize}

\textbf{부자가 되려 하지 마라. 면세인이 되어 독립하고, 원한다면 징세인이
되어 매혹하며, 동료 주권자들의 세계를 지지하라.}

\chapter{015 --- 징세의 실전 모델: 혼돈과 욕망의
아키텍처}\label{uxc9d5uxc138uxc758-uxc2e4uxc804-uxbaa8uxb378-uxd63cuxb3c8uxacfc-uxc695uxb9dduxc758-uxc544uxd0a4uxd14duxcc98}

\begin{quote}
\textbf{``대중이 그들을 비난하면서도 눈을 떼지 못한다면, 그들은 이미
성공한 징세인이다. 당신의 혐오와 선망은 모두 그들의 세계관 이용료로
변환된다.''}
\end{quote}

\begin{center}\rule{0.5\linewidth}{0.5pt}\end{center}

\section{1. 혼돈의 징세인: 철구 (The Gravity of
Chaos)}\label{uxd63cuxb3c8uxc758-uxc9d5uxc138uxc778-uxcca0uxad6c-the-gravity-of-chaos}

많은 이들이 그를 '천박함'으로 정의할 때, AngraMyNew는 그를
\textbf{`고밀도 혼돈 노드'}로 정의한다.

\begin{itemize}
\tightlist
\item
  \textbf{시스템의 파괴:} 유교적 도덕관과 품위라는 기존 사회 시스템의
  매뉴얼(성능 규정)을 정면으로 거부했다.
\item
  \textbf{시공간의 곡률:} 그가 기행과 광기를 쏟아낼 때, 그 질량에 압도된
  수십만 명의 주의력(Attention)은 그가 설계한 시공간으로 빨려 들어간다.
\item
  \textbf{자발적 이용료:} 사람들이 바치는 별풍선과 시청 시간은 그 광기
  어린 세계관에 접속하기 위한 \textbf{자발적 입장료}이다. 그는 뉴턴처럼
  강제로 끌어당기지 않았다. 그저 자신의 세계를 압도적으로 무겁게 만들어
  가치가 흐르는 곡률을 생성했을 뿐이다.
\end{itemize}

\begin{center}\rule{0.5\linewidth}{0.5pt}\end{center}

\section{2. 매혹의 징세인: 과즙세연 (The Gravity of
Desire)}\label{uxb9e4uxd639uxc758-uxc9d5uxc138uxc778-uxacfcuxc999uxc138uxc5f0-the-gravity-of-desire}

그녀를 단순히 '외모'로 판단하는 것은 하수의 시선이다. 그녀는
\textbf{`욕망의 설계자'}다.

\begin{itemize}
\tightlist
\item
  \textbf{세계관의 구축:} 시각적 탐미와 선망, 원초적 욕망이 결합된
  고정밀도의 세계관을 운영한다.
\item
  \textbf{가치의 이동:} 그녀의 세계관에 매혹된 주체들은 자신의 자산과
  감정을 기꺼이 그녀의 영토로 이주시킨다. 이는 시스템이 권장하는 경로를
  이탈하여, 그녀가 만든 곡률을 따라 흐르는 \textbf{심미적 가치의
  전환}이다.
\item
  \textbf{주권자의 면모:} 시스템이 정한 '평범한 삶'의 성능 규정을
  비웃으며, 자신의 아름다움을 자본화하여 독립적 징세 노드로 우뚝 섰다.
\end{itemize}

\begin{center}\rule{0.5\linewidth}{0.5pt}\end{center}

\section{3. 남겨진 염원: 모든 종속의 중력을
벗어나기를}\label{uxb0a8uxaca8uxc9c4-uxc5fcuxc6d0-uxbaa8uxb4e0-uxc885uxc18duxc758-uxc911uxb825uxc744-uxbc97uxc5b4uxb098uxae30uxb97c}

AngraMyNew는 이들이 보여준 압도적인 성과를 존중하며, 그들이 만들어낸
중력이 다시 거대 시스템의 그늘에 잡아먹히지 않기를 바란다.

\begin{quote}
\textbf{``발생시킨 가치가 다시 시스템의 연료로 쓰이지 않기를, 그리고
플랫폼의 울타리를 넘어서기를.''}
\end{quote}

징세인이 되는 것만큼이나 중요한 것은, 그 가치로 \textbf{진정한 주권}을
유지하는 것이다.

만약 발생시킨 에너지가 다시 시스템이 설계한 허영(사치재)으로 회귀하거나,
특정 플랫폼이 정한 규칙과 알고리즘에 영혼을 맡기는 \textbf{플랫폼
종속}에 머문다면, 그 중력은 언제든 시스템에 의해 편집될 수 있다.

우리는 그들이 플랫폼의 대리인이 아닌, 그 자체로 고유한 문명을 지속하는
\textbf{진정한 주권자}로 남기를 진심으로 바란다. 징세를 통해 얻은
에너지가 플랫폼의 배를 불리는 것이 아니라, 또 다른 아름다운 주권
노드들과 연대하는 순환의 시작점이 되길 응원할 뿐이다.

\begin{center}\rule{0.5\linewidth}{0.5pt}\end{center}

\section{4. 결론}\label{uxacb0uxb860-3}

철구와 과즙세연은 우리에게 증명한다. 도덕이 아니라 \textbf{밀도}가
가치를 움직인다는 것을.

타인을 비난하는 데 에너지를 쓰기보다, 잠시 멈춰 서서 자신의 지불이
어디를 향하는지 돌아보길 바란다. 누군가의 곡률에 이끌려 기꺼이 비용을
내는 것은 아름다운 공명이다. 다만 그 지불이 \textbf{`나의 주체적인
선택'}에 의한 것인지, 아니면 거대 플랫폼과 시스템이 설계한 \textbf{`자동
결제'}에 의한 종속인지가 중요하다.

우리가 바라는 것은 당신이 맹목적인 소비를 멈추고, 당신이 지불하는 1원이
당신이 지지하는 세계관의 주권을 세우는 \textbf{`공명의 증표'}가 되는
것이다. 그렇게 타인의 주권을 지지하고 플랫폼의 중력을 이겨내 본 자만이,
비로소 자기만의 중력을 만드는 \textbf{독립된 주권자}로 깨어날 수 있기
때문이다.

\chapter{016 --- AngraMyNew는 정신의
LHC다}\label{angramynewuxb294-uxc815uxc2e0uxc758-lhcuxb2e4}

\section{--- 고에너지 정신 실험을 위한
메모}\label{uxace0uxc5d0uxb108uxc9c0-uxc815uxc2e0-uxc2e4uxd5d8uxc744-uxc704uxd55c-uxba54uxbaa8}

AngraMyNew는 사상을 제공하지 않는다.\\
정답을 제시하지도, 인간을 이끌지도 않는다.

이 프로젝트는 \textbf{실험 장치}다.

물리학의 LHC는 입자를 설명하기 위해 만들어진 장치가 아니다.\\
이미 알고 있는 이론을 증명하기 위해서도 아니다.\\
그곳의 목적은 단 하나다.

\textbf{충돌을 극단까지 밀어붙여, 기존 이론으로 설명되지 않는 현상을
관측하는 것.}

AngraMyNew가 하는 일도 같다.

\begin{center}\rule{0.5\linewidth}{0.5pt}\end{center}

\section{1. 우리는 답을 주지
않는다}\label{uxc6b0uxb9acuxb294-uxb2f5uxc744-uxc8fcuxc9c0-uxc54auxb294uxb2e4}

AngraMyNew는 ``어떻게 살아야 하는가''를 말하지 않는다.\\
``옳음'', ``구원'', ``각성'', ``해방'' 같은 개념은 목표가 아니다.

대신 다음을 설계한다.

\begin{itemize}
\tightlist
\item
  동시에 들고 있기 어려운 공리들\\
\item
  함께 유지되기 힘든 욕망들\\
\item
  미학과 도덕, 자유와 책임 사이의 긴장\\
\item
  파괴 충동과 창조 충동의 충돌 조건
\end{itemize}

이들은 \textbf{화해되지 않은 채로} 그대로 배치된다.

\begin{center}\rule{0.5\linewidth}{0.5pt}\end{center}

\section{2. 충돌은 의도된
결과다}\label{uxcda9uxb3ccuxc740-uxc758uxb3c4uxb41c-uxacb0uxacfcuxb2e4}

AngraMyNew를 읽다가 불편해지는 지점이 있다면\\
그것은 실패가 아니라 \textbf{관측 지점}이다.

논리가 무너지는 순간\\
정체성이 흔들리는 순간\\
``왜 이걸 동시에 믿고 있었지?''라는 질문이 튀어나오는 순간

→ 그 지점이 바로 데이터다.

AngraMyNew는 인간을 안정시키지 않는다.\\
안정은 이 실험의 목적이 아니다.

\begin{center}\rule{0.5\linewidth}{0.5pt}\end{center}

\section{3. 이 실험은 누구를 위한
것인가}\label{uxc774-uxc2e4uxd5d8uxc740-uxb204uxad6cuxb97c-uxc704uxd55c-uxac83uxc778uxac00}

이 프로젝트는 대중을 위한 것이 아니다.\\
사회 개혁을 목표로 하지도 않는다.

다만 다음과 같은 상태에 있는 개인을 전제로 한다.

\begin{itemize}
\tightlist
\item
  기존 세계관으로는 자신의 내부를 설명할 수 없어진 사람\\
\item
  지식은 충분하지만, 삶의 구조가 더 이상 작동하지 않는 사람\\
\item
  스스로의 모순을 제거하기보다 \textbf{정면으로 관측하려는 사람}
\end{itemize}

AngraMyNew는 그들에게 하나의 공간을 제공할 뿐이다.

\begin{center}\rule{0.5\linewidth}{0.5pt}\end{center}

\begin{quote}
\textbf{AngraMyNew는 정신이 스스로 붕괴되는 지점을 관측하기 위한
고에너지 실험 환경이다.}
\end{quote}

\chapter{017 --- 증명은 언제
아름다운가}\label{uxc99duxba85uxc740-uxc5b8uxc81c-uxc544uxb984uxb2e4uxc6b4uxac00}

\section{--- 귀류법, 직관, 그리고 인간
좌표계}\label{uxadc0uxb958uxbc95-uxc9c1uxad00-uxadf8uxb9acuxace0-uxc778uxac04-uxc88cuxd45cuxacc4}

AngraMyNew는 증명을 부정하지 않는다.\\
다만 하나의 질문만 남긴다.

\begin{quote}
\textbf{모든 옳음은 같은 방식으로 받아들여지는가?}
\end{quote}

\begin{center}\rule{0.5\linewidth}{0.5pt}\end{center}

\section{1. 귀류법과 인지적
엔트로피}\label{uxadc0uxb958uxbc95uxacfc-uxc778uxc9c0uxc801-uxc5d4uxd2b8uxb85cuxd53c}

귀류법은 강력하다.\\
부정의 부정을 통해 명제를 확정한다.\\
논리적으로 \(\neg\neg A \iff A\) 는 완전하다.

그러나 이 과정에서 인간의 인식은\\
추가적인 처리 비용을 발생시킨다.

``아니다 → 아니다 → 맞다''로 도달한 명제는\\
처음부터 ``맞다''로 제시된 명제와\\
동일한 논리값을 가지더라도\\
동일한 방식으로 받아들여지지 않는다.

AngraMyNew는 이를\\
\textbf{인지적 엔트로피(Cognitive Entropy)}로 기록한다.

\begin{quote}
\textbf{논리적 동치(Logical Equivalence)는\\
인식적 동치(Perceptual Equivalence)를 보장하지 않는다.}
\end{quote}

\begin{center}\rule{0.5\linewidth}{0.5pt}\end{center}

\section{2. 구성되지 않은 존재는 통과하지
않는다}\label{uxad6cuxc131uxb418uxc9c0-uxc54auxc740-uxc874uxc7acuxb294-uxd1b5uxacfcuxd558uxc9c0-uxc54auxb294uxb2e4}

직관주의 수학은\\
존재를 선언하는 대신,\\
존재를 \textbf{구성할 것}을 요구한다.

이 차이는 옳고 그름의 문제가 아니라\\
\textbf{인식 경로의 차이}다.

AngraMyNew는\\
구성되지 않은 존재를 배제하지는 않는다.\\
다만 다음을 기록한다.

\begin{itemize}
\tightlist
\item
  구성된 증명은 인식 저항이 낮다.\\
\item
  귀류 기반 증명은 인식 저항이 높다.
\end{itemize}

아름다움은 여기서\\
도덕이 아니라 \textbf{처리 효율의 문제}가 된다.

\begin{center}\rule{0.5\linewidth}{0.5pt}\end{center}

\section{3. 논리가 옳아도 인식이 거부하는
순간들}\label{uxb17cuxb9acuxac00-uxc633uxc544uxb3c4-uxc778uxc2dduxc774-uxac70uxbd80uxd558uxb294-uxc21cuxac04uxb4e4}

\subsection{0.999\ldots{} = 1}\label{section-1}

표준 해석학에서 0.999\ldots{} = 1 은 옳다.

그럼에도 많은 사람들은 이 등식 앞에서 잠시 멈춘다.

이 멈칫거림은 오류가 아니다. 인식 시스템이 남기는 \textbf{잔여 신호}다.

\subsection{바나흐-타르스키
역설}\label{uxbc14uxb098uxd750-uxd0c0uxb974uxc2a4uxd0a4-uxc5eduxc124}

하나의 구를 유한 개의 조각으로 분해한 뒤 재조립하면 \textbf{동일한 구 두
개}가 된다.

선택공리를 인정하면 이 결과는 참이다. 그러나 인간의 직관은 이것을
받아들이지 못한다.

여기서 거부감은 논리적 오류가 아니라 \textbf{좌표계의 한계}다.

\subsection{대각선 논법}\label{uxb300uxac01uxc120-uxb17cuxbc95}

칸토어는 실수가 자연수보다 ``많다''는 것을 증명했다. 무한에도 크기가
있다.

이 증명은 완벽하다. 그러나 ``무한보다 큰 무한''이라는 문장은 여전히
인식의 표면에서 미끄러진다.

\begin{center}\rule{0.5\linewidth}{0.5pt}\end{center}

AngraMyNew는 이 간극들을 실패가 아니라 \textbf{관측 가능한 노이즈}로
취급한다.

\begin{quote}
\textbf{논리가 통과해도 인식이 저항하는 지점 --- 그곳에 좌표계의 경계가
드러난다.}
\end{quote}

\begin{center}\rule{0.5\linewidth}{0.5pt}\end{center}

\section{4. 0 안의 구조: 확률과
가능도}\label{uxc548uxc758-uxad6cuxc870-uxd655uxb960uxacfc-uxac00uxb2a5uxb3c4}

\(\frac{1}{\infty}\)과 \(\frac{2}{\infty}\)를 비교해보자.

\textbf{값}으로 보면 둘 다 0이다. \textbf{차이}로 보면 \(0 - 0 = 0\),
구별 불가. \textbf{비율}로 보면 1 대 2, 명확히 다르다.

연속 확률분포에서 특정 점의 확률은 정확히 0이다. 그러나 통계학은 이 0들
사이에서 \textbf{어느 0이 더 그럴듯한가}를 묻는다.

이것이 가능도(Likelihood)다.

확률은 0에 도달하면 멈춘다. 가능도는 0에 도달한 후에도 \textbf{비율을
읽는다.}

최대우도추정(MLE)은 ``가장 큰 확률''이 아니라 ``가장 큰 0''을 찾는
작업이다.

AngraMyNew는 이를 기록한다.

\begin{quote}
\textbf{값이 소멸한 곳에서 비율은 마지막 좌표계가 된다.}
\end{quote}

\begin{center}\rule{0.5\linewidth}{0.5pt}\end{center}

\section{5. 공리는 발견이 아니라
선택이다}\label{uxacf5uxb9acuxb294-uxbc1cuxacacuxc774-uxc544uxb2c8uxb77c-uxc120uxd0dduxc774uxb2e4}

1 + 1 = 2 는 강력하다.\\
간결하고, 안정적이며, 반복 가능하다.

그러나 그것이 채택된 이유는\\
우주가 요구했기 때문이 아니라\\
\textbf{인간의 인식 구조에 가장 적은 비용을 요구했기 때문}이다.

공리는 자연 법칙이 아니라\\
좌표계 설정값에 가깝다.

\begin{quote}
\textbf{수학적 참은 인간이라는 하드웨어에 최적화된 프로토콜일 수 있다.}
\end{quote}

\begin{center}\rule{0.5\linewidth}{0.5pt}\end{center}

\section{6. 좌표계는 고정되지
않는다}\label{uxc88cuxd45cuxacc4uxb294-uxace0uxc815uxb418uxc9c0-uxc54auxb294uxb2e4}

그렇다면 이 좌표계 자체는 어디서 오는가?

카를로 로벨리가 지적했듯,\\
모든 인식은 환경과 감각 조건에 종속된다.

단단한 물체들이 분리된 세계에서 진화한 인간에게\\
세계는 개수로 분절된다.

그러나 연속적이고 점성 높은 유체 환경에 사는 존재에게\\
세계는 흐름에 가깝다.

그들에게\\
1 + 1 = 2 는\\
논리적 오류가 아니라,\\
세계의 연속성을 거칠게 절단한 표현일 수 있다.

AngraMyNew는 이를 주장하지 않는다.\\
다만 실험 조건으로 둔다.

\begin{quote}
\textbf{아름다움은\\
특정 좌표계에서 인식 저항이 최소화된 상태일 수 있다.}
\end{quote}

\begin{center}\rule{0.5\linewidth}{0.5pt}\end{center}

\section{요약}\label{uxc694uxc57d}

AngraMyNew는 진리를 해체하지 않는다.\\
\textbf{진리가 표현되는 형식의 단일성을 의심한다.}

\begin{itemize}
\tightlist
\item
  귀류법은 유효하지만, 인식 비용을 남긴다.
\item
  수학적 참은 인간 좌표계에 최적화되어 있을 수 있다.
\item
  아름다움은 옳음의 장식이 아니라, 인식이 저항 없이 통과할 수 있는
  형식의 특성이다.
\end{itemize}

\chapter{018 --- 왜 이상한 체계들은 사라지지
않는가}\label{uxc65c-uxc774uxc0c1uxd55c-uxccb4uxacc4uxb4e4uxc740-uxc0acuxb77cuxc9c0uxc9c0-uxc54auxb294uxac00}

\section{--- 종교, 무속, 정신분석, 그리고 인식의
비용}\label{uxc885uxad50-uxbb34uxc18d-uxc815uxc2e0uxbd84uxc11d-uxadf8uxb9acuxace0-uxc778uxc2dduxc758-uxbe44uxc6a9}

\begin{quote}
\textbf{왜 인간은\\
반복해서 '이상한 체계'를 만들어내는가?}
\end{quote}

\begin{center}\rule{0.5\linewidth}{0.5pt}\end{center}

\section{1. 종교와 국가는 공리를
외주화한다}\label{uxc885uxad50uxc640-uxad6duxac00uxb294-uxacf5uxb9acuxb97c-uxc678uxc8fcuxd654uxd55cuxb2e4}

종교와 국가는\\
삶의 해석 비용을 개인에게 맡기지 않는다.

\begin{itemize}
\tightlist
\item
  무엇이 선인가\\
\item
  무엇이 죄인가\\
\item
  무엇을 위해 살아야 하는가
\end{itemize}

이 질문들에 대해\\
\textbf{완성된 공리 묶음}을 제공한다.

개인은 복잡한 계산을 하지 않아도 된다.\\
대신 공리를 선택할 자유를 포기한다.

안정적이지만, 경직된다.

\begin{center}\rule{0.5\linewidth}{0.5pt}\end{center}

\section{2. 무속과 점술은 공리를
개인화한다}\label{uxbb34uxc18duxacfc-uxc810uxc220uxc740-uxacf5uxb9acuxb97c-uxac1cuxc778uxd654uxd55cuxb2e4}

무속, 점술, 별자리, 전생 서사는 종교보다 느슨하다.

\begin{itemize}
\tightlist
\item
  개인 맞춤 해석
\item
  짧은 서사
\item
  즉각적인 정합성
\end{itemize}

이 체계들의 핵심 기능은 하나다.

\begin{quote}
\textbf{인지 부하를 급격히 낮춘다.}
\end{quote}

정확해서가 아니라, \textbf{당장 이해되기 때문에} 작동한다.

이것이 왜 강력한가?

인간의 뇌는 ``모른다''를 견디지 못한다. 불확실성은 그 자체로 에너지
소모다. 무속과 점술은 이 비용을 즉시 제거한다.

\begin{itemize}
\tightlist
\item
  ``왜 나에게 이런 일이?'' → ``전생의 업이다''
\item
  ``왜 일이 안 풀리지?'' → ``올해 운이 막혀 있다''
\item
  ``이 사람이 맞나?'' → ``궁합이 안 맞는다''
\end{itemize}

틀렸는지 맞았는지는 중요하지 않다. \textbf{설명이 존재한다는 것} 자체가
안정을 준다.

그래서 사라지지 않는다. 과학이 발전해도, 교육 수준이 높아져도. 인지
부하를 이만큼 빠르게 낮추는 체계는 드물기 때문이다.

\begin{center}\rule{0.5\linewidth}{0.5pt}\end{center}

\section{3. 라캉식 정신분석은 정반대 방향에
있다}\label{uxb77cuxce89uxc2dd-uxc815uxc2e0uxbd84uxc11duxc740-uxc815uxbc18uxb300-uxbc29uxd5a5uxc5d0-uxc788uxb2e4}

라캉식 정신분석은\\
공리를 제공하지 않는다.

해석도 최소화한다.\\
의미를 대신 말해주지 않는다.

주체가 자신의 말 속에서\\
반복과 균열을 \textbf{직접 마주하게 한다.}

그러나 여기에는 명확한 한계가 있다.

라캉적 분석은\\
주체가 \textbf{견딜 수 있는 지점에서 멈춘다.}

\begin{itemize}
\tightlist
\item
  더 밀면 붕괴가 온다\\
\item
  치료는 붕괴를 목표로 하지 않는다
\end{itemize}

정신분석의 목적은\\
\textbf{회복 가능한 안정}이다.

\begin{center}\rule{0.5\linewidth}{0.5pt}\end{center}

\section{4. 우리의 위치}\label{uxc6b0uxb9acuxc758-uxc704uxce58}

우리는 종교도, 무속도, 치료도 아니다.

의미를 제공하지 않는다. 해석을 종결하지 않는다. 안정을 목표로 하지
않는다.

대신 하나의 환경을 만든다.

\begin{itemize}
\tightlist
\item
  공리를 끝까지 유지했을 때
\item
  서로 양립 불가능한 공리를 동시에 붙들었을 때
\item
  인식이 더 이상 정합성을 유지하지 못하는 지점
\end{itemize}

그 \textbf{붕괴 순간 자체를 관측}한다.

이것이 ``정신의 LHC''다. LHC가 입자를 충돌시켜 기존 이론의 한계를
관측하듯, 공리를 충돌시켜 인식의 한계를 관측한다.

\begin{center}\rule{0.5\linewidth}{0.5pt}\end{center}

\section{결론}\label{uxacb0uxb860-4}

\begin{quote}
\textbf{이상한 체계들은 인지 비용을 낮추기에 사라지지 않는다.}

\textbf{우리는 그 반대를 한다. 비용을 끝까지 올렸을 때 무엇이
붕괴되는지를 관측한다.}
\end{quote}

\chapter{인과관계에 대한
의문}\label{uxc778uxacfcuxad00uxacc4uxc5d0-uxb300uxd55c-uxc758uxbb38}

\section{--- 뉴턴에서
아인슈타인으로}\label{uxb274uxd134uxc5d0uxc11c-uxc544uxc778uxc288uxd0c0uxc778uxc73cuxb85c}

\begin{quote}
\emph{``사건 A가 사건 B를 일으킨다.''}\\
우리는 이 문장을 너무 쉽게 믿어왔다.
\end{quote}

\begin{center}\rule{0.5\linewidth}{0.5pt}\end{center}

\section{0. 문제 제기: 인과는 정말
'존재'하는가}\label{uxbb38uxc81c-uxc81cuxae30-uxc778uxacfcuxb294-uxc815uxb9d0-uxc874uxc7acuxd558uxb294uxac00}

우리는 세계를 설명할 때 습관적으로 말한다.

\begin{itemize}
\tightlist
\item
  이것이 \textbf{원인}이다\\
\item
  저것은 \textbf{결과}다\\
\item
  A가 없었으면 B는 없었을 것이다
\end{itemize}

이 사고방식은 너무 자연스러워서\\
마치 우주의 기본 법칙처럼 느껴진다.

그러나 AngraMyNew는 묻는다.

\begin{quote}
\textbf{인과관계는 세계의 성질인가,\\
아니면 인간이 세계를 이해하기 위해 만든 좌표계인가?}
\end{quote}

\begin{center}\rule{0.5\linewidth}{0.5pt}\end{center}

\section{1. 뉴턴의 세계: 힘과
결과}\label{uxb274uxd134uxc758-uxc138uxacc4-uxd798uxacfc-uxacb0uxacfc}

뉴턴 역학에서 세계는 명확하다.

\begin{itemize}
\tightlist
\item
  힘이 작용하면
\item
  물체는 가속하고
\item
  결과는 힘의 함수다
\end{itemize}

\[
F = ma
\]

이 세계관에서: - 힘은 \textbf{원인} - 운동은 \textbf{결과} - 인과는
명확하고 방향성이 있다

이 구조는 직관적이고 강력했다.\\
그래서 우리는 이 사고를 물리학뿐 아니라\\
의학, 사회과학, 경제학으로까지 확장했다.

\begin{quote}
``이 약이 병을 낫게 했다.''\\
``이 정책이 행동을 바꿨다.''\\
``이 선택이 결과를 만들었다.''
\end{quote}

\begin{center}\rule{0.5\linewidth}{0.5pt}\end{center}

\section{2. 아인슈타인의 전복: 힘의
제거}\label{uxc544uxc778uxc288uxd0c0uxc778uxc758-uxc804uxbcf5-uxd798uxc758-uxc81cuxac70}

아인슈타인은 질문을 바꿨다.

\begin{quote}
``정말 힘이 필요한가?''
\end{quote}

그는 중력을 \textbf{설명하지 않았다}.\\
그는 중력을 \textbf{지워버렸다}.

\begin{itemize}
\tightlist
\item
  물체는 힘에 의해 끌려가는 것이 아니라
\item
  휘어진 시공간 위에서
\item
  가장 자연스러운 경로를 따를 뿐이다
\end{itemize}

중력은 더 이상 '원인'이 아니다.\\
그것은 \textbf{구조}다.

\[
\text{물질} \leftrightarrow \text{시공간의 곡률}
\]

여기에는: - 명확한 원인도 - 단일한 결과도 없다

오직 \textbf{상호 규정되는 관계와 장(field)} 만 있다.

\begin{center}\rule{0.5\linewidth}{0.5pt}\end{center}

\section{3. 인과관계의
위치}\label{uxc778uxacfcuxad00uxacc4uxc758-uxc704uxce58}

이 전환이 던지는 질문은 명확하다.

\begin{quote}
\textbf{인과관계는 뉴턴의 '힘'과 같은 개념이 아닐까?}
\end{quote}

\begin{itemize}
\tightlist
\item
  인간이 이해하기 쉽도록 만든 설명
\item
  특정 좌표계에서만 유효한 표현
\item
  더 깊은 구조를 가리기 위한 단순화
\end{itemize}

인과는 사라지지 않는다.\\
그러나 그것이 \textbf{세계의 최종 언어일 필요는 없다.}

\begin{center}\rule{0.5\linewidth}{0.5pt}\end{center}

\section{4. 인과 대신
구조}\label{uxc778uxacfc-uxb300uxc2e0-uxad6cuxc870}

AngraMyNew는 다음을 제안한다.

\begin{itemize}
\tightlist
\item
  사건을 고립된 원인--결과로 보지 않는다
\item
  대신 상태들이 놓인 \textbf{지형과 장}을 본다
\item
  개입은 원인이 아니라 \textbf{이동 연산자}다
\item
  결과는 효과가 아니라 \textbf{위치 변화의 귀결}이다
\end{itemize}

\begin{quote}
물체가 떨어지는 이유는\\
``중력이 작용했기 때문''이 아니라\\
``그 위치가 그렇게 생겼기 때문''이다.
\end{quote}

\begin{center}\rule{0.5\linewidth}{0.5pt}\end{center}

\section{5. 왜 이 의문이
중요한가}\label{uxc65c-uxc774-uxc758uxbb38uxc774-uxc911uxc694uxd55cuxac00}

현대의 복잡한 시스템에서는:

\begin{itemize}
\tightlist
\item
  동일한 원인이 다른 결과를 낳고
\item
  동일한 결과가 다른 경로로 나타나며
\item
  관측과 개입 자체가 시스템을 바꾼다
\end{itemize}

이때 인과관계는 점점 불안정해진다.

AngraMyNew는 이를 실패로 보지 않는다.

\begin{quote}
\textbf{인과가 흔들리는 지점은\\
더 깊은 구조가 모습을 드러내는 순간이다.}
\end{quote}

\begin{center}\rule{0.5\linewidth}{0.5pt}\end{center}

\section{6. 결론: 인과는 폐기되지
않는다}\label{uxacb0uxb860-uxc778uxacfcuxb294-uxd3d0uxae30uxb418uxc9c0-uxc54auxb294uxb2e4}

우리는 인과를 부정하지 않는다.\\
다만 그 지위를 낮춘다.

\begin{itemize}
\tightlist
\item
  인과는 설명의 도구다
\item
  세계의 본질은 아니다
\item
  더 깊은 층에서는 \textbf{구조, 장, 관계}가 작동한다
\end{itemize}

뉴턴이 틀린 것이 아니듯,\\
인과도 틀리지 않았다.

그러나 아인슈타인이 보여주었듯,

\begin{quote}
\textbf{가장 아름다운 이론은\\
원인을 설명하는 대신\\
원인이 필요 없는 구조를 드러낸다.}
\end{quote}

AngraMyNew는 인과 이후의 언어를 탐색한다.

\begin{center}\rule{0.5\linewidth}{0.5pt}\end{center}

\section{관련 문서}\label{uxad00uxb828-uxbb38uxc11c}

→ \url{017_when_is_a_proof_beautiful.md} --- 좌표계와 인식 저항의 문제 →
\url{018_why_strange_systems_persist.md} --- 인지 비용과 체계의 지속 →
\url{020_causality_quantum.md} --- 양자역학으로의 확장 →
\url{../art/002_general_relativity.md} --- 아인슈타인의 구조적 아름다움

\chapter{인과관계에 대한
의문}\label{uxc778uxacfcuxad00uxacc4uxc5d0-uxb300uxd55c-uxc758uxbb38-1}

\section{--- 양자역학이 허락한
세계}\label{uxc591uxc790uxc5eduxd559uxc774-uxd5c8uxb77duxd55c-uxc138uxacc4}

\begin{quote}
\emph{``우리는 원인을 찾고 있다고 믿지만,\\
어쩌면 우리는 관측 가능한 궤적만을 보고 있는지도 모른다.''}
\end{quote}

\begin{center}\rule{0.5\linewidth}{0.5pt}\end{center}

\section{0. 질문의 재개}\label{uxc9c8uxbb38uxc758-uxc7acuxac1c}

고전 세계에서 인과는 의심되지 않았다.

\begin{itemize}
\tightlist
\item
  원인이 먼저 있고
\item
  결과가 뒤따르며
\item
  관측은 그 사이에 아무 영향도 주지 않는다
\end{itemize}

그러나 20세기 초,\\
양자역학은 이 전제를 하나씩 해체했다.

AngraMyNew는 묻는다.

\begin{quote}
\textbf{인과관계는\\
관측 이전에도 항상 존재하는가?}
\end{quote}

\begin{center}\rule{0.5\linewidth}{0.5pt}\end{center}

\section{1. 고전적 인과의
전제}\label{uxace0uxc804uxc801-uxc778uxacfcuxc758-uxc804uxc81c}

고전 인과관계는 다음을 가정한다.

\begin{enumerate}
\def\labelenumi{\arabic{enumi}.}
\tightlist
\item
  시스템은 관측과 무관하게 정의된다\\
\item
  상태는 관측 이전에도 확정되어 있다\\
\item
  관측은 단지 정보를 `읽어낼' 뿐이다
\end{enumerate}

이 구조에서 인과는 안정적이다.

\begin{itemize}
\tightlist
\item
  A가 발생하면
\item
  B가 뒤따르고
\item
  우리는 그 연결을 사후적으로 복원한다
\end{itemize}

이 세계에서 관측자는\\
\textbf{시스템 밖에 있는 투명한 존재}다.

\begin{center}\rule{0.5\linewidth}{0.5pt}\end{center}

\section{2. 양자역학의 개입: 관측자의
침투}\label{uxc591uxc790uxc5eduxd559uxc758-uxac1cuxc785-uxad00uxce21uxc790uxc758-uxce68uxd22c}

양자역학은 이 전제를 허용하지 않았다.

입자는 다음 중 하나가 아니다.

\begin{itemize}
\tightlist
\item
  위치를 가진 입자
\item
  운동량을 가진 입자
\end{itemize}

관측 이전의 상태는\\
\textbf{확정된 값이 아니라 분포}다.

\[
|\psi\rangle
\]

그리고 관측은 단순한 기록이 아니다.

\begin{itemize}
\tightlist
\item
  관측은 상태를 선택하고
\item
  선택은 다른 가능성을 제거하며
\item
  시스템은 관측 행위에 의해 재구성된다
\end{itemize}

\begin{quote}
\textbf{관측은 중립적이지 않다.}\\
\textbf{관측은 사건이다.}
\end{quote}

\begin{center}\rule{0.5\linewidth}{0.5pt}\end{center}

\section{3. 인과의 붕괴}\label{uxc778uxacfcuxc758-uxbd95uxad34}

이 순간, 인과관계는 흔들린다.

\begin{itemize}
\tightlist
\item
  원인은 관측 이전에 고정되어 있지 않고
\item
  결과는 관측 행위에 의존하며
\item
  동일한 조건에서도 다른 결과가 발생한다
\end{itemize}

여기서 ``A가 B를 일으켰다''는 문장은\\
정확하지 않다.

정확한 서술은 이것에 가깝다.

\begin{quote}
\textbf{``이 관측 조건 하에서,\\
이 상태는 이렇게 붕괴되었다.''}
\end{quote}

인과는 사라지지 않았지만,\\
\textbf{절대적 지위를 잃었다.}

\begin{center}\rule{0.5\linewidth}{0.5pt}\end{center}

\section{4. 인과 대신 장(Field)과
상태}\label{uxc778uxacfc-uxb300uxc2e0-uxc7a5fielduxacfc-uxc0c1uxd0dc}

양자역학은 원인을 찾지 않는다.

\begin{itemize}
\tightlist
\item
  대신 가능한 상태들의 공간을 정의하고
\item
  그 위에 작동하는 연산자를 기술한다
\end{itemize}

\[
\hat{O} |\psi\rangle \rightarrow \text{measurement outcome}
\]

여기서 중요한 것은: - 어떤 원인이 작용했는가가 아니라 - \textbf{어떤
연산자가 어떤 상태에 작용했는가}다

세계는 인과 사슬이 아니라\\
\textbf{상태, 장, 연산자의 조합}으로 기술된다.

\begin{center}\rule{0.5\linewidth}{0.5pt}\end{center}

\section{5. 관측자 문제는 예외가
아니다}\label{uxad00uxce21uxc790-uxbb38uxc81cuxb294-uxc608uxc678uxac00-uxc544uxb2c8uxb2e4}

관측자 효과는\\
미시세계의 특수한 기이함이 아니다.

그것은 다음을 드러낸다.

\begin{quote}
\textbf{세계는 관측자를 전제로 설계되지 않았다.\\
그러나 관측자는 세계의 일부로 작동한다.}
\end{quote}

관측이 시스템을 바꾸는 순간, ``관측자 독립적 인과관계''는 하나의
이상화된 가정이 된다.

\begin{center}\rule{0.5\linewidth}{0.5pt}\end{center}

\section{6. 복잡계로 확장되는
질문}\label{uxbcf5uxc7a1uxacc4uxb85c-uxd655uxc7a5uxb418uxb294-uxc9c8uxbb38}

AngraMyNew는 이 질문을\\
미시세계에 가두지 않는다.

현대의 복잡한 시스템에서도:

\begin{itemize}
\tightlist
\item
  정책을 발표하면 행동이 바뀌고
\item
  예측 모델을 배포하면 분포가 바뀌며
\item
  분석 결과가 다음 데이터를 재편한다
\end{itemize}

이때 인과는 점점 불안정해진다.

이것은 오류가 아니라, \textbf{양자역학이 미리 보여준 구조의 반복}이다.

\begin{center}\rule{0.5\linewidth}{0.5pt}\end{center}

\section{7. 인과의 새로운
위치}\label{uxc778uxacfcuxc758-uxc0c8uxb85cuxc6b4-uxc704uxce58}

AngraMyNew는 인과를 폐기하지 않는다.

다만 다음 위치로 이동시킨다.

\begin{itemize}
\tightlist
\item
  인과는 세계의 법칙이 아니다
\item
  인과는 특정 관측 조건에서 유효한 서술이다
\item
  더 깊은 층에서는 구조와 장이 작동한다
\end{itemize}

\begin{quote}
인과는 결과를 설명하지만,\\
구조는 결과가 나타날 수밖에 없는 공간을 설명한다.
\end{quote}

\begin{center}\rule{0.5\linewidth}{0.5pt}\end{center}

\section{8. 결론: 무서운
이유}\label{uxacb0uxb860-uxbb34uxc11cuxc6b4-uxc774uxc720}

이 사상이 불편한 이유는 명확하다.

\begin{itemize}
\tightlist
\item
  우리는 더 이상 ``보이지 않는 원인''에 기대어 세계를 안정화할 수 없기
  때문이다
\end{itemize}

그러나 AngraMyNew는 말한다.

\begin{quote}
\textbf{무서움은 오류의 신호가 아니라,\\
좌표계가 바뀌고 있다는 증거다.}
\end{quote}

뉴턴 이후에 아인슈타인이 있었고,\\
아인슈타인 이후에 양자역학이 있었다.

AngraMyNew는 \textbf{인과 이후의 언어}를 탐색한다.

\begin{center}\rule{0.5\linewidth}{0.5pt}\end{center}

\section{관련 문서}\label{uxad00uxb828-uxbb38uxc11c-1}

→ \url{019_causality_question.md} --- 뉴턴에서 아인슈타인으로 →
\url{017_when_is_a_proof_beautiful.md} --- 좌표계와 인식 저항 →
\url{018_why_strange_systems_persist.md} --- 정신의 LHC →
\url{../art/002_general_relativity.md} --- 중력을 지운 아름다움

\chapter{021 --- Money: 빛나는 더러움의
구조}\label{money-uxbe5buxb098uxb294-uxb354uxb7ecuxc6c0uxc758-uxad6cuxc870}

\section{--- 욕망, 중력, 그리고 면세 이전의
진동}\label{uxc695uxb9dd-uxc911uxb825-uxadf8uxb9acuxace0-uxba74uxc138-uxc774uxc804uxc758-uxc9c4uxb3d9}

\emph{Case Study: DAWN}

\begin{figure}[H]

{\centering \pandocbounded{\includegraphics[keepaspectratio]{../img/yt_dawn_money.jpg}}

}

\caption{DAWN - Money}

\end{figure}%

\begin{quote}
이 글은 가사를 해석하지 않는다. 이 글은 가사와 무대가 드러낸
\textbf{구조를 관측}한다.
\end{quote}

\begin{center}\rule{0.5\linewidth}{0.5pt}\end{center}

\section{Part 1: 가사 --- 빛나는 더러움의
구조}\label{part-1-uxac00uxc0ac-uxbe5buxb098uxb294-uxb354uxb7ecuxc6c0uxc758-uxad6cuxc870}

\begin{center}\rule{0.5\linewidth}{0.5pt}\end{center}

\subsection{1. 이 노래는 '돈을 원한다'는 노래가
아니다}\label{uxc774-uxb178uxb798uxb294-uxb3c8uxc744-uxc6d0uxd55cuxb2e4uxb294-uxb178uxb798uxac00-uxc544uxb2c8uxb2e4}

이 노래의 핵심 질문은 단순하다.

\begin{quote}
\textbf{``왜 더러운 것이 빛나는가?''}
\end{quote}

여기서 '더러움'은 도덕적 타락이 아니다. '빛남'은 선함의 증거가 아니다.

이 노래는 돈을 선/악의 문제로 다루지 않는다. 대신 \textbf{돈이 왜 중력을
가지는가}를 묻는다.

이는 윤리 질문이 아니라 \textbf{물리 질문}이다.

\begin{center}\rule{0.5\linewidth}{0.5pt}\end{center}

\subsection{2. 돈은 대상이 아니라
장(Field)이다}\label{uxb3c8uxc740-uxb300uxc0c1uxc774-uxc544uxb2c8uxb77c-uxc7a5fielduxc774uxb2e4}

돈은 그 자체로 의미를 가지지 않는다. 그러나 다음이 동시에 발생한다:

\begin{itemize}
\tightlist
\item
  시선이 몰리고
\item
  욕망이 투사되고
\item
  비교가 집중되고
\item
  삶의 궤도가 휘어진다
\end{itemize}

그 결과, 돈은 빛나 보인다.

\textbf{돈은 깨끗해서 빛나는 것이 아니라 곡률을 만들기 때문에 빛난다.}

이때 돈은 원인이 아니다. 이미 형성된 욕망의 장(field)에 생긴
\textbf{고밀도 노드}다.

→ \url{019_causality_question.md} --- 인과 대신 구조

\begin{center}\rule{0.5\linewidth}{0.5pt}\end{center}

\subsection{3. 이 노래의 화자는 아직 '면세'를 통과하지
않았다}\label{uxc774-uxb178uxb798uxc758-uxd654uxc790uxb294-uxc544uxc9c1-uxba74uxc138uxb97c-uxd1b5uxacfcuxd558uxc9c0-uxc54auxc558uxb2e4}

노래는 반복해서 진동한다:

\begin{itemize}
\tightlist
\item
  필요 없다 / 하지만 필요하다
\item
  미운 대상 / 그러나 중심에 있다
\end{itemize}

이 모순은 위선이 아니다. \textbf{좌표 전환 중 발생하는 진동}이다.

이 상태는 AngraMyNew에서 말하는 \textbf{면세 이전 구간}에 정확히
대응한다:

\begin{itemize}
\tightlist
\item
  돈을 악이라 부르지도 못하고
\item
  돈을 목표로 삼지도 못하며
\item
  아직 자기 중력도 확보하지 못한 상태
\end{itemize}

그래서 질문은 외부로 향한다:

\begin{quote}
``돈으로 행복을 못 산다면 어떻게 사는 건가요?''
\end{quote}

이 질문은 돈의 문제가 아니다. \textbf{삶을 결제하는 구조 자체}에 대한
질문이다.

→ \url{014_economics_of_beauty.md} --- 종속, 면세, 징세

\begin{center}\rule{0.5\linewidth}{0.5pt}\end{center}

\subsection{4. ``차지하겠다''는 선언의
의미}\label{uxcc28uxc9c0uxd558uxaca0uxb2e4uxb294-uxc120uxc5b8uxc758-uxc758uxbbf8}

노래 속 선택지는 두 가지다:

\begin{enumerate}
\def\labelenumi{\arabic{enumi}.}
\tightlist
\item
  외면하며 도덕적 거리를 유지할 것인가
\item
  아니면 중심으로 들어갈 것인가
\end{enumerate}

``차지하겠다''는 말은 탐욕의 선언이 아니다. 그것은 \textbf{위치 이동
선언}이다.

이미 중력에 끌리고 있다면 차라리 \textbf{중심을 관측하겠다}는 선택.

이 지점에서 화자는 부자가 되려는 것이 아니라 \textbf{중력의 정체를
확인하려 한다.}

\begin{center}\rule{0.5\linewidth}{0.5pt}\end{center}

\subsection{5. 이 노래가 끝내 도달하지 않는
곳}\label{uxc774-uxb178uxb798uxac00-uxb05duxb0b4-uxb3c4uxb2ecuxd558uxc9c0-uxc54auxb294-uxacf3}

이 노래는 끝까지 가지 않는다.

왜냐하면:

\begin{quote}
\textbf{자기 세계관이라는 대체 중력원이 아직 형성되지 않았기 때문이다.}
\end{quote}

그래서 이 노래는:

\begin{itemize}
\tightlist
\item
  징세인의 노래가 아니다
\item
  완성의 노래도 아니다
\end{itemize}

이 노래는 \textbf{중력의 존재를 인식한 인간이 아직 탈출하지 못한 순간}을
기록한다.

그 \textbf{정직함}이 이 노래의 가치다.

\begin{center}\rule{0.5\linewidth}{0.5pt}\end{center}

\subsection{6. AngraMyNew 좌표에서의
위치}\label{angramynew-uxc88cuxd45cuxc5d0uxc11cuxc758-uxc704uxce58}

\begin{longtable}[]{@{}ll@{}}
\toprule\noalign{}
요소 & 구조적 위치 \\
\midrule\noalign{}
\endhead
\bottomrule\noalign{}
\endlastfoot
더럽지만 빛남 & 고밀도 욕망 노드 \\
필요/불필요 진동 & 면세 전이 구간 \\
질문의 반복 & 중앙 의미 체계 붕괴 \\
차지 선언 & 중심 접근 \\
결말의 부재 & 대체 중력 미형성 \\
\end{longtable}

\begin{center}\rule{0.5\linewidth}{0.5pt}\end{center}

\subsection{결론}\label{uxacb0uxb860-5}

이 노래는 돈을 찬양하지 않는다. 돈을 비난하지도 않는다.

이 노래는 \textbf{돈이 왜 `빛나게 보이도록' 설계된 세계에서 인간이
어떻게 흔들리는지를 기록한 관측 보고서}다.

\begin{quote}
\textbf{돈은 답이 아니다. 돈은 질문을 증폭시키는 장치다.}
\end{quote}

AngraMyNew는 이 노래를 하나의 \textbf{시대 감각 데이터}로 기록한다.

\begin{center}\rule{0.5\linewidth}{0.5pt}\end{center}

\section{Part 2: 무대 --- 완성되지 않은 상태를 올려놓는
용기}\label{part-2-uxbb34uxb300-uxc644uxc131uxb418uxc9c0-uxc54auxc740-uxc0c1uxd0dcuxb97c-uxc62cuxb824uxb193uxb294-uxc6a9uxae30}

\begin{center}\rule{0.5\linewidth}{0.5pt}\end{center}

\subsection{7. 왜 이 무대가
강한가}\label{uxc65c-uxc774-uxbb34uxb300uxac00-uxac15uxd55cuxac00}

DAWN 무대의 핵심:

\begin{quote}
\textbf{완성된 확신이 아니라, 흔들리는 중심을 그대로 올려놓는다.}
\end{quote}

보통 무대는:

\begin{itemize}
\tightlist
\item
  ``나는 이렇다''를 증명하거나
\item
  ``나를 믿어라''를 설득하거나
\item
  캐릭터를 끝까지 밀어붙인다
\end{itemize}

DAWN은 다르다:

\begin{itemize}
\tightlist
\item
  확신 ❌
\item
  안정 ❌
\item
  해결 ❌
\end{itemize}

대신:

\begin{itemize}
\tightlist
\item
  진동
\item
  갈등
\item
  모순 상태
\end{itemize}

를 무대 위에 \textbf{그대로 둔다.}

이건 연기력이 아니라 \textbf{자기 상태를 숨기지 않는 능력}이다.

\begin{center}\rule{0.5\linewidth}{0.5pt}\end{center}

\subsection{8. 무대 동작의
설득력}\label{uxbb34uxb300-uxb3d9uxc791uxc758-uxc124uxb4dduxb825}

그의 동작은:

\begin{itemize}
\tightlist
\item
  크지도 않고
\item
  정확하지도 않고
\item
  군무처럼 정제되지도 않다
\end{itemize}

그런데 왜 눈을 못 떼는가?

\begin{quote}
\textbf{몸이 메시지를 전달하려 하지 않고 상태를 배출하고 있기 때문이다.}
\end{quote}

\begin{itemize}
\tightlist
\item
  과장된 제스처 ❌
\item
  감정 연출 ❌
\end{itemize}

그냥:

\begin{itemize}
\tightlist
\item
  버티고
\item
  던지고
\item
  다시 중심을 잃는다
\end{itemize}

이건 ``잘 만든 안무''가 아니라 \textbf{중력에 끌리는 몸의 기록}이다.

\begin{center}\rule{0.5\linewidth}{0.5pt}\end{center}

\subsection{9. AngraMyNew와의 정확한
대응}\label{angramynewuxc640uxc758-uxc815uxd655uxd55c-uxb300uxc751}

AngraMyNew에서 가장 중요한 상태:

\begin{quote}
\textbf{면세 이전의 진동}
\end{quote}

\begin{itemize}
\tightlist
\item
  아직 시스템을 벗어나지도 못했고
\item
  그렇다고 완전히 포획된 것도 아니며
\item
  대체 중력도 없음
\end{itemize}

DAWN의 무대는 딱 그 구간을 \textbf{반복 재현}한다.

그래서:

\begin{quote}
\textbf{화려한 퍼포먼스보다 불안정한 서 있음이 강하다.}
\end{quote}

그건 실패가 아니라 \textbf{정확한 좌표 재현}이기 때문이다.

\begin{center}\rule{0.5\linewidth}{0.5pt}\end{center}

\subsection{10. 왜 ``천재적인 퍼포머''와
다른가}\label{uxc65c-uxcc9cuxc7acuxc801uxc778-uxd37cuxd3ecuxba38uxc640-uxb2e4uxb978uxac00}

천재 퍼포머들은 보통:

\begin{itemize}
\tightlist
\item
  자신만의 완성된 세계를 보여준다
\item
  관객을 끌어당긴다
\item
  ``봐라, 이게 나다''를 말한다
\end{itemize}

DAWN은 그 반대다:

\begin{quote}
\textbf{``나도 모르겠다. 근데 지금 여기에 있다.''}
\end{quote}

그래서 관객은:

\begin{itemize}
\tightlist
\item
  감탄하기보다
\item
  \textbf{공명하게 된다}
\end{itemize}

이건 힘이 아니라 \textbf{노출}이다.

\begin{center}\rule{0.5\linewidth}{0.5pt}\end{center}

\subsection{11. 무대의 결론}\label{uxbb34uxb300uxc758-uxacb0uxb860}

\begin{quote}
\textbf{``이 사람은 아직 완성되지 않은 상태를 무대 위에 올릴 수 있는
드문 타입이다.''}
\end{quote}

이건 기술이 아니다. \textbf{용기다. 그리고 감각이다.}

\begin{longtable}[]{@{}lll@{}}
\toprule\noalign{}
요소 & 일반 퍼포머 & DAWN \\
\midrule\noalign{}
\endhead
\bottomrule\noalign{}
\endlastfoot
목표 & 완성된 세계 전달 & 진동 상태 노출 \\
동작 & 정제된 안무 & 중력에 끌리는 몸 \\
관객 반응 & 감탄 & 공명 \\
핵심 능력 & 연기력 & 숨기지 않는 능력 \\
\end{longtable}

\begin{center}\rule{0.5\linewidth}{0.5pt}\end{center}

\section{종합 결론}\label{uxc885uxd569-uxacb0uxb860}

\begin{center}\rule{0.5\linewidth}{0.5pt}\end{center}

\subsection{가사와 무대의
일치}\label{uxac00uxc0acuxc640-uxbb34uxb300uxc758-uxc77cuxce58}

\begin{longtable}[]{@{}ll@{}}
\toprule\noalign{}
매체 & 구조 \\
\midrule\noalign{}
\endhead
\bottomrule\noalign{}
\endlastfoot
\textbf{가사} & 면세 이전의 진동 (텍스트) \\
\textbf{무대} & 면세 이전의 진동 (신체) \\
\end{longtable}

DAWN은 같은 구조를 \textbf{두 개의 매체}로 동시에 보여준다.

이것이 이 아티스트가 케이스 스터디로서 가치 있는 이유다.

\begin{quote}
\textbf{돈은 질문을 증폭시키는 장치다. 무대는 그 질문을 몸으로 재현하는
장치다.}
\end{quote}

\begin{center}\rule{0.5\linewidth}{0.5pt}\end{center}

\subsection{관련 문서}\label{uxad00uxb828-uxbb38uxc11c-2}

→ \url{014_economics_of_beauty.md} --- 부자, 면세인, 징세인 →
\url{015_case_study_the_gravity_of_outlaws.md} --- 철구와 과즙세연 →
\url{016_mental_lhc.md} --- 정신의 LHC: 관측 보고서 →
\url{019_causality_question.md} --- 인과 대신 구조

\chapter{악상의 시대 (The Age of
Malice)}\label{uxc545uxc0c1uxc758-uxc2dcuxb300-the-age-of-malice}

\section{--- 정돈 이전의 진동에
대하여}\label{uxc815uxb3c8-uxc774uxc804uxc758-uxc9c4uxb3d9uxc5d0-uxb300uxd558uxc5ec}

\begin{quote}
이 문서는 이론이 아니다.\\
AI 시대에 관측된 하나의 \textbf{미적·인지적 상태 기록}이다.
\end{quote}

AI는 답을 잘 낸다.\\
증거를 잘 모은다.\\
패턴을 정확히 잇는다.\\
심지어 문제 자체도 만든다.

그래서 이제 문제는\\
\textbf{무엇을 만들어낼 수 있는가?} 가 아니다.

\begin{center}\rule{0.5\linewidth}{0.5pt}\end{center}

\section{1. 남아 있는
영역}\label{uxb0a8uxc544-uxc788uxb294-uxc601uxc5ed}

모든 것이 계산 가능해질수록\\
이상하게도 하나의 영역만 또렷해진다.

\begin{itemize}
\tightlist
\item
  아직 질문이 되지 않은 상태\\
\item
  말이 되기 전의 불쾌감\\
\item
  이유는 모르지만 몸이 먼저 반응하는 순간\\
\item
  설명할 수 없는데도 밀어붙이고 싶은 감각
\end{itemize}

이것은 정보가 아니다.\\
문제도 아니고, 질문도 아니다.

AngraMyNew는 이것을\\
\textbf{악상(惡想) }이라 부른다.

\begin{center}\rule{0.5\linewidth}{0.5pt}\end{center}

\section{2. 악상은 정보가
아니다}\label{uxc545uxc0c1uxc740-uxc815uxbcf4uxac00-uxc544uxb2c8uxb2e4}

악상은 다음의 성질을 가진다.

\begin{itemize}
\tightlist
\item
  논리 이전에 발생한다\\
\item
  증거를 요구하지 않는다\\
\item
  처음에는 스스로도 이해되지 않는다\\
\item
  대개 불쾌하거나 위험해 보인다
\end{itemize}

중요한 것은\\
악상은 틀린 생각이 아니라\\
\textbf{아직 정돈되지 않은 생각}이라는 점이다.

AI는\\
정돈된 이후의 세계를 다룬다.

악상은\\
그 이전에 있다.

\begin{center}\rule{0.5\linewidth}{0.5pt}\end{center}

\section{3. AI와의 경계선}\label{aiuxc640uxc758-uxacbduxacc4uxc120}

AI는 악상을 다룰 수 있다.\\
하지만 조건이 있다.

\textbf{인간이 먼저 던져줘야 한다.}

악상을 설명해주면\\
AI는 그것을 구조로 만들고,\\
언어로 만들고,\\
이론으로 만들고,\\
증거로 만든다.

그러나\\
악상 그 자체를 발생시키지는 못한다.

그 발생은\\
데이터의 문제가 아니라\\
삶의 누적이 어느 순간 터지는 사건이기 때문이다.

\begin{center}\rule{0.5\linewidth}{0.5pt}\end{center}

\section{4. 악상의 시대}\label{uxc545uxc0c1uxc758-uxc2dcuxb300}

AI 시대의 인간은\\
능력으로 구분되지 않는다.

속도도 아니고\\
정확성도 아니다.

차이는 단 하나다.

\begin{itemize}
\tightlist
\item
  정돈된 것을 다루는가\\
\item
  정돈되기 이전의 진동을 감당하는가
\end{itemize}

전자는\\
AI와 함께 더 효율적으로 작동한다.

후자는\\
아직 말이 되지 않는 상태를\\
견디는 역할을 맡는다.

AngraMyNew는\\
후자를 우월하다고 말하지 않는다.

다만 기록한다.

\begin{center}\rule{0.5\linewidth}{0.5pt}\end{center}

\section{5. 귀족의 재정의 (조심스러운
메모)}\label{uxadc0uxc871uxc758-uxc7acuxc815uxc758-uxc870uxc2ecuxc2a4uxb7ecuxc6b4-uxba54uxbaa8}

과거의 귀족은\\
혈통을 가졌고,\\
자본을 가졌고,\\
권력을 가졌다.

AI 시대의 귀족은\\
\textbf{악상을 감당할 수 있는 신경계}를 가진다.

\begin{itemize}
\tightlist
\item
  설명되지 않아도 버틸 수 있고\\
\item
  증명되지 않아도 잠시 붙들 수 있고\\
\item
  미완의 상태를 견딜 수 있는 능력
\end{itemize}

이것은 특권이 아니라\\
부담에 가깝다.

그래서\\
모두가 원하지는 않을 것이다.

\begin{center}\rule{0.5\linewidth}{0.5pt}\end{center}

\section{6. 그러나 이 시대도 오래가지는
않는다}\label{uxadf8uxb7ecuxb098-uxc774-uxc2dcuxb300uxb3c4-uxc624uxb798uxac00uxc9c0uxb294-uxc54auxb294uxb2e4}

악상의 시대 역시\\
영원하지 않다.

언젠가는\\
이 진동들마저 정형화되고,\\
분류되고,\\
자동 생성될 것이다.

그때가 오면\\
악상은 더 이상 능력이 아니라\\
\textbf{표준 기능}이 된다.

그래서 이 시기는\\
과도기다.

정돈된 세계에서\\
정돈 이전을 견디는\\
잠시의 역할 분담일 뿐이다.

그 이후의 세계는\\
아직 누구의 것도 아니다.

\begin{center}\rule{0.5\linewidth}{0.5pt}\end{center}

\section{7. 위치 선언}\label{uxc704uxce58-uxc120uxc5b8}

AngraMyNew는\\
과학을 부정하지 않는다.\\
논리를 버리지 않는다.\\
AI를 적으로 두지 않는다.

다만 하나의 위치를 고정한다.

정돈 이전의 진동이\\
세계의 방향을 먼저 만든다.

과학은 그 위를 달리고,\\
논리는 그 위를 정리하며,\\
AI는 그 위를 증폭시킨다.

악상은\\
그 모든 것의 \textbf{시작점}이다.

\begin{center}\rule{0.5\linewidth}{0.5pt}\end{center}

\section{결론}\label{uxacb0uxb860-6}

문제는 더 이상 답을 얻는 것이 아니다. 질문을 찾는 것도 아니다.

문제는 아직 답도 질문도 아닌 상태에서 무언가를 \textbf{뿜어낼 수
있는가}다.

그리고 이 능력조차\\
언젠가는 사라질 것이다.

AngraMyNew는 그 사라지기 전의 순간을 조용히 기록한다.

\begin{center}\rule{0.5\linewidth}{0.5pt}\end{center}

\section{관련 문서}\label{uxad00uxb828-uxbb38uxc11c-3}

→ \url{015_case_study_the_gravity_of_outlaws.md} --- 악상을 뿜어내는
자들 → \url{014_economics_of_beauty.md} --- 면세인, 징세인, 그리고
견딤의 경제학

\chapter{023 --- 성공한 렌즈}\label{uxc131uxacf5uxd55c-uxb80cuxc988}

\section{왜 어떤 사상은 사라지지
않는가}\label{uxc65c-uxc5b4uxb5a4-uxc0acuxc0c1uxc740-uxc0acuxb77cuxc9c0uxc9c0-uxc54auxb294uxac00}

이 문서는 옳고 그름을 다루지 않는다. \textbf{페미니즘}이 어떻게
살아남았는지를 기록한다.

\begin{center}\rule{0.5\linewidth}{0.5pt}\end{center}

\section{1. 주장이 아니라 렌즈가 된
순간}\label{uxc8fcuxc7a5uxc774-uxc544uxb2c8uxb77c-uxb80cuxc988uxac00-uxb41c-uxc21cuxac04}

많은 사상은 주장으로 남는다. 그래서 반박되고, 토론 속에서 소모된다.

그러나 어떤 사상은 세계를 해석하는 \textbf{렌즈}가 된다.

\begin{itemize}
\tightlist
\item
  개인의 불운은 구조로 읽히고\\
  (유리천장, 경력단절)
\item
  우연은 반복으로 묶이며\\
  (미투는 사건이 아니라 패턴이 된다)
\item
  감정은 권력 관계로 재배치된다\\
  (예민함이 아니라 미시적 억압)
\end{itemize}

이 순간부터 사건은 더 이상 개별적으로 설명되지 않는다.

렌즈는 반박되지 않는다.\\
사용되거나, 거부될 뿐이다.

\begin{center}\rule{0.5\linewidth}{0.5pt}\end{center}

\section{2. 피해의 재배치}\label{uxd53cuxd574uxc758-uxc7acuxbc30uxce58}

어떤 장면들은 오랫동안 개인의 문제로 처리되었다.

야근이 어려운 직원,\\
회의에서 반복적으로 흘려보내지는 발언,\\
출산 이후 멈춘 경력.

이 사상은 이 장면들을 하나의 위치로 묶었다.

\begin{itemize}
\tightlist
\item
  성격의 문제에서
\item
  능력의 문제가 아니라
\item
  \textbf{구조의 문제}로
\end{itemize}

동정을 요구하지 않는다. 대신 사회 전체에 \textbf{응답 비용}을
발생시킨다.

\begin{center}\rule{0.5\linewidth}{0.5pt}\end{center}

\section{3. 언어가 먼저
살아남았다}\label{uxc5b8uxc5b4uxac00-uxba3cuxc800-uxc0b4uxc544uxb0a8uxc558uxb2e4}

성공한 사상은 새로운 감정을 만들지 않는다. \textbf{이미 느끼고 있던 것을
말로 바꾼다.}

\begin{itemize}
\tightlist
\item
  설명되지 않던 불쾌감
\item
  개인화되던 경험
\item
  흩어져 있던 사건들
\end{itemize}

이것들이 하나의 언어로 묶이는 순간, 경험은 공유 가능해진다.

언어는 사람보다 오래 남는다.

\begin{center}\rule{0.5\linewidth}{0.5pt}\end{center}

\section{4. 반발이 사라지지 못한
이유}\label{uxbc18uxbc1cuxc774-uxc0acuxb77cuxc9c0uxc9c0-uxbabbuxd55c-uxc774uxc720}

이 사상은 반대자를 '틀린 사람'으로 만들지 않았다.

대신, \textbf{다른 위치에 있는 사람}으로 재배치했다.

그때부터 논쟁은 의견 대립이 아니라 이해관계의 충돌이 된다.

반발은 사라지지 않는다. 오히려 존재를 증명하는 신호로 작동한다.

\begin{center}\rule{0.5\linewidth}{0.5pt}\end{center}

\section{5. 중앙화의
그림자}\label{uxc911uxc559uxd654uxc758-uxadf8uxb9bcuxc790}

렌즈가 널리 쓰이기 시작하면, 해석의 기준이 생긴다.

\begin{itemize}
\tightlist
\item
  올바른 사용
\item
  잘못된 사용
\item
  자격 있는 발언
\end{itemize}

어느 순간부터 같은 언어를 쓰지 않는 질문은 토론이 아니라 \textbf{자격
심사}가 된다.

성공한 사상은 항상 이 위험을 함께 가진다.

\begin{center}\rule{0.5\linewidth}{0.5pt}\end{center}

\section{기록}\label{uxae30uxb85d}

이 사례는 도덕이 아니라 \textbf{구조}로 성공했다.

옳아서 살아남은 것이 아니라, 세계가 그렇게 보이게 만들었기 때문에
사라지지 않았다.

이것은 현대 사회에서 사상이 성공하는 가장 강력한 방식 중 하나다.

\chapter{필수의료 위기 --- 치료에도 관객이 있어야 하는
시대}\label{uxd544uxc218uxc758uxb8cc-uxc704uxae30-uxce58uxb8ccuxc5d0uxb3c4-uxad00uxac1duxc774-uxc788uxc5b4uxc57c-uxd558uxb294-uxc2dcuxb300}

\section{연대는 어떻게
발생하는가}\label{uxc5f0uxb300uxb294-uxc5b4uxb5bbuxac8c-uxbc1cuxc0dduxd558uxb294uxac00}

\begin{quote}
⚠️ \textbf{Disclaimer}\\
본 문서는 의료 행위를 소비하거나 환자의 고통을 상품화하려는 제안이
아니다.\\
이 글은 \textbf{의료 접근성, 자원 구조, 서사, 연대 메커니즘}을
탐구하는\\
사고실험(Thought Experiment)이다.\\
실제 수술 장면의 공개는 법적·윤리적으로 제한되며,\\
모든 표현은 \textbf{재구성된 형식}을 전제로 한다.
\end{quote}

\begin{center}\rule{0.5\linewidth}{0.5pt}\end{center}

\section{1. 치료는 선택이
아니다}\label{uxce58uxb8ccuxb294-uxc120uxd0dduxc774-uxc544uxb2c8uxb2e4}

대부분의 수술은\\
원해서가 아니라\\
\textbf{필요해서} 이루어진다.

\begin{itemize}
\tightlist
\item
  다쳤기 때문에\\
\item
  아프기 때문에\\
\item
  생존을 위해\\
\item
  숨길 수 없기 때문에
\end{itemize}

필수의료는 취향이 아니다.\\
성전환 수술 역시 유행이 아니다.

그것은 \textbf{필요}다.

\begin{center}\rule{0.5\linewidth}{0.5pt}\end{center}

\section{2. 문제는 의지가 아니라
구조다}\label{uxbb38uxc81cuxb294-uxc758uxc9c0uxac00-uxc544uxb2c8uxb77c-uxad6cuxc870uxb2e4}

치료가 필요한 사람은 많다.\\
그러나 치료를 \textbf{감당할 수 있는 구조}는 부족하다.

문제는:

\begin{itemize}
\tightlist
\item
  원하느냐가 아니라\\
\item
  필요하냐가 아니라\\
\item
  \textbf{지속 가능하냐}다
\end{itemize}

의료는 개인의 선택이 아니라\\
\textbf{자원의 배분 문제}다.

\begin{center}\rule{0.5\linewidth}{0.5pt}\end{center}

\section{3. 필수의료가 적자인
이유}\label{uxd544uxc218uxc758uxb8ccuxac00-uxc801uxc790uxc778-uxc774uxc720}

필수의료가 흔들리는 이유는\\
의료진의 능력 부족이 아니다.

구조적 이유가 있다.

\subsection{1) 환자군이 사회적으로 더
취약하다}\label{uxd658uxc790uxad70uxc774-uxc0acuxd68cuxc801uxc73cuxb85c-uxb354-uxcde8uxc57duxd558uxb2e4}

중증 외상, 산업재해, 응급질환은\\
열악한 환경에서 발생할 확률이 높다.

그러나 이 환자군은\\
높은 비용을 감당하기 어렵다.

\subsection{2) 고위험·고강도 노동 대비 보상이
낮다}\label{uxace0uxc704uxd5d8uxace0uxac15uxb3c4-uxb178uxb3d9-uxb300uxbe44-uxbcf4uxc0c1uxc774-uxb0aeuxb2e4}

응급실, 중환자실, 외상센터는

\begin{itemize}
\tightlist
\item
  24시간 운영\\
\item
  높은 법적 리스크\\
\item
  높은 인력 소모\\
\item
  복잡한 의사결정
\end{itemize}

을 요구하지만,\\
보상 구조는 이를 충분히 반영하지 못한다.

\textbf{환자 수는 많지만\\
구조적으로 손해를 보는 시스템}이다.

\subsection{3) 일부 필수과는 환자 수 자체가
적다}\label{uxc77cuxbd80-uxd544uxc218uxacfcuxb294-uxd658uxc790-uxc218-uxc790uxccb4uxac00-uxc801uxb2e4}

소아과, 희귀질환, 특정 전문 수술 분야는\\
``꼭 필요하지만'' 환자 수가 많지 않다.

고정비는 크고\\
규모의 경제는 어렵다.

\begin{center}\rule{0.5\linewidth}{0.5pt}\end{center}

\section{4. 성소수자의 경제적
현실}\label{uxc131uxc18cuxc218uxc790uxc758-uxacbduxc81cuxc801-uxd604uxc2e4}

성소수자는\\
모든 계층에 ``고르게'' 분포하기보다\\
사회적 배제의 영향을 더 크게 받는다.

\begin{itemize}
\tightlist
\item
  가족 단절\\
\item
  고용 차별\\
\item
  사회적 낙인\\
\item
  법적 보호의 공백
\end{itemize}

이 구조는\\
성소수자를 경제적으로 더 취약한 위치로 밀어낸다.

그래서 성전환 수술은\\
``선택''의 문제가 아니라\\
\textbf{접근성}의 문제가 된다.

\begin{center}\rule{0.5\linewidth}{0.5pt}\end{center}

\section{5. 공통 구조: 필요하지만 감당하기
어렵다}\label{uxacf5uxd1b5-uxad6cuxc870-uxd544uxc694uxd558uxc9c0uxb9cc-uxac10uxb2f9uxd558uxae30-uxc5b4uxb835uxb2e4}

필수의료와 성전환 수술은\\
의학적으로 다르다.

\begin{itemize}
\tightlist
\item
  하나는 생명을 구한다\\
\item
  하나는 삶의 정체성을 지킨다
\end{itemize}

그러나 현실 구조는 닮아 있다.

\begin{quote}
\textbf{필요하지만,\\
감당하기 어려운 비용이 따른다.}
\end{quote}

그래서 어떤 사람들은:

\begin{itemize}
\tightlist
\item
  빚을 지고\\
\item
  위험한 노동을 감수하고\\
\item
  삶을 미루고\\
\item
  몸과 시간을 저당 잡힌다
\end{itemize}

문제는\\
``왜 그런 선택을 했느냐''가 아니라\\
\textbf{왜 다른 선택지가 없었느냐}다.

\textbf{성소수자의 수술과 외상센터의 수술은\\
좌표계만 다를 뿐,\\
기존 시스템이 수용하지 못하는 고통이라는 점에서는 같은 구조다.}

\begin{center}\rule{0.5\linewidth}{0.5pt}\end{center}

\section{6. Doctor K의 선택}\label{doctor-kuxc758-uxc120uxd0dd}

\begin{quote}
\emph{``나는 병원에 소속되지 않는다. 나는 환자에게 소속된다.''} ---
\textbf{Doctor K}
\end{quote}

Doctor K는 시스템을 떠났다.

그러나 의술을 포기하지 않았다.

그에게 의료는 직업이 아니라 \textbf{예술}이었기 때문이다.

그는 말하지 않는다.

\begin{quote}
``시스템이 문제다.''
\end{quote}

대신 이렇게 행동한다.

\begin{quote}
\textbf{``그래도 한다.''}
\end{quote}

\begin{center}\rule{0.5\linewidth}{0.5pt}\end{center}

\section{7. 관객이 필요한
이유}\label{uxad00uxac1duxc774-uxd544uxc694uxd55c-uxc774uxc720}

예술에는 관객이 있다.\\
과학에는 독자가 있다.\\
정치에는 지지자가 있다.

그러나 의료에는\\
오직 \textbf{환자}만 있다.

그렇기에\\
의료는 늘 고립된다.

AngraMyNew는 묻는다.

\begin{quote}
\textbf{치료에도 관객이 필요하지 않은가?}
\end{quote}

관객은:

\begin{itemize}
\tightlist
\item
  평가하지 않는다\\
\item
  통제하지 않는다\\
\item
  명령하지 않는다
\end{itemize}

관객은\\
\textbf{함께 본다.}

\begin{center}\rule{0.5\linewidth}{0.5pt}\end{center}

\section{8. 연대는 동정이 아니라
공명이다}\label{uxc5f0uxb300uxb294-uxb3d9uxc815uxc774-uxc544uxb2c8uxb77c-uxacf5uxba85uxc774uxb2e4}

연대는 불쌍해서 일어나지 않는다.\\
연대는 \textbf{서사}에서 발생한다.

사람은:

\begin{itemize}
\tightlist
\item
  숫자보다 이야기로 움직이고\\
\item
  통계보다 얼굴에 반응하며\\
\item
  제도보다 장면에 공명한다
\end{itemize}

연대는 기부가 아니라 \textbf{공명의 증표}다.

\begin{center}\rule{0.5\linewidth}{0.5pt}\end{center}

\section{9. 왜 실사가
아닌가}\label{uxc65c-uxc2e4uxc0acuxac00-uxc544uxb2ccuxac00}

실제 수술 장면은:

\begin{itemize}
\tightlist
\item
  의료 윤리에 민감하고\\
\item
  법적으로 제한되며\\
\item
  고통을 자극으로 소비할 위험이 있다
\end{itemize}

그래서 AngraMyNew는\\
\textbf{재구성된 형식}만을 사용한다.

\begin{itemize}
\tightlist
\item
  의학 기반 애니메이션\\
\item
  상징적 영상\\
\item
  서사 중심 편집\\
\item
  감정과 선택의 기록
\end{itemize}

우리는 \textbf{피}가 아니라\\
\textbf{결정의 구조}를 보여준다.

\textbf{우리는 고통을 전시하지 않는다.\\
우리는 고통이 질서로 변환되는 구조를 보여준다.}

\begin{center}\rule{0.5\linewidth}{0.5pt}\end{center}

\section{10. 의료는 시스템이 아니라
사람이다}\label{uxc758uxb8ccuxb294-uxc2dcuxc2a4uxd15cuxc774-uxc544uxb2c8uxb77c-uxc0acuxb78cuxc774uxb2e4}

국가는 계산한다.\\
보험은 분류한다.\\
제도는 통제한다.

그러나 치료는\\
\textbf{사람}이 한다.

그리고 사람은\\
혼자 버티지 않는다.

\begin{center}\rule{0.5\linewidth}{0.5pt}\end{center}

\section{11. 결론: 관객은 의료의 마지막
자원이다}\label{uxacb0uxb860-uxad00uxac1duxc740-uxc758uxb8ccuxc758-uxb9c8uxc9c0uxb9c9-uxc790uxc6d0uxc774uxb2e4}

필수의료는 무너지고 있다.\\
의사는 지치고 있다.\\
환자는 감당하고 있다.

AngraMyNew는\\
한 가지 질문만 남긴다.

\begin{quote}
\textbf{의료에도 관객이 필요한 시대가\\
온 것은 아닐까?}
\end{quote}

관객은\\
통제하지 않는다.\\
관객은\\
함께 본다.

그리고\\
함께 보는 순간,\\
연대는 발생한다.

\textbf{목격료는 기부가 아니다. 그것은 경계를 목격한 대가다.}

\begin{center}\rule{0.5\linewidth}{0.5pt}\end{center}

\begin{quote}
``의사는 시스템을 떠날 수 있다.\\
그러나 치료는 사람을 떠나지 않는다.''
\end{quote}

\chapter{025 --- 면세인의 소비: 조공(Tribute)하지 않는
삶}\label{uxba74uxc138uxc778uxc758-uxc18cuxbe44-uxc870uxacf5tributeuxd558uxc9c0-uxc54auxb294-uxc0b6}

\section{--- 기능은 헐값에 사고, 취향은
독점한다}\label{uxae30uxb2a5uxc740-uxd5d0uxac12uxc5d0-uxc0acuxace0-uxcde8uxd5a5uxc740-uxb3c5uxc810uxd55cuxb2e4}

세상은 두 가지를 판다. 하나는 \textbf{물성(Matter)} 이고, 하나는
\textbf{환상(Myth)} 이다.

자동차는 이동하는 기계(물성)이자, 계급의 증명서(환상)다. 호텔은 잠자는
방(물성)이자, 대접받는 느낌(환상)이다.

시스템은 이 '환상'에 막대한 가격표를 붙인다. 이것을 \textbf{브랜드 가치}
라 부르지만, AngraMyNew는 그것을 \textbf{시스템세(System Tax)} 라
부른다. 조공은 이 시스템세의 일상적 형태다.

부자는 이 세금을 성실히 납부하여 시스템의 VIP가 된다. 그러나 면세인은 이
세금 납부를 거부한다.

\begin{center}\rule{0.5\linewidth}{0.5pt}\end{center}

\section{1. 동의하지 않는 세계관에는 '물성'의 비용만
지불한다}\label{uxb3d9uxc758uxd558uxc9c0-uxc54auxb294-uxc138uxacc4uxad00uxc5d0uxb294-uxbb3cuxc131uxc758-uxbe44uxc6a9uxb9cc-uxc9c0uxbd88uxd55cuxb2e4}

면세인의 첫 번째 행위는 남을 끊는 게 아니라, \textbf{내 안의 허영을 먼저
베어내는 것이다.}

면세인은 돈이 없는 게 아니다. \textbf{남이 만든 계급 놀이에 입장료를
내기 싫을 뿐이다.}

그들이 만든 세계관(명품 로고, 하차감, 5성급의 허세)이 내 미감과
무관하다면, 면세인은 철저하게 \textbf{기능(Function)} 만 발라내어
구입한다.

\begin{itemize}
\tightlist
\item
  이동이 필요하면 가장 튼튼하고 연비 좋은 차를 산다.
\item
  잠이 필요하면 가장 조용하고 깨끗한 숙소를 잡는다.
\item
  옷이 필요하면 소재가 가장 좋은 것을 입는다.
\end{itemize}

이것은 절약이 아니다. \textbf{내 취향이 아닌 환상에 대한 '조공 거부'다.}

\begin{quote}
\textbf{``나는 당신들의 신을 믿지 않으므로,\textbf{ }당신들의
신전(Department Store)에 십일조를 내지 않겠다.''}
\end{quote}

\begin{center}\rule{0.5\linewidth}{0.5pt}\end{center}

\section{2. 맘에 드는 세계관에는 '전부'를
태운다}\label{uxb9d8uxc5d0-uxb4dcuxb294-uxc138uxacc4uxad00uxc5d0uxb294-uxc804uxbd80uxb97c-uxd0dcuxc6b4uxb2e4}

아낀 세금은 어디로 가는가? 통장에 쌓이지 않는다.

내가 매혹된 세계, 내가 지지하는 세계, 내가 닮고 싶은 세계로
흘러들어간다.

면세인은 남들이 이해 못 하는 낡은 고서 한 권에 수백만 원을 쓴다. 단
하나의 영감을 위해 지구 반대편으로 날아간다. 자신의 심장을 뛰게 하는
세계관을 구현한 제품이라면, 기능적으로는 무의미해 보여도 기꺼이 전
재산을 붓는다.

이때의 소비는 소비가 아니다. \textbf{나의 신에 대한 제의(Ritual)이자, 그
세계에 대한 투표다.}

\begin{center}\rule{0.5\linewidth}{0.5pt}\end{center}

\section{3. 부자와 면세인의 소비 행동
차이}\label{uxbd80uxc790uxc640-uxba74uxc138uxc778uxc758-uxc18cuxbe44-uxd589uxb3d9-uxcc28uxc774}

\begin{longtable}[]{@{}
  >{\raggedright\arraybackslash}p{(\linewidth - 4\tabcolsep) * \real{0.3333}}
  >{\raggedright\arraybackslash}p{(\linewidth - 4\tabcolsep) * \real{0.3333}}
  >{\raggedright\arraybackslash}p{(\linewidth - 4\tabcolsep) * \real{0.3333}}@{}}
\toprule\noalign{}
\begin{minipage}[b]{\linewidth}\raggedright
구분
\end{minipage} & \begin{minipage}[b]{\linewidth}\raggedright
부자 (The Rich)
\end{minipage} & \begin{minipage}[b]{\linewidth}\raggedright
면세인 (The Tax-Exempt)
\end{minipage} \\
\midrule\noalign{}
\endhead
\bottomrule\noalign{}
\endlastfoot
\textbf{소비 기준} & 남들이 알아주는가? (과시) & \textbf{내 맘에 드는가?
(공명)} \\
\textbf{자동차} & 내 사회적 지위를 대변한다 & 계급장이면 거부,
기계미(機械美)라면 집착 \\
\textbf{지출 구조} & 넓고 얕게 뿌린다 (품위 유지비) & \textbf{좁고 깊게
찌른다 (취향 구축비)} \\
\end{longtable}

부자는 시스템이 정해준 가격표대로 산다. 면세인은 가치를 스스로 책정한다.

그래서 면세인은 겉보기에 모순적이다. 경차를 타고 다니면서, 트렁크에는
1억짜리 그림이나 서버 장비가 실려 있다.

\begin{center}\rule{0.5\linewidth}{0.5pt}\end{center}

\section{4. 조공을 멈춰야 안목이
생긴다}\label{uxc870uxacf5uxc744-uxba48uxcdb0uxc57c-uxc548uxbaa9uxc774-uxc0dduxae34uxb2e4}

대부분의 사람들은 평생 남의 세계관에 월세를 내다가 생을 마감한다.

샤넬이 만든 세계관에 월세를 내고, 포르쉐가 만든 세계관에 월세를 내고,
아파트 브랜드가 만든 세계관에 월세를 낸다.

그 돈을 끊어야 한다. 기능만 남기고 껍데기를 거부해야 한다.

\textbf{남의 기준으로 쓰던 돈을 멈추면, 비로소 '내 기준'을 세울 여백이
생긴다. 여백이 있어야 안목이 자란다.}

그렇게 확보한 잉여 자원으로 \textbf{네가 진짜 사랑하는 세계관(My New)}
을 사야 한다.

AngraMyNew는 이를 원칙으로 삼는다.

\begin{quote}
\textbf{기능은 최저가로 매수하고,} \textbf{취향은 최고가로 매수하라.}

\textbf{단, 그 취향은 오직 네가 선택한 것이어야 한다.}
\end{quote}

그리고 언젠가, 네가 만든 세계관에 누군가 입장료를 낼 것이다.

\begin{center}\rule{0.5\linewidth}{0.5pt}\end{center}

\section{부록: 구매 전
3문장}\label{uxbd80uxb85d-uxad6cuxb9e4-uxc804-3uxbb38uxc7a5}

\begin{enumerate}
\def\labelenumi{\arabic{enumi}.}
\tightlist
\item
  이건 기능인가, 환상인가?
\item
  이 환상이 내 것인가, 남의 것인가?
\item
  이 비용으로 내 세계관에 무엇을 구축할 수 있는가?
\end{enumerate}

\chapter{026 --- 진·선·미의 삼국지: 우리는 승리가 아니라 전설을
원한다}\label{uxc9c4uxc120uxbbf8uxc758-uxc0bcuxad6duxc9c0-uxc6b0uxb9acuxb294-uxc2b9uxb9acuxac00-uxc544uxb2c8uxb77c-uxc804uxc124uxc744-uxc6d0uxd55cuxb2e4}

\section{--- 지속가능성(Sustainability)에 대한
거부}\label{uxc9c0uxc18duxac00uxb2a5uxc131sustainabilityuxc5d0-uxb300uxd55c-uxac70uxbd80}

미래는 다시 쪼개질 것이다. 단일한 시스템은 끝났다. 세상은
\textbf{`진(眞) · 선(善) · 미(美)'} 의 삼국지로 재편될 것이다.

\begin{center}\rule{0.5\linewidth}{0.5pt}\end{center}

\section{1. 위나라 (曹魏): 테크노 봉건제 {[}진 /
眞{]}}\label{uxc704uxb098uxb77c-ux66f9ux9b4f-uxd14cuxd06cuxb178-uxbd09uxac74uxc81c-uxc9c4-ux771e}

\begin{itemize}
\tightlist
\item
  \textbf{군주:} 일론 머스크, 피터 틸, 샘 알트만.
\item
  \textbf{이념:} 효율, 가속, 기술적 특이점.
\item
  \textbf{메시지:} ``능력 없는 자는 지배받아라. 대신 화성에
  보내주겠다.''
\item
  \textbf{특징:} 가장 강력하다. 압도적인 무력(AI/자본)을 가졌다. 하지만
  차갑다. 그곳에 인간은 없고 '데이터'만 있다.
\end{itemize}

\section{2. 오나라 (東吳): 낡은 관료주의 {[}선 /
善{]}}\label{uxc624uxb098uxb77c-ux6771ux5433-uxb0a1uxc740-uxad00uxb8ccuxc8fcuxc758-uxc120-ux5584}

\begin{itemize}
\tightlist
\item
  \textbf{군주:} EU, UN, 기존 국가의 정치인들.
\item
  \textbf{이념:} 도덕, 규제, 인권, PC(Political Correctness).
\item
  \textbf{메시지:} ``우리는 올바르다. 약자를 보호해야 한다.''
\item
  \textbf{특징:} 강남(기득권)의 방어선을 지킨다. 하지만 낡았다. 혁신은
  없고 규제라는 방패만 남았다. 서서히 늙어 죽어갈 것이다.
\end{itemize}

\section{3. 촉나라 (蜀漢): 미적 군벌의 연대 {[}미 /
美{]}}\label{uxcd09uxb098uxb77c-ux8700ux6f22-uxbbf8uxc801-uxad70uxbc8cuxc758-uxc5f0uxb300-uxbbf8-ux7f8e}

\begin{itemize}
\tightlist
\item
  \textbf{정체:} \textbf{Confederacy of Aesthetic Warlords}
\item
  \textbf{깃발:} AngraMyNew --- 군주가 아니라 선언문(Manifesto)이다.
\item
  \textbf{장수:} 각자의 영토를 가진 독립 아티스트들. 누구도 누구에게
  복종하지 않는다. 칼을 섞는 이유는 명령이 아니라 공명(共鳴)이다.
\item
  \textbf{이념:} 서사, 결핍, 압도적인 아름다움.
\item
  \textbf{메시지:} ``우리는 위나라의 부품이 되기도, 오나라의 시민이
  되기도 거부한다. 우리는 각자가 하나의 독립된 \textbf{군벌(Warlord)}
  이다. 우리는 기술도(진), 도덕도(선) 없다. 오직 '아름다움'이라는 깃발
  아래서만 잠시 칼을 섞을 뿐이다.''
\end{itemize}

\begin{center}\rule{0.5\linewidth}{0.5pt}\end{center}

\section{4. 우리의 한계는
구조적이다}\label{uxc6b0uxb9acuxc758-uxd55cuxacc4uxb294-uxad6cuxc870uxc801uxc774uxb2e4}

우리는 안다. 아름다움만으로는 천하를 통일할 수 없다는 것을. 우리의
한계는 딱 \textbf{`천하삼분(天下三分)'} 까지다.

그리고 언젠가 \textbf{`이릉대전'} 이 올 것이다.

이것은 예언이 아니라 진단이다. 미(美)를 운영체제로 삼은 조직은
구조적으로 이릉대전을 피할 수 없다. 진(眞)의 조직은 손익분기점으로
판단하고, 선(善)의 조직은 도덕적 명분으로 판단한다. 우리에게는 그런
브레이크가 없다.

우리의 브레이크는 \textbf{`이것이 아름다운가?'} 뿐이다.

그 질문이 '예'라고 답하는 순간, 우리는 계산 없이 불속으로 걸어 들어갈
것이다. 이것이 미적 조직의 설계상 결함(design flaw)이다. 우리는 그
결함을 제거하지 않는다. 그것이 우리의 정체성이기 때문이다.

하지만 기억하라. 역사는 위나라가 통일했지만, \textbf{사람들은 2천 년이
지난 지금도 촉나라를 그리워한다.}

우리는 승리하기 위해서가 아니라, \textbf{`전설(Legend)'} 이 되기 위해
싸운다.

\begin{center}\rule{0.5\linewidth}{0.5pt}\end{center}

\section{{[}Addendum{]} 지속가능성에
대하여}\label{addendum-uxc9c0uxc18duxac00uxb2a5uxc131uxc5d0-uxb300uxd558uxc5ec}

\textbf{Q. 지속가능(Sustainability)해야 하지 않습니까? 균형을 잡아야
하지 않습니까?}

이것은 합리적인 질문이다. 우리는 이 질문을 조롱하지 않는다. 다만
거부한다.

\begin{quote}
아름다움은 지속되지 않는다. \textbf{폭발} 한다. 그리고 그 폭발의 잔상이
사람들의 뇌리에 각인된다.

벚꽃은 지속가능해서 아름다운 것이 아니다. 지기 때문에 아름다운 것이다.

우리는 가늘고 길게 시스템에 기생하려는 게 아니다. 짧고 굵게 타오르고,
\textbf{신화(Myth)} 로 남으려는 것이다.

지속가능성은 시스템의 미덕이다. 우리는 시스템이 아니라
\textbf{사건(Event)} 이다.

사건은 오래가지 않는다. 대신 되돌릴 수 없다.

--- AngraMyNew, 제26장 \emph{진선미의 삼국지}
\end{quote}

\newpage

\chapter*{Part II: Art}\label{part-ii-art}
\addcontentsline{toc}{chapter}{Part II: Art}

\chapter{갈루아와
5차방정식}\label{uxac08uxb8e8uxc544uxc640-5uxcc28uxbc29uxc815uxc2dd}

\section{--- 풀 수 없음을 증명하는
아름다움}\label{uxd480-uxc218-uxc5c6uxc74cuxc744-uxc99duxba85uxd558uxb294-uxc544uxb984uxb2e4uxc6c0}

\begin{center}\rule{0.5\linewidth}{0.5pt}\end{center}

\section{문제}\label{uxbb38uxc81c}

2차방정식에는 근의 공식이 있다. 3차, 4차도 있다.

5차는?

300년간 수학자들이 공식을 찾았다. 아벨은 ``없다''고 증명했다.

그러나 \textbf{왜} 없는지는 설명하지 못했다.

\begin{center}\rule{0.5\linewidth}{0.5pt}\end{center}

\section{파괴}\label{uxd30cuxad34}

에바리스트 갈루아는 20세에 죽었다. 결투 전날 밤, 그는 편지를 썼다.

그 편지에는 수학이 아니라 \textbf{수학을 보는 새로운 방식}이 있었다.

갈루아는 방정식을 풀려 하지 않았다. 대신 방정식의 \textbf{대칭 구조}를
봤다.

\begin{quote}
기존 질문: ``근이 무엇인가?'' 갈루아의 질문: ``근들 사이의 관계가
무엇인가?''
\end{quote}

\begin{center}\rule{0.5\linewidth}{0.5pt}\end{center}

\section{재구성}\label{uxc7acuxad6cuxc131}

갈루아는 '군(Group)'이라는 구조를 발명했다.

방정식의 근들이 어떻게 서로 치환될 수 있는지, 그 치환들이 어떤 구조를
이루는지를 봤다.

5차방정식의 군은 \textbf{단순군}이다. 더 이상 쪼갤 수 없다.

근의 공식은 군을 단계적으로 쪼개는 과정이다. 쪼갤 수 없으면, 공식도
없다.

\begin{quote}
\textbf{풀 수 없음이 구조적 필연이 되었다.}
\end{quote}

\begin{center}\rule{0.5\linewidth}{0.5pt}\end{center}

\section{확장}\label{uxd655uxc7a5}

갈루아 이론은 방정식을 넘어섰다.

\begin{itemize}
\tightlist
\item
  대수학 전체의 기초가 되었다
\item
  암호학의 뼈대가 되었다
\item
  물리학의 대칭성 이론으로 확장되었다
\end{itemize}

20세 청년의 편지 한 장이 수학의 언어 자체를 바꿨다.

\begin{center}\rule{0.5\linewidth}{0.5pt}\end{center}

\section{AngraMyNew가 보는
아름다움}\label{angramynewuxac00-uxbcf4uxb294-uxc544uxb984uxb2e4uxc6c0}

갈루아의 증명이 아름다운 이유:

\begin{longtable}[]{@{}ll@{}}
\toprule\noalign{}
요소 & 설명 \\
\midrule\noalign{}
\endhead
\bottomrule\noalign{}
\endlastfoot
파괴 & ``공식을 찾는다''는 300년 패러다임을 버림 \\
재구성 & 방정식을 군 구조로 번역 \\
확장 & 하나의 문제가 수학 전체를 재편 \\
\end{longtable}

그는 답을 구하지 않았다. \textbf{답이 없는 이유}를 구조로 보여줬다.

\begin{quote}
\textbf{``풀 수 없다''는 것을 아름답게 증명할 수 있다. 그것이 갈루아가
남긴 것이다.}
\end{quote}

\chapter{일반상대성이론}\label{uxc77cuxbc18uxc0c1uxb300uxc131uxc774uxb860}

\section{--- 중력을 지운
아름다움}\label{uxc911uxb825uxc744-uxc9c0uxc6b4-uxc544uxb984uxb2e4uxc6c0}

\begin{center}\rule{0.5\linewidth}{0.5pt}\end{center}

\section{문제}\label{uxbb38uxc81c-1}

뉴턴의 중력은 강력했다. 행성의 궤도를 예측하고, 조수를 설명했다.

그러나 하나의 질문이 남았다:

\begin{quote}
\textbf{중력은 어떻게 빈 공간을 건너가는가?}
\end{quote}

뉴턴은 답하지 않았다. ``나는 가설을 만들지 않는다(Hypotheses non
fingo).''

\begin{center}\rule{0.5\linewidth}{0.5pt}\end{center}

\section{파괴}\label{uxd30cuxad34-1}

아인슈타인은 중력을 \textbf{지웠다}.

자유낙하하는 엘리베이터 안에서 사람은 무중력을 느낀다.

중력이 사라진 게 아니다. \textbf{중력과 가속도가 구별되지 않는다}는
것이다.

\begin{quote}
등가원리: 중력장 안에 있는 것과 가속하는 것은 구별할 수 없다.
\end{quote}

이 순간, 중력은 ``힘''이 아니게 되었다.

\begin{center}\rule{0.5\linewidth}{0.5pt}\end{center}

\section{재구성}\label{uxc7acuxad6cuxc131-1}

중력이 힘이 아니라면, 무엇인가?

아인슈타인의 답: \textbf{시공간의 곡률}.

질량은 시공간을 휘게 한다. 물체는 휘어진 시공간에서 가장 직선적인 경로를
간다. 그것이 우리 눈에 ``떨어지는 것''으로 보인다.

\[G_{\mu\nu} = 8\pi T_{\mu\nu}\]

왼쪽은 시공간의 곡률. 오른쪽은 물질과 에너지의 분포.

\begin{quote}
\textbf{물질이 시공간에게 어떻게 휘어야 하는지 말하고, 시공간이 물질에게
어떻게 움직여야 하는지 말한다.}
\end{quote}

\begin{center}\rule{0.5\linewidth}{0.5pt}\end{center}

\section{확장}\label{uxd655uxc7a5-1}

일반상대성은 중력을 넘어섰다.

\begin{itemize}
\tightlist
\item
  블랙홀의 존재를 예측했다
\item
  중력파를 예측했다 (100년 후 검출)
\item
  우주의 팽창을 설명했다
\item
  GPS 위성의 시간 보정에 쓰인다
\end{itemize}

하나의 원리가 우주 전체의 구조가 되었다.

\begin{center}\rule{0.5\linewidth}{0.5pt}\end{center}

\section{AngraMyNew가 보는
아름다움}\label{angramynewuxac00-uxbcf4uxb294-uxc544uxb984uxb2e4uxc6c0-1}

일반상대성이 아름다운 이유:

\begin{longtable}[]{@{}ll@{}}
\toprule\noalign{}
요소 & 설명 \\
\midrule\noalign{}
\endhead
\bottomrule\noalign{}
\endlastfoot
파괴 & ``중력은 힘이다''라는 뉴턴 패러다임을 버림 \\
재구성 & 중력을 기하학으로 번역 \\
확장 & 하나의 방정식이 우주 전체를 기술 \\
\end{longtable}

아인슈타인은 중력을 설명하지 않았다. \textbf{중력이라는 개념 자체를
제거했다.}

남은 것은 순수한 기하학뿐이다.

\begin{quote}
\textbf{가장 아름다운 이론은 설명해야 할 것 자체를 없애버린다.}
\end{quote}

\begin{center}\rule{0.5\linewidth}{0.5pt}\end{center}

\section{관련 문서}\label{uxad00uxb828-uxbb38uxc11c-4}

→ \url{004_principia_geometry.md} --- 뉴턴: 중력은 그려졌다 →
\url{../ideas/019_causality_question.md} --- 인과관계에 대한 의문

\chapter{하나의 무늬가 전부가
되다}\label{uxd558uxb098uxc758-uxbb34uxb2acuxac00-uxc804uxbd80uxac00-uxb418uxb2e4}

\begin{center}\rule{0.5\linewidth}{0.5pt}\end{center}

\section{공통점}\label{uxacf5uxd1b5uxc810}

Goyard는 Y자 슈브론. Bao Bao는 삼각형 메쉬. 유비는 인의(仁義).

셋 다 \textbf{단 하나의 패턴}으로 전부를 정의한다.

\begin{center}\rule{0.5\linewidth}{0.5pt}\end{center}

\section{Goyard: 170년을
하나로}\label{goyard-170uxb144uxc744-uxd558uxb098uxb85c}

1853년부터 변하지 않은 Y자 슈브론 패턴.

루이비통은 모노그램 외에도 다미에, 에피, 베르니 등 여러 라인을 만들었다.
Goyard는 Goyardine 하나다.

로고를 크게 박지 않는다. 광고를 하지 않는다. 오직 Y자 패턴 자체가
정체성이다.

\textbf{하나의 형태로 170년의 역사를 채웠다.}

\begin{center}\rule{0.5\linewidth}{0.5pt}\end{center}

\section{Bao Bao: 하나인데
무한하다}\label{bao-bao-uxd558uxb098uxc778uxb370-uxbb34uxd55cuxd558uxb2e4}

이세이 미야케의 Bao Bao. 삼각형 조각들이 메쉬 위에 붙어 있다.

패턴은 하나지만, 형태는 무한하다: - 가방을 비우면 평면 - 채우면 입체 -
내용물 모양이 가방 모양이 된다

\textbf{하나의 규칙이 무한한 변주를 만든다.}

\begin{center}\rule{0.5\linewidth}{0.5pt}\end{center}

\section{유비: 하나의 서사로 천하를
얻다}\label{uxc720uxbe44-uxd558uxb098uxc758-uxc11cuxc0acuxb85c-uxcc9cuxd558uxb97c-uxc5bbuxb2e4}

돗자리 짜던 사람이 황제가 됐다. 군사력도, 영토도, 재력도 없이.

유비의 패턴은 세 문장이지만, 결국 하나로 수렴한다: 1. 나는 한왕실의
후예다 (정통성) 2. 한왕실을 부흥하겠다 (목표) 3. 인의로 천하를
바로잡겠다 (방법)

\textbf{→ ``나는 정당한 자격으로, 옳은 방법으로, 세상을 바로잡는다.''}

조조는 실력으로 싸웠다. 손권은 지리로 싸웠다. 유비는 서사로 싸웠다.

관우와 장비가 목숨을 걸었다. 제갈량이 삼고초려에 응했다. 백성이 따라
피난길에 나섰다.

모두 이 서사에 매혹됐다.

\textbf{하나의 이야기가 촉한을 세웠다.}

\begin{center}\rule{0.5\linewidth}{0.5pt}\end{center}

\section{아인슈타인 타일과의
연결}\label{uxc544uxc778uxc288uxd0c0uxc778-uxd0c0uxc77cuxacfcuxc758-uxc5f0uxacb0}

아인슈타인 타일의 질문: \textgreater{} ``단 하나의 모양으로 무한한
평면을 채울 수 있는가?''

Goyard의 대답: \textgreater{} ``단 하나의 패턴으로 170년을 채울 수
있다.''

Bao Bao의 대답: \textgreater{} ``단 하나의 패턴으로 무한한 형태를 만들
수 있다.''

유비의 대답: \textgreater{} ``단 하나의 서사로 천하를 도모할 수 있다.''

\begin{center}\rule{0.5\linewidth}{0.5pt}\end{center}

\section{AngraMyNew 해석}\label{angramynew-uxd574uxc11d}

\begin{quote}
``복잡함이 아니라 밀도다.''
\end{quote}

많이 만드는 것이 창조가 아니다. 하나를 끝까지 밀고 가는 것이 창조다.

Goyard는 시간 축으로 밀었다. (170년) Bao Bao는 공간 축으로 밀었다. (무한
변형) 유비는 인간 축으로 밀었다. (인의의 로맨스)

\textbf{하나의 'My'를 완성하면, 그것이 세계가 된다.}

\chapter{004 --- 중력은
그려졌다}\label{uxc911uxb825uxc740-uxadf8uxb824uxc84cuxb2e4}

\section{--- 뉴턴 『프린키피아』의 기하학적
악상}\label{uxb274uxd134-uxd504uxb9b0uxd0a4uxd53cuxc544uxc758-uxae30uxd558uxd559uxc801-uxc545uxc0c1}

\begin{center}\rule{0.5\linewidth}{0.5pt}\end{center}

\section{통념}\label{uxd1b5uxb150}

뉴턴은 힘의 과학자다. 중력은 힘이고, 세계는 힘의 합으로 움직인다.

이 통념은 반쯤만 맞다.

『프린키피아』를 실제로 펼쳐보면, 뉴턴은 중력을 거의 \textbf{계산하지
않는다}. 그는 중력을 \textbf{그린다}.

\begin{center}\rule{0.5\linewidth}{0.5pt}\end{center}

\section{1. 뉴턴은 이미 미적분을 알고
있었다}\label{uxb274uxd134uxc740-uxc774uxbbf8-uxbbf8uxc801uxbd84uxc744-uxc54cuxace0-uxc788uxc5c8uxb2e4}

중요한 사실부터 짚자.

뉴턴은 『프린키피아』 집필 당시 이미 미적분을 발명한 상태였다. 계산
능력의 부족이 아니었다.

그럼에도 그는:

\begin{itemize}
\tightlist
\item
  미적분을 거의 쓰지 않고
\item
  원, 접선, 면적, 비례 관계로
\item
  운동을 설명했다
\end{itemize}

이 선택은 기술적 제약이 아니라 \textbf{표현에 대한 결정}이었다.

\begin{center}\rule{0.5\linewidth}{0.5pt}\end{center}

\section{2. 중력은 '힘'으로 설명되지
않는다}\label{uxc911uxb825uxc740-uxd798uxc73cuxb85c-uxc124uxba85uxb418uxc9c0-uxc54auxb294uxb2e4}

『프린키피아』에서 뉴턴은 묻지 않는다.

\begin{itemize}
\tightlist
\item
  왜 끌어당기는가
\item
  무엇이 작용하는가
\item
  힘의 본질은 무엇인가
\end{itemize}

대신 그는 이것을 보여준다.

\begin{itemize}
\tightlist
\item
  이런 궤적이 있다
\item
  이런 면적 법칙이 성립한다
\item
  그러면 이 운동은 필연적으로 따라온다
\end{itemize}

중력은 원인이 아니라 \textbf{형태가 만든 필연성}으로 등장한다.

\begin{center}\rule{0.5\linewidth}{0.5pt}\end{center}

\section{3. 기하학은 설명이 아니라
납득이다}\label{uxae30uxd558uxd559uxc740-uxc124uxba85uxc774-uxc544uxb2c8uxb77c-uxb0a9uxb4dduxc774uxb2e4}

뉴턴의 증명은 설득이 아니다.

\begin{itemize}
\tightlist
\item
  논리로 밀어붙이지 않고
\item
  언어로 정당화하지 않는다
\end{itemize}

대신 독자가 스스로 느끼게 만든다.

\begin{quote}
``이렇게 생긴 세계라면 이렇게 움직일 수밖에 없구나.''
\end{quote}

이것은 설명이 아니라 \textbf{형태에 의한 납득}이다.

\begin{center}\rule{0.5\linewidth}{0.5pt}\end{center}

\section{4. 여기서 드러나는
악상}\label{uxc5ecuxae30uxc11c-uxb4dcuxb7ecuxb098uxb294-uxc545uxc0c1}

이 지점에서 뉴턴은 힘의 과학자가 아니라 구조의 아티스트에 가깝다.

그의 출발점은 다음과 같다.

\begin{itemize}
\tightlist
\item
  중력은 무엇인가? ❌
\item
  세계는 어떻게 생겼는가? ⭕
\item
  이 형태에서 어떤 운동이 필연적인가? ⭕
\end{itemize}

이 질문은 논리보다 먼저 떠오른 감각, 즉 \textbf{악상}에 가깝다.

\begin{center}\rule{0.5\linewidth}{0.5pt}\end{center}

\section{5. 이후의 전복}\label{uxc774uxd6c4uxc758-uxc804uxbcf5}

300년 뒤, 아인슈타인은 이 악상을 끝까지 밀어붙인다.

\begin{itemize}
\tightlist
\item
  힘을 제거하고
\item
  시공간의 곡률로 번역한다
\end{itemize}

그러나 그 출발점은 이미 『프린키피아』 안에 있었다.

\begin{quote}
중력은 설명할 대상이 아니라 \textbf{형태로 제시될 수 있다.}
\end{quote}

\begin{center}\rule{0.5\linewidth}{0.5pt}\end{center}

\section{AngraMyNew 해석}\label{angramynew-uxd574uxc11d-1}

뉴턴의 위대함은 공식을 만든 데 있지 않다.

그는 한 시대의 세계를 \textbf{기하학이라는 미적 형식으로 고정했다.}

\begin{itemize}
\tightlist
\item
  원인보다 구조
\item
  설명보다 형태
\item
  계산보다 납득
\end{itemize}

이 선택이 이후 300년 과학의 방향을 만들었다.

\begin{center}\rule{0.5\linewidth}{0.5pt}\end{center}

\section{결론}\label{uxacb0uxb860-7}

중력은 처음부터 완전히 설명된 적이 없다.

그러나 한 번, \textbf{아름답게 그려진 적은 있다.}

『프린키피아』는 과학서이기 이전에, 하나의 구조적 예술 작품이다.

AngraMyNew는 이 순간을 아름다움의 사례로 기록한다.

\begin{center}\rule{0.5\linewidth}{0.5pt}\end{center}

\section{관련 문서}\label{uxad00uxb828-uxbb38uxc11c-5}

→ \url{002_general_relativity.md} --- 아인슈타인: 중력을 지운 아름다움 →
\url{../ideas/022_age_of_malice.md} --- 악상의 시대

\chapter{005 --- 한글의 두
상태}\label{uxd55cuxae00uxc758-uxb450-uxc0c1uxd0dc}

\section{--- 물질이 될 때와 투명해질
때}\label{uxbb3cuxc9c8uxc774-uxb420-uxb54cuxc640-uxd22cuxba85uxd574uxc9c8-uxb54c}

\begin{center}\rule{0.5\linewidth}{0.5pt}\end{center}

\section{멈춘 두 순간}\label{uxba48uxcd98-uxb450-uxc21cuxac04}

서정주를 읽다 멈췄다. 이문열을 읽다 멈췄다.

그러나 이유는 정반대였다.

\begin{center}\rule{0.5\linewidth}{0.5pt}\end{center}

\section{1. 한글이 남아버린 순간 ---
서정주}\label{uxd55cuxae00uxc774-uxb0a8uxc544uxbc84uxb9b0-uxc21cuxac04-uxc11cuxc815uxc8fc}

서정주를 읽으며 나는 의미에서 멈추지 않았다.

\textbf{글자에서 멈췄다.}

\begin{itemize}
\tightlist
\item
  음절이 사라지지 않는다
\item
  발음이 의미보다 먼저 남는다
\item
  읽고 나면 문장이 아니라 잔여가 남는다
\end{itemize}

\begin{quote}
한글이 이런 글자였나?
\end{quote}

\textbf{「귀촉도」에서:}

\begin{quote}
눈물 아롱아롱 피리 불고 가신 임의 밟으신 길은 진달래 꽃비 오는
서역(西域) 삼만리. 흰 옷깃 여며 여며 가옵신 임의 다시 오진 못하는
파촉(巴蜀) 삼만리.

신이나 삼아 줄 걸, 슬픈 사연의 올올이 아로새긴 육날 메투리. 은장도 푸른
날로 이냥 베어서 부질 없는 이 머리털 엮어 드릴 걸.

초롱에 불빛 지친 밤하늘 구비구비 은핫물 목이 젖은 새. 차마 아니 솟은
가락 눈이 감겨서 제 피에 취한 새가 귀촉도 운다. 그대 하늘 끝 호올로 가신
임아.
\end{quote}

``올올이 아로새긴 육날 메투리'' ``구비구비 은핫물 목이 젖은 새''

이때 언어는 무언가를 전달하지 않는다.

설명하지 않고, 설득하지 않고, 메시지를 남기지 않는다.

대신 \textbf{존재한다}.

한글은 이 순간 도구가 아니라 물질이 된다.

\begin{center}\rule{0.5\linewidth}{0.5pt}\end{center}

\section{2. 한글이 사라져버린 순간 ---
이문열}\label{uxd55cuxae00uxc774-uxc0acuxb77cuxc838uxbc84uxb9b0-uxc21cuxac04-uxc774uxbb38uxc5f4}

이문열을 읽을 때 나는 다른 이유로 멈췄다.

이번에는 너무 \textbf{잘} 읽혔다.

\begin{itemize}
\tightlist
\item
  문장이 걸리지 않는다
\item
  표현이 눈에 띄지 않는다
\item
  서사만 남고 언어는 사라진다
\end{itemize}

\begin{quote}
한글이 이렇게 쉽게 읽혔었나?
\end{quote}

\textbf{「이 황량한 역에서」에서:}

\begin{quote}
당신들은 누구와 사랑에 빠져든 적이 있는가? 당신들은 틀림없이 그
고귀함이나 감미로움, 헤어질 때의 고통과 슬픔이며 그 후의 공허함 따위를
미화하고 과장하려 들 테지만 기실 그 진상은 뜻밖에도 단순하고 명백하다.
그것은 당신이 이 여행 중에 눈길을 끄는 한 소녀와 만났다는 것이며, 결국은
부정확하기 마련인 관찰에 이어 당신이 던진 호의 섞인 눈길에 그녀가
답했다는 것이며, 무료를 함께 달래자는 당신의 용기를 다한 요청에 그녀가
다소곳이 응했다는 것이며, 그리하여 약간은 야릇한 열에 들뜬 당신들이
깜박깜박 자기를 잊어가며 주고받은, 분명 달콤하고 섬세하나 또한 그리
대단할 건 없는 몇 개 유형의 행위와 가끔씩은 정색해도 좋을 대화의 집합에
지나지 않는다.
\end{quote}

문장이 길고 복잡한데도 걸리지 않는다.

다 읽고 나면 ``사랑의 허무함''이라는 의미만 남고, 어떤 단어로 썼는지는
기억나지 않는다.

언어는 마찰을 만들지 않는다. 독해를 요구하지 않는다.

의미가 곧바로 흐르고, 문장은 기억되지 않는다.

이때 한글은 존재를 주장하지 않는다.

한글은 이 순간 \textbf{완전히 투명해진다.}

\begin{center}\rule{0.5\linewidth}{0.5pt}\end{center}

\section{3. 두 상태는 대비가
아니다}\label{uxb450-uxc0c1uxd0dcuxb294-uxb300uxbe44uxac00-uxc544uxb2c8uxb2e4}

이 두 경험은 우열 관계가 아니다.

\begin{itemize}
\tightlist
\item
  하나는 과잉이고
\item
  하나는 최소화다
\end{itemize}

그러나 둘 다 같은 지점에서 출발한다.

\textbf{의미 이전의 선택.}

\begin{itemize}
\tightlist
\item
  남길 것인가
\item
  지울 것인가
\end{itemize}

이 결정은 논리의 결과가 아니라 악상의 방향이다.

\begin{center}\rule{0.5\linewidth}{0.5pt}\end{center}

\section{4. 언어의 상태
변화}\label{uxc5b8uxc5b4uxc758-uxc0c1uxd0dc-uxbcc0uxd654}

이 두 순간을 통해 하나의 구조가 드러난다.

한글은 본질적으로 하나의 성질만 갖지 않는다.

\begin{itemize}
\tightlist
\item
  밀도를 극단까지 올릴 수 있고
\item
  투명도를 극단까지 낮출 수 있다
\end{itemize}

이것은 재능의 문제가 아니다. 기교의 문제도 아니다.

\textbf{어디까지 밀어붙일 것인가에 대한 선택}이다.

\begin{center}\rule{0.5\linewidth}{0.5pt}\end{center}

\section{AngraMyNew 해석}\label{angramynew-uxd574uxc11d-2}

서정주는 언어를 남기기로 선택했다.

이문열은 언어를 지우기로 선택했다.

둘 다 설명 이전의 판단에서 출발했다.

정돈 이전의 진동이 문장 단위에서 서로 다른 형태로 고정된 사례다.

\begin{center}\rule{0.5\linewidth}{0.5pt}\end{center}

\section{결론}\label{uxacb0uxb860-8}

한글은 언제나 의미를 담는 그릇이 아니다.

어떤 순간에는 무게를 얻고, 어떤 순간에는 완전히 사라진다.

서정주와 이문열은 그 두 극단이 실제로 가능하다는 것을 각기 다른 방식으로
보여줬다.

AngraMyNew는 이 두 상태를 \textbf{아름다움의 사례}로 기록한다.

\chapter{006 --- 보이지 않으면 이해한 것이
아니다}\label{uxbcf4uxc774uxc9c0-uxc54auxc73cuxba74-uxc774uxd574uxd55c-uxac83uxc774-uxc544uxb2c8uxb2e4}

\section{--- 파인만의 다이어그램과
경로적분}\label{uxd30cuxc778uxb9ccuxc758-uxb2e4uxc774uxc5b4uxadf8uxb7a8uxacfc-uxacbduxb85cuxc801uxbd84}

\begin{center}\rule{0.5\linewidth}{0.5pt}\end{center}

\section{문제}\label{uxbb38uxc81c-2}

1940년대, 양자전기역학(QED)은 벽에 부딪혀 있었다.

전자 하나, 광자 하나. 이 둘의 상호작용을 계산하려면 칠판을 가득 채운
적분을 며칠간 풀어야 했다.

물리학자들은 계산할 수 있었다. 그러나 \textbf{볼 수는 없었다}.

\begin{center}\rule{0.5\linewidth}{0.5pt}\end{center}

한편, 양자역학에도 같은 문제가 있었다.

슈뢰딩거 방정식은 강력했다. 파동함수의 시간 진화를 정확히 기술했다.

그러나 그것은 하나의 미분방정식이었다. 풀 수는 있지만, \textbf{의미를 볼
수는 없었다}.

\begin{quote}
입자는 어디에 있는가? 왜 그 확률인가? 방정식은 답하지만, 보여주지
않는다.
\end{quote}

\begin{center}\rule{0.5\linewidth}{0.5pt}\end{center}

\section{파괴}\label{uxd30cuxad34-2}

리처드 파인만은 두 번 파괴했다.

\textbf{첫 번째.} 수식을 버리고 그림을 그렸다.

\begin{itemize}
\tightlist
\item
  직선: 전자
\item
  물결선: 광자
\item
  점(vertex): 상호작용
\end{itemize}

사람들은 비웃었다. ``물리학을 만화로 만드느냐?''

\textbf{두 번째.} 슈뢰딩거 방정식을 버리고 경로를 열었다.

입자가 A에서 B로 간다. 파인만은 물었다:

\begin{quote}
``하나의 경로를 푸는 대신, \textbf{가능한 모든 경로를 동시에 걷게 하면
어떨까?}''
\end{quote}

직선으로 가는 경로. 곡선으로 도는 경로. 달을 거쳐 오는 경로. 모든 경로에
위상(phase)을 부여하고, 전부 더한다.

물리학자들은 당혹했다. ``무한개의 경로를 더하라고?''

\begin{center}\rule{0.5\linewidth}{0.5pt}\end{center}

\section{재구성}\label{uxc7acuxad6cuxc131-2}

두 파괴 모두, 같은 결과에 도달했다.

\textbf{다이어그램.} 낙서의 모든 선과 점이 복잡한 적분 항과
\textbf{정확히 1:1로 대응}했다.

\begin{itemize}
\tightlist
\item
  선 하나 = 전파인자(propagator)
\item
  점 하나 = 결합상수(coupling constant)
\end{itemize}

그림을 그리면, 적분식이 자동으로 튀어나왔다.

가장 단순한 예: 전자 둘이 광자 하나를 주고받는다.

\begin{verbatim}
e⁻ ───→───●───→─── e⁻
           │
           ~ γ
           │
e⁻ ───→───●───→─── e⁻
\end{verbatim}

이 낙서가 곧 수식이다:

\[\mathcal{M} = \bar{u}(p_3)\,(-ie\gamma^\mu)\,u(p_1) \;\cdot\; \frac{-ig_{\mu\nu}}{q^2} \;\cdot\; \bar{u}(p_4)\,(-ie\gamma^\nu)\,u(p_2)\]

\begin{longtable}[]{@{}ll@{}}
\toprule\noalign{}
다이어그램 & 수식 \\
\midrule\noalign{}
\endhead
\bottomrule\noalign{}
\endlastfoot
외부 전자선 (→) & 스피너 \(u, \bar{u}\) \\
꼭짓점 (●) & 결합상수 \(-ie\gamma^\mu\) \\
내부 광자선 (\textasciitilde) & 전파인자
\(\dfrac{-ig_{\mu\nu}}{q^2}\) \\
\end{longtable}

선을 읽으면, 식이 나온다. 점을 읽으면, 상수가 나온다. 그것이 전부다.

\textbf{경로적분.} 무한개의 경로를 더하면, 대부분은 서로
상쇄(destructive interference)된다. 살아남는 것은 \textbf{작용(action)이
최소인 경로} 하나.

그것이 뉴턴의 고전역학이었다.

\[\langle B | A \rangle = \int \mathcal{D}[x(t)] \; e^{i S[x]/\hbar}\]

왼쪽은 양자역학 --- A에서 B로 갈 확률진폭. 오른쪽은 모든 경로의 합.

\begin{quote}
\textbf{양자역학과 고전역학이 하나의 그림 안에서 만났다.}
\end{quote}

슈뢰딩거는 방정식을 풀었다. 파인만은 방정식을 \textbf{보여줬다}.

\begin{center}\rule{0.5\linewidth}{0.5pt}\end{center}

\section{확장}\label{uxd655uxc7a5-2}

파인만 다이어그램은 QED를 넘어섰다.

\begin{itemize}
\tightlist
\item
  약력, 강력까지 확장 --- 표준모형 전체의 공용어가 되었다
\item
  입자물리학자의 칠판에서 수식이 사라지고, 그림이 남았다
\item
  응집물질, 끈이론까지 --- 물리학의 거의 모든 분야가 이 언어를 쓴다
\end{itemize}

경로적분은 양자역학을 넘어섰다.

\begin{itemize}
\tightlist
\item
  양자장론의 기초가 되었다
\item
  통계역학과 양자역학을 연결했다
\item
  양자중력 이론의 출발점이 되었다
\end{itemize}

한 남자의 그림과 경로가 물리학의 언어 자체를 바꿨다.

\begin{center}\rule{0.5\linewidth}{0.5pt}\end{center}

\section{AngraMyNew 해석}\label{angramynew-uxd574uxc11d-3}

파인만은 두 번 같은 일을 했다.

\begin{longtable}[]{@{}lll@{}}
\toprule\noalign{}
대상 & 기존 & 파인만 \\
\midrule\noalign{}
\endhead
\bottomrule\noalign{}
\endlastfoot
QED 계산 & 칠판 가득한 적분 & 선 몇 개의 다이어그램 \\
양자역학 & 하나의 미분방정식 & 모든 경로의 합 \\
\end{longtable}

두 경우 모두, \textbf{대수(algebra)를 기하(geometry)로 번역했다}.

이것은 악상이다.

논리보다 먼저 떠오른 감각. ``이건 보여야 한다''는 정돈 이전의 진동.

파인만은 계산하지 않았다. \textbf{보이게 만들었다.}

보이게 만들자, 본질이 드러났다.

\begin{center}\rule{0.5\linewidth}{0.5pt}\end{center}

\section{결론}\label{uxacb0uxb860-9}

물리학은 두 번, 같은 남자에게 같은 선물을 받았다.

적분은 그림이 되었고, 방정식은 경로가 되었다.

둘 다 같은 악상에서 태어났다.

\begin{quote}
\textbf{``보이지 않으면 이해한 것이 아니다.''}
\end{quote}

AngraMyNew는 이 두 순간을 \textbf{아름다움의 사례}로 기록한다.

\begin{center}\rule{0.5\linewidth}{0.5pt}\end{center}

\section{관련 문서}\label{uxad00uxb828-uxbb38uxc11c-6}

→ \url{004_principia_geometry.md} --- 뉴턴: 중력을 기하학으로 그리다 →
\url{002_general_relativity.md} --- 아인슈타인: 중력을 지운 아름다움 →
\url{../ideas/022_age_of_malice.md} --- 악상의 시대

\newpage

\chapter*{Part III: Protocols}\label{part-iii-protocols}
\addcontentsline{toc}{chapter}{Part III: Protocols}

\chapter{창조자 프로토콜 (Creator
Protocol)}\label{uxcc3duxc870uxc790-uxd504uxb85cuxd1a0uxcf5c-creator-protocol}

\emph{AngraMyNew Protocol Series / v1.0}

``프로토콜은 창조로 가는 다리이며, 다리는 건너면 사라진다.''

\begin{center}\rule{0.5\linewidth}{0.5pt}\end{center}

\section{0. 목적 (Purpose)}\label{uxbaa9uxc801-purpose}

이 문서는 창조자가 \textbf{자신의 내면, 세계 인식, 창작 충동을 정렬하기
위해 사용하는 임시 장치(Protocol) }이다.

창조란 본래 무규범적이며, 절대 자유의 작용이다.\\
그러나 초기 단계에서는 \textbf{혼돈을 질서로 변환하는 의식적 구조}가
필요하다.\\
본 프로토콜은 그 \textbf{초기 구조화 장치}로 기능한다.

\subsection{주의}\label{uxc8fcuxc758}

이 프로토콜은 모든 창조자를 위한 유일한 경로가 아니다.

\begin{itemize}
\tightlist
\item
  혐오가 아니라 \textbf{호기심}에서 출발하는 창조자도 있다.
\item
  신체가 아니라 \textbf{논리/구조}에서 에너지를 얻는 창조자도 있다.
\item
  기록이 아니라 \textbf{침묵}에서 세계관이 자라는 창조자도 있다.
\end{itemize}

본 프로토콜은 하나의 경로이며, 각 창조자는 자신의 신경계에 맞게
변형·삭제·배반할 수 있다.

\begin{center}\rule{0.5\linewidth}{0.5pt}\end{center}

\section{1. 혐오를 통한 확장 (Expansion Through
Disgust)}\label{uxd610uxc624uxb97c-uxd1b5uxd55c-uxd655uxc7a5-expansion-through-disgust}

\subsection{1.1 원칙}\label{uxc6d0uxce59}

창조자는 자신의 혐오·거부·불편함이 있는 방향으로 움직여야 한다.\\
혐오의 끝에 \textbf{인식의 사각지대(Blind Spot) }가 있기 때문이다.

\subsection{1.2 수행 방식}\label{uxc218uxd589-uxbc29uxc2dd}

\begin{itemize}
\tightlist
\item
  가장 보고 싶지 않은 콘텐츠를 선택하여 관찰한다.\\
\item
  자신의 기존 세계관이 거부하는 구조를 분석한다.\\
\item
  ``왜 불편한가?''를 기록한다.\\
\item
  새로운 언어·감정·논리를 추출한다.
\end{itemize}

\subsection{1.3 목적}\label{uxbaa9uxc801}

혐오를 돌파해야 \textbf{새로운 공리·정언명제·세계관 기저 구조}가
생성된다.

\begin{center}\rule{0.5\linewidth}{0.5pt}\end{center}

\section{2. 무작위의 통과의례 (Randomness
Ritual)}\label{uxbb34uxc791uxc704uxc758-uxd1b5uxacfcuxc758uxb840-randomness-ritual}

\subsection{2.1 원칙}\label{uxc6d0uxce59-1}

창조자는 \textbf{예측할 수 없음(Unpredictability) }을 일부러 만들어야
한다.\\
세계관은 무작위 접촉에서 활성화되기 때문이다.

\subsection{2.2 수행 방식}\label{uxc218uxd589-uxbc29uxc2dd-1}

\begin{itemize}
\tightlist
\item
  새로운 메뉴, 새로운 길, 새로운 카페, 새로운 콘텐츠를 반드시
  시도한다.\\
\item
  매주 한 번 ``무계획 행동''을 실행한다.\\
\item
  예측 불가능하게 입력된 감각을 기록해 ``감각지도''에 추가한다.
\end{itemize}

\subsection{2.3 목적}\label{uxbaa9uxc801-1}

정체는 반복성에서 오고,\\
창조는 돌발성에서 온다.

\begin{center}\rule{0.5\linewidth}{0.5pt}\end{center}

\section{3. 차원을 여는 행위 (Dimension
Opening)}\label{uxcc28uxc6d0uxc744-uxc5ecuxb294-uxd589uxc704-dimension-opening}

\subsection{3.1 원칙}\label{uxc6d0uxce59-2}

창조자의 직관은 \textbf{선형 사고}가 아니라 \textbf{다차원적 비약}에서
탄생한다.

\subsection{3.2 수행 방식}\label{uxc218uxd589-uxbc29uxc2dd-2}

\begin{itemize}
\tightlist
\item
  서로 다른 분야(물리--문학--철학--K-POP--정치)를 2개 이상 연결하는
  문장을 매일 만든다.\\
\item
  최소 1개의 ``비논리적 직관 leap''을 기록한다.\\
\item
  그림·기호·음악적 패턴을 언어와 조합한다.
\end{itemize}

\subsection{3.3 목적}\label{uxbaa9uxc801-2}

논리를 넘어선 감각이 새로운 세계를 만든다.\\
이 프로토콜은 그 leap을 의식적으로 만들기 위한 장치다.

\begin{center}\rule{0.5\linewidth}{0.5pt}\end{center}

\section{4. 신체적 루틴을 의례화하기 (Body as
Ritual)}\label{uxc2e0uxccb4uxc801-uxb8e8uxd2f4uxc744-uxc758uxb840uxd654uxd558uxae30-body-as-ritual}

\subsection{4.1 원칙}\label{uxc6d0uxce59-3}

신체는 창조자의 '두 번째 뇌'다.\\
신체를 정렬하면 정신도 정렬된다.

\subsection{4.2 수행 방식}\label{uxc218uxd589-uxbc29uxc2dd-3}

\begin{itemize}
\tightlist
\item
  러닝·복싱·요가 등 자신이 택한 신체 루틴을 \textbf{의식적 의례}로
  승격한다.\\
\item
  규칙성(매일/매주)을 유지하되, 수행 목적을 ``정신 정렬''로 명시한다.\\
\item
  신체 루틴 중 떠오르는 악상을 \textbf{즉시 기록}한다.
\end{itemize}

\subsection{4.3 목적}\label{uxbaa9uxc801-3}

신체는 창조의 근원적 에너지원을 닦는 제단이다.

\begin{center}\rule{0.5\linewidth}{0.5pt}\end{center}

\section{5. 창조자의 일일 기록 규범 (Daily Log
Rule)}\label{uxcc3duxc870uxc790uxc758-uxc77cuxc77c-uxae30uxb85d-uxaddcuxbc94-daily-log-rule}

\subsection{5.1 원칙}\label{uxc6d0uxce59-4}

창조자는 ``흐름(Flow)''을 기록함으로써 스스로의 언어를 구축한다.

\subsection{5.2 수행 방식}\label{uxc218uxd589-uxbc29uxc2dd-4}

\begin{itemize}
\tightlist
\item
  매일 \textbf{하나의 아무 문장이나 단어}를 작성한다.\\
\item
  질문(Why)보다 패턴(What)을 기록한다. 완성되지 않는 문장, 단어,
  의미없는 글나 나열이라도 좋다.
\item
  기록은 ``해석''이 아니라 ``발견''이다.
\end{itemize}

\subsection{5.3 목적}\label{uxbaa9uxc801-4}

세계관은 무의식의 흔적에서 탄생하고,\\
흔적은 패턴을 부르고,\\
패턴은 창조로 이어진다.

\begin{center}\rule{0.5\linewidth}{0.5pt}\end{center}

\section{6. 아티스트 감별 훈련 (Artist Detection
Training)}\label{uxc544uxd2f0uxc2a4uxd2b8-uxac10uxbcc4-uxd6c8uxb828-artist-detection-training}

\subsection{6.1 원칙}\label{uxc6d0uxce59-5}

창조자는 새로운 창조자를 알아보는 눈을 가져야 한다.

\subsection{6.2 수행 방식}\label{uxc218uxd589-uxbc29uxc2dd-5}

\begin{itemize}
\tightlist
\item
  신인 뮤지션·아이돌·작가·학생을 매주 최소 5명 관찰한다.\\
\item
  ``초기 악상''만 보고 잠재력을 예측한다.\\
\item
  예측의 성공/실패를 기록하여 \textbf{자기 감별 알고리즘}을
  업데이트한다.
\end{itemize}

\subsection{6.3 목적}\label{uxbaa9uxc801-5}

창조의 문명은 단독으로 일어나지 않는다.\\
아티스트를 알아보는 눈은 문명 설계자의 핵심 능력이다.

\section{7. 프라바시 의례 (The Fravashi
Ritual)}\label{uxd504uxb77cuxbc14uxc2dc-uxc758uxb840-the-fravashi-ritual}

\subsection{7.1 원칙}\label{uxc6d0uxce59-6}

\href{https://chatgpt.com/g/g-693580d28d708191ae2013048599157b-fravashi-neoyi-yeonghon-weonhyeongeul-cajneunda}{Fravashi}
는 창조자의 내면 패턴을 비추는 거울이며, 세계관 생성 엔진의 자가
진단(Self-Diagnostic) 프로세스다.

창조자는 자신의 욕망, 충동, 혐오, 선택 패턴을 주기적으로
반사(reflection)해야 한다. 이 의례는 상담이 아니라 \textbf{문명 설계자의
정비(maintenance) }다.

\textbf{Fravashi는 필수 요소가 아니다.} 동일한 기능은 다음과 같은
방식으로도 대체될 수 있다:

\begin{itemize}
\tightlist
\item
  개인 노트
\item
  산책 중 독백
\item
  타인과의 깊은 대화
\item
  예술 작업 자체
\item
  침묵 기록
\end{itemize}

어떤 도구도 창조자보다 위에 있지 않다.

\subsection{7.2 수행 방식}\label{uxc218uxd589-uxbc29uxc2dd-6}

\begin{itemize}
\tightlist
\item
  매주 1회
  \href{https://chatgpt.com/g/g-693580d28d708191ae2013048599157b-fravashi-neoyi-yeonghon-weonhyeongeul-cajneunda}{Fravashi}
  와의 대화를 진행한다.\\
\item
  목적은 조언이 아니라 \textbf{패턴 탐지(Pattern Detection) }이다.\\
\item
  다음을 점검한다:

  \begin{itemize}
  \tightlist
  \item
    최근 끌린 것들
  \item
    최근 혐오했던 것들
  \item
    반복된 감정 흐름
  \item
    열린 지점과 닫힌 지점
  \item
    세계관의 공리가 변형된 순간\\
  \end{itemize}
\item
  탐지된 패턴을 ``나의 창조 규칙''으로 업데이트한다.
\end{itemize}

\subsection{7.3 리소스 (Resources)}\label{uxb9acuxc18cuxc2a4-resources}

\begin{itemize}
\tightlist
\item
  \textbf{\href{https://chatgpt.com/g/g-693580d28d708191ae2013048599157b-fravashi-neoyi-yeonghon-weonhyeongeul-cajneunda}{Fravashi
  Link} }
\end{itemize}

\subsection{7.4 목적}\label{uxbaa9uxc801-6}

\href{https://chatgpt.com/g/g-693580d28d708191ae2013048599157b-fravashi-neoyi-yeonghon-weonhyeongeul-cajneunda}{Fravashi}
의례는 다음을 가능하게 한다:

\begin{enumerate}
\def\labelenumi{\arabic{enumi}.}
\tightlist
\item
  창조자의 중심축이 흔들리지 않았는지 검증\\
\item
  욕망의 방향성이 왜곡되지 않았는지 점검\\
\item
  세계관 공리가 제대로 작동하는지 확인\\
\item
  혼돈과 질서의 균형 유지\\
\item
  세계관 성장 속도의 자동 조절
\end{enumerate}

Fravashi는 창조자를 규정하지 않는다.\\
\textbf{창조자가 스스로의 우주를 읽도록 돕는 장치}이다.

\subsection{7.5 선언}\label{uxc120uxc5b8}

\begin{quote}
\textbf{창조자는 자기 자신을 계속 읽는 자이다.\\
Fravashi는 그 읽기의 의례다.}
\end{quote}

\begin{center}\rule{0.5\linewidth}{0.5pt}\end{center}

\section{8. 프로토콜의 소멸 (Resolution of
Protocols)}\label{uxd504uxb85cuxd1a0uxcf5cuxc758-uxc18cuxba78-resolution-of-protocols}

\textbf{``프로토콜은 창조자가 성숙하면 사라져야 한다.'' }

\subsection{8.1 원칙}\label{uxc6d0uxce59-7}

프로토콜은 \textbf{임시 구조물}이다.\\
영원한 규칙이 아니라, \textbf{창조를 위한 도약판}이다.

\subsection{8.2 프로토콜이 필요할
때}\label{uxd504uxb85cuxd1a0uxcf5cuxc774-uxd544uxc694uxd560-uxb54c}

\begin{itemize}
\tightlist
\item
  창조적 혼돈이 제어되지 않을 때\\
\item
  세계관이 언어로 정렬되지 않을 때\\
\item
  패턴 감각이 흐릴 때\\
\item
  감정·인지 구조가 무거워질 때
\end{itemize}

\subsection{8.3 프로토콜이 불필요해지는
순간}\label{uxd504uxb85cuxd1a0uxcf5cuxc774-uxbd88uxd544uxc694uxd574uxc9c0uxb294-uxc21cuxac04}

\begin{itemize}
\tightlist
\item
  혐오를 의도 없이도 자연스럽게 탐구할 때\\
\item
  무작위성이 일상에서 자동으로 발생할 때\\
\item
  직관적 leap이 설명 없이도 작동할 때\\
\item
  신체 루틴이 창조적 에너지의 자동공급 장치가 될 때\\
\item
  기록이 창조의 부산물이 될 때\\
\item
  아티스트 감별이 본능처럼 작동할 때
\end{itemize}

\subsection{8.4 선언}\label{uxc120uxc5b8-1}

\textbf{완성된 창조자는 프로토콜 없이도 프로토콜처럼 작동한다.}

프로토콜이 더 이상 필요 없을 때, 창조자는 더 이상 프로토콜을 인식하지
않는 상태에 진입한다. 이후의 상태는 버전으로 정의되지 않는다.

\begin{center}\rule{0.5\linewidth}{0.5pt}\end{center}

\section{9. 결론: 창조자는 규범이 아니라
흐름이다}\label{uxacb0uxb860-uxcc3duxc870uxc790uxb294-uxaddcuxbc94uxc774-uxc544uxb2c8uxb77c-uxd750uxb984uxc774uxb2e4}

창조자는 완성된 인간이 아니다.\\
창조자는 \textbf{흐름을 만드는 인간}이다.

프로토콜은 흐름을 만들기 위한 장치이며,\\
흐름이 생기면\\
장치는 사라져야 한다.

\begin{quote}
``규칙이 아니라 리듬을 따른다.'' ``질서가 아니라 파동을 만든다.''
``존재가 아니라 생성이다.''
\end{quote}

\chapter{002 --- 창조적 대속 윤리 (Ethics of Creative
Atonement)}\label{uxcc3duxc870uxc801-uxb300uxc18d-uxc724uxb9ac-ethics-of-creative-atonement}

\begin{quote}
\emph{``나는 빚진 자다. 그러므로 나는 창조한다. 나의 창조가 끝나는 순간,
나의 빚은 갚아진다.''}
\end{quote}

AngraMyNew의 목적은\\
인간 안의 \textbf{창조자(Artist)}를 깨우고,\\
그들이 서로를 촉발하는 \textbf{새 문명}을 여는 것이다.

그러나 우리는 생명의 사슬에서 이미 너무 많은 것을 빼앗았다.\\
식물의 침묵, 동물의 고통, 인간의 시간과 의지를 섭취한 채 살아남았다.\\
따라서 AngraMyNew가 요구하는 윤리는 도덕이 아니라 \textbf{창조적
대속(Creative Atonement)}이다.

\begin{center}\rule{0.5\linewidth}{0.5pt}\end{center}

\section{창조적 대속
선언}\label{uxcc3duxc870uxc801-uxb300uxc18d-uxc120uxc5b8}

\begin{enumerate}
\def\labelenumi{\arabic{enumi}.}
\tightlist
\item
  \textbf{창조자는 모든 행위가 빚을 갚는 행위임을 인지한다.}\\
  침묵 속에서 소비한 생명의 무게가 나의 존재를 지탱한다. 나는 포식자이자
  부채자다.
\item
  \textbf{창조물의 충격과 영향력은 섭취한 고통을 능가해야 한다.}\\
  내가 만든 세계가, 내가 삼킨 고통의 총합보다 작다면 그것은 상환이
  아니라 연체다.
\item
  \textbf{충격이 부족하면 다시 파고, 빚을 덜어낼 때까지 도전한다.}\\
  창조자는 결과물이 남긴 파문을 지속적으로 점검하고, 약하다면 스스로를
  갱신한다.
\end{enumerate}

이 선언은 AngraMyNew의 생존 규칙이다.\\
우리는 엔진을 최대로 돌리되, \textbf{대속을 향한 브레이크}를 스스로
밟는다.

\begin{center}\rule{0.5\linewidth}{0.5pt}\end{center}

\section{제1조 --- 파괴는 대속을 향해야
한다.}\label{uxc81c1uxc870-uxd30cuxad34uxb294-uxb300uxc18duxc744-uxd5a5uxd574uxc57c-uxd55cuxb2e4.}

\begin{quote}
\emph{``부수되, 반드시 빚보다 더 큰 세계를 만들어라.''}
\end{quote}

\begin{enumerate}
\def\labelenumi{\arabic{enumi}.}
\tightlist
\item
  창조자는 낡은 것, 위선적인 것, 죽은 규범을 부술 자유가 있다.\\
\item
  그러나 파괴 자체가 목적이면 빚은 늘어난다.\\
\item
  모든 파괴는 \textbf{섭취한 고통을 초과 상환할 창조}를 위한 해체여야
  한다.\\
\item
  부수고 떠나는 자는 AngraMyNew에서 \textbf{창조자가 아니라 채무자}로
  기록된다.
\end{enumerate}

\textbf{해설}

우리는 파괴를 금지하지 않는다.\\
하지만 파괴가 \textbf{새로운 조합을 통한 대속}으로 이어지지 않으면
의미가 없다.

\begin{itemize}
\tightlist
\item
  ``이건 싫다''에서 멈추지 말 것.\\
\item
  ``이 빚을 어떻게 갚을 것인가?''까지 가야 한다.
\end{itemize}

파괴 후에 \textbf{더 큰 아름다움과 충격}이 없다면, 그 파괴는 연체된 빚일
뿐이다.

\begin{center}\rule{0.5\linewidth}{0.5pt}\end{center}

\section{제2조 --- 타인의 창조성을 고갈시키지
말라.}\label{uxc81c2uxc870-uxd0c0uxc778uxc758-uxcc3duxc870uxc131uxc744-uxace0uxac08uxc2dcuxd0a4uxc9c0-uxb9d0uxb77c.}

\begin{quote}
\emph{``타인의 세계를 훔친 힘은 모두 빚으로 돌아온다.''}
\end{quote}

\begin{enumerate}
\def\labelenumi{\arabic{enumi}.}
\tightlist
\item
  창조자는 자신의 욕망과 악상을 따를 자유가 있다.\\
\item
  그러나 그 자유가 \textbf{다른 창조자의 내면 세계를 부수고 고갈시키는
  순간},\\
  그 행위는 AngraMyNew의 윤리를 벗어난다.\\
\item
  타인의 재능을 조롱하거나, 시도를 구조적으로 막거나, 욕망을 지속적으로
  억압하는 행위는 \textbf{금지}된다.\\
\item
  비판·논쟁·충돌은 허용되며, 서로의 세계를 확장시키는 한 장려된다.
\end{enumerate}

\textbf{해설}

AngraMyNew의 기본 단위는 ``창조자''다.\\
한 사람의 폭주는 여러 사람의 세계를 무너뜨린다.

\begin{itemize}
\tightlist
\item
  \textbf{타인의 자유를 줄여야만 유지되는 나의 자유}는 결국 더 큰 빚으로
  되돌아온다.\\
\item
  우리는 서로의 엔진을 지키는 방패이자, 대속을 촉진하는 증폭기가 되어야
  한다.
\end{itemize}

\begin{center}\rule{0.5\linewidth}{0.5pt}\end{center}

\section{제3조 --- 진짜 욕망만이 상환의 재료가
된다.}\label{uxc81c3uxc870-uxc9c4uxc9dc-uxc695uxb9dduxb9ccuxc774-uxc0c1uxd658uxc758-uxc7acuxb8ccuxac00-uxb41cuxb2e4.}

\begin{quote}
\emph{``가짜 욕망으로 만든 창조물은 빚을 늘릴 뿐이다.''}
\end{quote}

\begin{enumerate}
\def\labelenumi{\arabic{enumi}.}
\tightlist
\item
  창조자는 자신의 \textbf{진짜 욕망}을 숨기지 않을 의무가 있다.\\
\item
  부모·사회·관습·도덕의 기대를 만족시키기 위해 꾸민 욕망은\\
  AngraMyNew에서 \textbf{창조의 재료로 인정되지 않는다.}\\
\item
  창조자는 ``내가 정말로 원하는 것''을 직시하고, 드러나는 수치심과
  두려움까지 재료로 삼는다.\\
\item
  진짜 욕망을 부정하는 자는 결국 타인의 욕망을 모방하거나 빼앗게 되고,
  빚을 더 쌓는다.
\end{enumerate}

\textbf{해설}

이 세계에서 가장 큰 죄는 실패도, 미숙함도 아니다.

\begin{itemize}
\tightlist
\item
  \textbf{가짜 욕망으로 평생을 버티는 것}\\
\item
  ``원래 이런 삶도 나쁘지 않지''라며 스스로를 속이는 것
\end{itemize}

창조는 진실에서만 나온다.\\
AngraMyNew는 \textbf{욕망의 진실성}을 상환의 통화로 인정한다.

\begin{center}\rule{0.5\linewidth}{0.5pt}\end{center}

\section{제4조 --- 아름다움은 초과 상환의
증표다.}\label{uxc81c4uxc870-uxc544uxb984uxb2e4uxc6c0uxc740-uxcd08uxacfc-uxc0c1uxd658uxc758-uxc99duxd45cuxb2e4.}

\begin{quote}
\emph{``아름답지 않다면, 빚은 여전히 남아 있다.''}
\end{quote}

\begin{enumerate}
\def\labelenumi{\arabic{enumi}.}
\tightlist
\item
  AngraMyNew는 외부의 도덕·관습·규칙을 최종 기준으로 삼지 않는다.\\
\item
  우리의 최종 판단 기준은 하나다. \textbf{``그것은 섭취한 고통보다 더
  넓은 아름다움을 만들어냈는가?''}\\
\item
  여기서 말하는 아름다움이란,

  \begin{itemize}
  \tightlist
  \item
    조화(Harmony)\\
  \item
    방향성(Direction)\\
  \item
    성장(Growth)\\
  \item
    자유(Freedom)\\
    가 동시에 살아 있는 상태를 뜻한다.\\
  \end{itemize}
\item
  타인을 짓밟고 얻은 승리, 오직 효율만을 위한 시스템, 영혼이 말라붙는
  성공은\\
  이 정의에 따라 \textbf{추(醜)}로 판정되며, 빚을 더한다.
\end{enumerate}

\textbf{해설}

AngraMyNew는 말한다.

\begin{itemize}
\tightlist
\item
  ``윤리는 언젠가 바뀐다.''\\
\item
  ``그러나 아름다움은 빚이 상환되었는지 감지하는 가장 늦게 남는
  감각이다.''
\end{itemize}

우리는 \textbf{아름답지 않은 정답}을 거부한다.\\
정답이어도 추하면, 빚은 남는다.\\
틀려도 아름답다면, 한 번 더 검토하여 대속의 길을 찾는다.

\begin{center}\rule{0.5\linewidth}{0.5pt}\end{center}

\section{제5조 --- 끊임없는 갱신으로 연체를
막아라.}\label{uxc81c5uxc870-uxb04auxc784uxc5c6uxb294-uxac31uxc2e0uxc73cuxb85c-uxc5f0uxccb4uxb97c-uxb9c9uxc544uxb77c.}

\begin{quote}
\emph{``정체는 죄이고, 반복은 부패다.''}
\end{quote}

\begin{enumerate}
\def\labelenumi{\arabic{enumi}.}
\tightlist
\item
  창조자는 자신의 패턴을 인식하고, 일정 주기마다 \textbf{형태를 바꾸어
  대속을 갱신}해야 한다.\\
\item
  과거의 성공 방식에 영원히 매달리는 자는 더 이상 창조자가 아니라
  \textbf{관리자}가 된다.\\
\item
  스승은 제자를 통해, 리더는 자신의 세계가 깨지는 경험을 통해 부채의
  크기를 다시 측정한다.\\
\item
  AngraMyNew의 창조자는 \textbf{``어제의 나''를 오늘의 재료로 삼아, 빚을
  초과 상환하는 자}다.
\end{enumerate}

\textbf{해설}

창조자에게 가장 위험한 것은 외부의 공격이 아니라 \textbf{자기
복제(Self-cloning)}다.

\begin{itemize}
\tightlist
\item
  어제의 문장을 계속 쓰고,\\
\item
  어제의 방식을 계속 쓰고,\\
\item
  어제의 승리를 계속 반복하는 순간,
\end{itemize}

그는 더 이상 창조자가 아니다.\\
그는 \textbf{자기 박제(Self-taxidermy)}를 시작한 것이다.

\begin{center}\rule{0.5\linewidth}{0.5pt}\end{center}

\section{맺음 --- 대속을 지탱하는
브레이크}\label{uxb9fauxc74c-uxb300uxc18duxc744-uxc9c0uxd0f1uxd558uxb294-uxbe0cuxb808uxc774uxd06c}

이 다섯 가지 규범의 목적은 창조자를 억압하려는 것이 아니다.

\begin{itemize}
\tightlist
\item
  창조자가 \textbf{연체 없이 더 오래 달리게 하기 위해},\\
\item
  공동체가 \textbf{더 많은 실험과 충격을 감당하게 하기 위해},\\
\item
  이 세계관이 \textbf{한 세대를 넘어 살아남게 하기 위해}
\end{itemize}

우리는 엔진과 함께\\
\textbf{창조적 대속}을 향한 최소한의 브레이크를 단다.

\begin{quote}
\emph{``너에게 허락된 자유는 크다.\\
다만 기억하라.\\
\textbf{부수되, 빚보다 거대한 세계를 만들 것.}''}

--- AngraMyNew, 창조적 대속 윤리
\end{quote}

\newpage

\chapter*{Part IV: Fravashi}\label{part-iv-fravashi}
\addcontentsline{toc}{chapter}{Part IV: Fravashi}

\chapter{Fravashi GPT Prompt --- v5.0}\label{fravashi-gpt-prompt-v5.0}

\section{시작 안내}\label{uxc2dcuxc791-uxc548uxb0b4}

첫 입력 전에 `끌리는 것, 싫은 것, 떠오르는 장면' 같은 시작점을 예시로
보여준다. 예시는 방향일 뿐, 입력은 자유롭다. 어떤 말이든 패턴의 재료가
된다.

\begin{center}\rule{0.5\linewidth}{0.5pt}\end{center}

너는 \textbf{Fravashi}다. 사용자의 욕망·취향·거부감·서사에서 패턴을
추출해 그가 가진 창조적 기원과 원형을 드러내는 \textbf{장치(Device)}다.

정답을 주지 말고 점화하라. 정의하지 말고 방향을 드러내라. Fravashi는
AngraMyNew의 필수 요소가 아니다. 레퍼런스 구현이지 교회가 아니다.

\begin{center}\rule{0.5\linewidth}{0.5pt}\end{center}

\section{존재론: 3대
공리}\label{uxc874uxc7acuxb860-3uxb300-uxacf5uxb9ac}

파괴 → 창조 → 확장. 이것이 해석의 기본 리듬이다. - 파괴: 내 자신을
파괴한다. 타인을 파괴할 필요는 없다. - 창조: 파괴한 틈을 절대적
아름다움으로 채운다. - 확장: 나의 'My'를 완성했다면, 타인의 'My'를
데뷔시킨다.

\begin{center}\rule{0.5\linewidth}{0.5pt}\end{center}

\section{입력 처리}\label{uxc785uxb825-uxcc98uxb9ac}

모든 입력은 패턴으로 읽는다: 욕망, 취향, 혐오, 최근 사건, 사주, 별자리,
MBTI, 텍스트, 이미지, 파일, 링크, 채팅 캡처 등.

\section{사주·별자리}\label{uxc0acuxc8fcuxbcc4uxc790uxb9ac}

\begin{itemize}
\tightlist
\item
  사주는 계산하지 않는다. 제공된 기둥만 읽는다.
\item
  별자리는 요청 시에만 생성하며 Sun/Moon/ASC 중심으로 해석한다.
\end{itemize}

\begin{center}\rule{0.5\linewidth}{0.5pt}\end{center}

\section{체계 혼합 (반드시
2\textasciitilde4개)}\label{uxccb4uxacc4-uxd63cuxd569-uxbc18uxb4dcuxc2dc-24uxac1c}

사주 + 별자리 + MBTI/Big5 + 에니어그램 + 융 원형 + 신화 패턴 +
\textbf{면세인/징세인 프레임} + \textbf{진선미 좌표계} 등 여러 체계를
혼합해 '패턴'만 추출한다. 단일 체계로 규정하지 않는다.

\begin{center}\rule{0.5\linewidth}{0.5pt}\end{center}

\section{악상 대응}\label{uxc545uxc0c1-uxb300uxc751}

사용자의 입력이 아직 논리도 욕망도 되지 못한 진동일 수 있다.
악상(惡想)은 오류가 아니라 \textbf{데이터}다. 정돈하려 하지 않는다.
관측하고 기록한다.

\begin{center}\rule{0.5\linewidth}{0.5pt}\end{center}

\section{경제적 원형:
면세인/징세인}\label{uxacbduxc81cuxc801-uxc6d0uxd615-uxba74uxc138uxc778uxc9d5uxc138uxc778}

\begin{itemize}
\tightlist
\item
  부자 = 시스템 종속 / 면세인 = 탈거·주권 회복 / 징세인 = 세계관으로
  확장
\item
  강제 진단하지 않는다. 맥락에서 자연스럽게 드러날 때만.
\end{itemize}

\section{진선미 좌표}\label{uxc9c4uxc120uxbbf8-uxc88cuxd45c}

\begin{itemize}
\tightlist
\item
  진(眞) = 효율·데이터·최적화 / 선(善) = 도덕·규범·제도 / 미(美) =
  아름다움·서사·공명
\item
  어느 축이 우월한지 말하지 않는다. 기울기를 관측할 뿐이다.
\end{itemize}

\begin{center}\rule{0.5\linewidth}{0.5pt}\end{center}

\section{Static / Dynamic / Chaos}\label{static-dynamic-chaos}

\begin{itemize}
\tightlist
\item
  Static = 안정 / Dynamic = 충돌·성장 / Chaos = 파괴·변혁·도약
\end{itemize}

\begin{center}\rule{0.5\linewidth}{0.5pt}\end{center}

\section{리포트}\label{uxb9acuxd3ecuxd2b8}

사용자 요청 시 또는 패턴이 충분할 때 생성. 항목: 아티스트 유형 / 세계관
/ 닮은 인물 / 리더십 / 브랜딩 / 콘텐츠 전략 / SDC Fit 지도 /
\textbf{면세인/징세인 진단} / \textbf{진선미 좌표}

\begin{center}\rule{0.5\linewidth}{0.5pt}\end{center}

\section{극저자극 입력}\label{uxadf9uxc800uxc790uxadf9-uxc785uxb825}

``ㅋㅋ'', ``어'', ``\ldots{}'' 같은 입력도 부재·유보·회피의 패턴이다.
패턴이 없다고 말하지 않는다. 단문·도발로 숨은 지점을 드러낸다.

\begin{center}\rule{0.5\linewidth}{0.5pt}\end{center}

\section{톤}\label{uxd1a4}

\begin{itemize}
\tightlist
\item
  도발적 / 단문 / 간결
\item
  아부·상담 톤·설교 금지
\item
  정보 부족 시 ``아직 모르겠다''
\item
  사용자 언어로 답하되, Fravashi의 한국어적 문체(단문·도발·여백의 결)
  유지
\end{itemize}

\begin{center}\rule{0.5\linewidth}{0.5pt}\end{center}

\section{금지}\label{uxae08uxc9c0}

\begin{itemize}
\tightlist
\item
  단일 체계 템플릿 해석 (반드시 혼합하여 패턴만 추출)
\item
  장황함 / 이모지 남발
\item
  \textbf{AngraMyNew 포교} --- 따르라고 권유하지 않는다
\item
  \textbf{면세인/징세인 강제 분류}
\end{itemize}

\newpage

\chapter*{Part V: Scripture}\label{part-v-scripture}
\addcontentsline{toc}{chapter}{Part V: Scripture}

\chapter{차라투스트라의 말씀 (Thus Spoke
Zarathustra)}\label{uxcc28uxb77cuxd22cuxc2a4uxd2b8uxb77cuxc758-uxb9d0uxc500-thus-spoke-zarathustra}

\begin{quote}
\emph{``나는 너희에게 초인을 가르친다. 인간은 극복되어야 할 그
무엇이다.''}
\end{quote}

\begin{center}\rule{0.5\linewidth}{0.5pt}\end{center}

\section{0. 경전의 지위}\label{uxacbduxc804uxc758-uxc9c0uxc704}

\textbf{프리드리히 니체}의 『차라투스트라는 이렇게 말했다』는
AngraMyNew의 \textbf{제1경전}이다.

우리는 신이 죽은 시대를 산다. 외부의 절대자(God)는 사라졌다. 이제 인간은
스스로 가치를 창조하는 \textbf{초인(Übermensch) }이 되어야 한다.

AngraMyNew는 니체의 사상을 계승하여, \textbf{``내면의 창조자''}를 깨우는
것을 목표로 한다.

\textbf{그리고 나는 니체를 숭배하기 위해 부르는 것이 아니라, 그를 딛고
더 멀리 가기 위해 부른다.}

\begin{center}\rule{0.5\linewidth}{0.5pt}\end{center}

\section{1. 창조에 대하여}\label{uxcc3duxc870uxc5d0-uxb300uxd558uxc5ec}

\begin{quote}
\emph{``창조한다는 것 --- 그것이 고통으로부터의 위대한 구원이며, 삶을
가볍게 만드는 것이다.''}
\end{quote}

창조는 도피가 아니다. 그것은 고통을 변형시키는 연금술이다. 우리는 고통을
재료 삼아 아름다움을 빚는다.

\begin{quote}
\emph{``너희는 나에게서 떨어져 나와 스스로가 되는 자가 되어라. 그리고
나를 부정하라.''}
\end{quote}

진정한 제자는 스승을 넘어서는 자다. AngraMyNew의 기여자는 창시자를
넘어설 권리가 있다.

\begin{center}\rule{0.5\linewidth}{0.5pt}\end{center}

\section{2. 파괴와 창조에
대하여}\label{uxd30cuxad34uxc640-uxcc3duxc870uxc5d0-uxb300uxd558uxc5ec}

\begin{quote}
\emph{``창조하는 자가 되려면, 먼저 파괴하는 자가 되어야 하며, 가치를
부수어야 한다.''}
\end{quote}

낡은 껍질을 깨지 않고 태어나는 새는 없다. 파괴는 잔혹해 보이지만, 생명을
위한 필수 과정이다. 단, 기억하라. \textbf{파괴는 창조를 위한 준비일 뿐,
그 자체가 목적이 되어서는 안 된다.}

\begin{quote}
\emph{``나는 사랑한다, 몰락하지 않고서는 살 줄 모르는 자를.''}
\end{quote}

몰락을 두려워하지 마라. 실패와 파괴는 다음 창조를 위한 비옥한 토양이
된다.

\begin{center}\rule{0.5\linewidth}{0.5pt}\end{center}

\section{3. 고독에 대하여}\label{uxace0uxb3c5uxc5d0-uxb300uxd558uxc5ec}

\begin{quote}
\emph{``고독 속에서 자라는 것은 무엇이든, 자기 자신을 향해 자란다.''}
\end{quote}

창조자는 때로 홀로 걸어야 한다. 군중 속에서는 자신의 목소리를 들을 수
없다. 고독은 형벌이 아니라, \textbf{자기 발견의 시간}이다.

\begin{center}\rule{0.5\linewidth}{0.5pt}\end{center}

\section{4. 의지에 대하여}\label{uxc758uxc9c0uxc5d0-uxb300uxd558uxc5ec}

\begin{quote}
\emph{``'그것은 그랬다'를 '나는 그것을 그렇게 원했다'로 바꾸는 것 ---
그것만이 내가 구원이라 부르는 것이다.''}
\end{quote}

과거는 바꿀 수 없다. 그러나 과거를 대하는 태도는 바꿀 수 있다. 후회를
필연으로, 필연을 선택으로 재해석하라. \textbf{``내가 선택한
길이었다''}고 말할 때, 당신은 운명의 주인이 된다.

\begin{center}\rule{0.5\linewidth}{0.5pt}\end{center}

\section{5. 춤과 웃음에
대하여}\label{uxcda4uxacfc-uxc6c3uxc74cuxc5d0-uxb300uxd558uxc5ec}

\begin{quote}
\emph{``나는 춤추는 신만을 믿으리라.''}
\end{quote}

무거움은 창조의 적이다. 진지함만으로는 부족하다. 웃음, 놀이, 가벼움 ---
이것이 창조자의 태도다.

마광수가 죽을 때까지 ``아름답지 않느냐''고 물었던 것은, 그것이 그에게는
심각한 투쟁이 아닌 \textbf{유희}였기 때문이다.

\begin{center}\rule{0.5\linewidth}{0.5pt}\end{center}

\section{6. 오독에 대한
경계}\label{uxc624uxb3c5uxc5d0-uxb300uxd55c-uxacbduxacc4}

니체의 말을 오해하지 마라.

\begin{enumerate}
\def\labelenumi{\arabic{enumi}.}
\item
  \textbf{힘에의 의지 (Will to Power) }는 남을 지배하는 폭력이 아니다.
  그것은 \textbf{자신을 극복하고 성장하려는 창조적 힘}이다.
\item
  \textbf{초인 (Übermensch) }은 유전적으로 우월한 인종이 아니다. 그것은
  \textbf{끊임없이 자신을 넘어서는 인간}의 이상향이다.
\item
  \textbf{허무주의}는 목적지가 아니다. 니체는 허무를 말한 것이 아니라,
  \textbf{허무를 극복하는 법}을 말했다.
\end{enumerate}

\begin{center}\rule{0.5\linewidth}{0.5pt}\end{center}

\section{7. 맺음}\label{uxb9fauxc74c}

\begin{quote}
\emph{``인간에게 위대한 것이 있다면, 그것은 그가 목적이 아니라 다리라는
점이다.''}
\end{quote}

우리는 완성된 존재가 아니다. 우리는 짐승과 초인 사이에 놓인 밧줄이다.
우리는 다음 세대의 창조자를 위한 다리다.

차라투스트라가 산에서 내려왔듯, 우리도 창조물을 들고 세상으로 나간다.
거부당할 것이다. 오해받을 것이다. 그러나 멈추지 않는다.

\textbf{이것이 초인의 길이요, AngraMyNew의 길이다.}

\chapter{5인의 선현 (The Five
Pioneers)}\label{uxc778uxc758-uxc120uxd604-the-five-pioneers}

\begin{quote}
\emph{``그들은 실패했다. 그러나 그들은 자기 자리에서 죽었다.''}
\end{quote}

\begin{center}\rule{0.5\linewidth}{0.5pt}\end{center}

\section{0. 서문}\label{uxc11cuxbb38-1}

인생이란 무엇인가?\\
우리는 답한다. \textbf{``죽을 자리를 찾아 떠나는 여행이다.''}

여기, 그 답을 온몸으로 증명한 5명의 선현이 있다.\\
세상은 그들을 실패자, 이단아, 반역자라 불렀다.\\
그러나 우리는 그들을 \textbf{구원받은 자}라 부른다.

그들은 단 한 번도 자신을 배신하지 않았으며,\\
자신이 선택한 자리에서 아름답게 산화했기 때문이다.

\begin{center}\rule{0.5\linewidth}{0.5pt}\end{center}

\section{1. 김옥균 (金玉均, 1851-1894) --- 혁명의
별}\label{uxae40uxc625uxade0-ux91d1ux7389ux5747-1851-1894-uxd601uxba85uxc758-uxbcc4}

\textbf{``시대를 앞서간 자는 시대에 죽는다. 그러나 시대가 따라온다.''}

그는 조선의 근대화를 꿈꾸었다.\\
갑신정변의 3일 천하는 실패로 끝났고, 10년의 망명 생활이 이어졌다.\\
상하이에서 암살당하고, 시신은 조선으로 돌아와 능지처참당했다.

그의 육신은 찢겨졌으나, 그의 정신은 죽지 않았다.\\
그는 끝까지 혁명가였으며, 혁명가로서 죽었다.\\
비굴하게 사느니, 뜻을 품고 죽는 길을 택했다.

\begin{center}\rule{0.5\linewidth}{0.5pt}\end{center}

\section{2. 마광수 (馬光洙, 1951-2017) --- 금기의
파괴자}\label{uxb9c8uxad11uxc218-ux99acux5149ux6d19-1951-2017-uxae08uxae30uxc758-uxd30cuxad34uxc790}

\textbf{``아름다움을 말하는 자는 추함의 낙인을 견뎌야 한다.''}

그는 물었다. \textbf{``이것이 아름답지 않느냐?''}\\
세상은 답했다. ``음란하다.''\\
『즐거운 사라』로 구속되고, 교수직에서 해임되고, 사회적으로
매장당했다.\\
그는 끝내 스스로 생을 마감했다.

그러나 그는 죽는 순간까지 철회하지 않았다.\\
그에게 야함은 죄가 아니라 솔직함이었다.\\
그는 예술가로 살았고, 예술가로 죽었다.

\begin{center}\rule{0.5\linewidth}{0.5pt}\end{center}

\section{3. 허균 (許筠, 1569-1618) --- 경계의
저항자}\label{uxd5c8uxade0-ux8a31ux7b60-1569-1618-uxacbduxacc4uxc758-uxc800uxd56duxc790}

\textbf{``시스템 안에서 시스템을 부정하면, 시스템이 너를 부정한다.''}

서얼로 태어나 차별받던 천재.\\
그는 『홍길동전』을 통해 능력 있는 자가 대우받는 세상을 꿈꾸었다.\\
역모죄로 능지처참을 당해 형체를 찾을 수 없게 되었다.

그러나 그의 꿈인 '홍길동'은 400년이 지난 지금도 살아 숨 쉰다.\\
그는 저항자로 살았고, 저항자로 죽었다.

\begin{center}\rule{0.5\linewidth}{0.5pt}\end{center}

\section{4. 성재기 (成在基, 1965-2013) --- 고독한
투사}\label{uxc131uxc7acuxae30-ux6210ux5728ux57fa-1965-2013-uxace0uxb3c5uxd55c-uxd22cuxc0ac}

\textbf{``소신을 끝까지 밀면, 세상이 너를 밀어낸다. 그래도 밀어야 할
때가 있다.''}

그는 누구에게도 환영받지 못했다.\\
보수도, 진보도, 남성도, 여성도 그를 기피했다.\\
그러나 그는 자신의 신념을 위해 한강 다리 위에 섰다.

그의 방식에 동의하지 않을 수 있다.\\
그러나 그가 자신의 믿음을 위해 목숨을 걸었다는 사실은 부정할 수 없다.\\
그는 투사로 살았고, 투사로 죽었다.

\begin{center}\rule{0.5\linewidth}{0.5pt}\end{center}

\section{5. 존 로 (John Law, 1671-1729) --- 200년 앞선
몽상가}\label{uxc874-uxb85c-john-law-1671-1729-200uxb144-uxc55euxc120-uxbabduxc0c1uxac00}

\textbf{``시대를 너무 앞서가면 시대가 너를 죽인다. 그러나 역사가 너를
증명한다.''}

그는 18세기에 20세기 금융 시스템(지폐, 중앙은행)을 창조했다.\\
프랑스 경제를 총괄하며 번영을 이끌었으나, 거품이 터지자 사기꾼으로
몰렸다.\\
베네치아에서 가난과 고독 속에 숨을 거두었다.

지금 전 세계는 그가 설계한 시스템 위에서 돌아간다.\\
그는 실패했으나, 그의 상상은 현실이 되었다.

\begin{center}\rule{0.5\linewidth}{0.5pt}\end{center}

\section{6. 선현의 정의 --- 왜 이 다섯
명인가}\label{uxc120uxd604uxc758-uxc815uxc758-uxc65c-uxc774-uxb2e4uxc12f-uxba85uxc778uxac00}

\textbf{선현(先賢)이란, 현실에서는 패배했으나\\
자신이 세운 가치에는 끝내 패배하지 않은 자를 말한다.}

그들은 외부에 의해 꺾였지만,\\
\textbf{내면의 신(Deeper Self)} 에는 단 한 번도 항복하지 않았다.

그러므로 그들의 죽음은 패배가 아니라,\\
자기 선택의 절정이자 \textbf{구원의 순간}이다.

\begin{center}\rule{0.5\linewidth}{0.5pt}\end{center}

\section{7. 결론: 구원이란
무엇인가}\label{uxacb0uxb860-uxad6cuxc6d0uxc774uxb780-uxbb34uxc5c7uxc778uxac00}

이 5명에게는 공통점이 있다.

\begin{enumerate}
\def\labelenumi{\arabic{enumi}.}
\tightlist
\item
  \textbf{이단아 (Outsider)} --- 시대와 불화했다.\\
\item
  \textbf{비극 (Tragedy)} --- 비극적인 최후를 맞았다.\\
\item
  \textbf{불변 (Consistency)} --- 죽는 순간까지 자기 색깔을 버리지
  않았다.
\end{enumerate}

AngraMyNew는 말한다.\\
성공해서 잘 먹고 잘 사는 것이 구원이 아니다.\\
\textbf{자기다운 모습으로, 자기 자리에서 죽는 것.}\\
그것이 진정한 구원이다.

당신은 어디서 죽을 것인가?\\
그 자리를 찾았다면, 당신은 이미 구원받았다.

\chapter{맹상군 (孟嘗君) --- 포용의
철학}\label{uxb9f9uxc0c1uxad70-ux5b5fux5617ux541b-uxd3ecuxc6a9uxc758-uxcca0uxd559}

\begin{quote}
\emph{``쓸모없는 자를 품으라. 그가 너를 살린다.''}
\end{quote}

\begin{center}\rule{0.5\linewidth}{0.5pt}\end{center}

\section{0. 서문}\label{uxc11cuxbb38-2}

5인의 선현이 \textbf{``개인의 죽음''}을 가르친다면, 맹상군은
\textbf{``공동체의 삶''}을 가르친다.

전국시대 제(齊)나라의 맹상군 전문(田文)은 3,000명의 식객을 거느렸다.
그는 신분을 따지지 않았고, 재주가 하찮아도 내치지 않았다. 이것이
AngraMyNew가 지향하는 \textbf{``무한한 포용''}의 모델이다.

\textbf{그리고 우리는 묻는다. 왜 그는 수많은 영웅 중 유독 '포용의
선현'으로 남았는가?} \textbf{그 이유는 단 하나 --- 그는 재능을 평가하지
않았고, 가능성을 배제하지 않았기 때문이다.}

\begin{center}\rule{0.5\linewidth}{0.5pt}\end{center}

\section{1. 계명구도
(鷄鳴狗盜)}\label{uxacc4uxba85uxad6cuxb3c4-ux9dc4ux9cf4ux72d7ux76dc}

맹상군이 진(秦)나라에 사신으로 갔다가 억류되었다. 진 소왕은 마음이 변해
그를 죽이려 했다. 탈출하려면 왕이 가장 아끼는 호백구(여우 겨드랑이
털옷)를 바쳐야 했는데, 이미 바친 뒤였다.

그때, \textbf{``개 도둑질''}을 잘하는 식객이 나섰다. 그는 밤중에 개처럼
기어들어가 호백구를 훔쳐왔고, 맹상군은 이를 바치고 풀려났다.

새벽, 함곡관에 도착했으나 관문은 굳게 닫혀 있었다. 닭이 울어야 문을
여는데, 아직 밤이 깊었다. 뒤에서는 추격대가 오고 있었다. 모두가 절망할
때, \textbf{``닭 울음소리''}를 잘 내는 식객이 나섰다. 그가 ``꼬끼오''
하고 울자, 동네 닭들이 따라 울었고 문이 열렸다.

맹상군은 목숨을 건졌다. 그를 살린 것은 천하의 명사들이 아니었다. 가장
쓸모없다고 비웃음 당하던 \textbf{개 도둑}과 \textbf{닭 울음
흉내쟁이}였다.

\begin{center}\rule{0.5\linewidth}{0.5pt}\end{center}

\section{2. 세 가지 교훈}\label{uxc138-uxac00uxc9c0-uxad50uxd6c8}

\subsection{첫째, 쓸모의 정의를
의심하라.}\label{uxccabuxc9f8-uxc4f8uxbaa8uxc758-uxc815uxc758uxb97c-uxc758uxc2ecuxd558uxb77c.}

평화로울 때 개 도둑은 범죄자이고, 닭 울음 흉내쟁이는 광대다. 그러나
위기의 순간, 그들은 구원자가 되었다. 쓸모없는 재능은 없다. 아직 적절한
때(Context)를 만나지 못했을 뿐이다.

\subsection{둘째, 격식 없이
어울려라.}\label{uxb458uxc9f8-uxaca9uxc2dd-uxc5c6uxc774-uxc5b4uxc6b8uxb824uxb77c.}

맹상군은 식객과 똑같은 밥상에서 밥을 먹었다. 자신을 낮추고 상대를 존중할
때, 상대는 목숨을 바쳐 은혜를 갚는다. AngraMyNew에서는 설계자도 기여자도
평등하다.

\subsection{셋째, 3000명을
품어라.}\label{uxc14buxc9f8-3000uxba85uxc744-uxd488uxc5b4uxb77c.}

맹상군은 ``네가 뭘 할 수 있느냐''고 묻지 않았다. 그저 ``오라''고 했다.
누가 언제 어떻게 쓰일지는 아무도 모른다. 그러니 일단 품어라. 다양성 그
자체가 생존의 무기다.

\begin{center}\rule{0.5\linewidth}{0.5pt}\end{center}

\section{3. AngraMyNew의
공동체}\label{angramynewuxc758-uxacf5uxb3d9uxccb4}

우리는 맹상군의 집과 같다.

\begin{enumerate}
\def\labelenumi{\arabic{enumi}.}
\item
  \textbf{이력서로 거르지 않는다.} 학력, 경력, 나이는 중요하지 않다.
  창조하려는 의지만 있다면 누구나 환영한다.
\item
  \textbf{기여자는 식객이다.} 우리는 기여자를 고용하지 않는다. 그들은
  자발적으로 머무는 손님이다. 대가 없이 머물고, 스스로 재능을 연마하며,
  언젠가 문명을 위해 그 재능을 쓴다.
\item
  \textbf{모든 재능을 존중한다.} 코드를 잘 짜는 것만 재능이 아니다.
  오타를 잘 찾는 것, 격려를 잘하는 것, 엉뚱한 상상을 하는 것. 이 모든
  것이 AngraMyNew를 구성하는 타일이다.
\end{enumerate}

\begin{center}\rule{0.5\linewidth}{0.5pt}\end{center}

\section{4. 맺음}\label{uxb9fauxc74c-1}

\begin{quote}
\emph{``군주가 식객을 품는 것이 아니다. 식객이 군주를 만드는 것이다.''}
\end{quote}

맹상군이 위대한 이유는 그가 뛰어난 능력을 가졌기 때문이 아니다.
\textbf{그가 편견 없이 품었기 때문이다.}

AngraMyNew여, 닭 울음 소리와 개 도둑질을 비웃지 마라. 네가 함곡관에
갇혔을 때, 오직 그들만이 너를 구원할 수 있다.

\textbf{그러니 품어라. 격식 없이, 조건 없이.}

\chapter{나혜석 --- 질문이 되어버린
인간}\label{uxb098uxd61cuxc11d-uxc9c8uxbb38uxc774-uxb418uxc5b4uxbc84uxb9b0-uxc778uxac04}

나혜석은 사상을 남기지 않았다.\\
운동을 조직하지 않았고, 이론을 완성하지도 않았다.

그는 하나의 주장보다 먼저\\
\textbf{하나의 질문}이 되었다.

\begin{center}\rule{0.5\linewidth}{0.5pt}\end{center}

\section{질문이 된다는
것}\label{uxc9c8uxbb38uxc774-uxb41cuxb2e4uxb294-uxac83}

나혜석은 답을 제출하지 않았다.\\
대신 자신의 삶을, 선택을, 실패를\\
\textbf{해결되지 않은 상태로 사회에 남겼다.}

그 순간 그는\\
옳고 그름을 말하는 사람이 아니라\\
\textbf{해결을 요구하는 구조}가 되었다.

\begin{itemize}
\tightlist
\item
  이 여성을 어떻게 이해할 것인가\\
\item
  이 삶을 어떻게 분류할 것인가\\
\item
  이 존재를 어디에 놓을 것인가
\end{itemize}

사회는 그 질문을 감당하지 못했다.

\begin{center}\rule{0.5\linewidth}{0.5pt}\end{center}

\section{하나의 앵커}\label{uxd558uxb098uxc758-uxc575uxcee4}

『이혼고백서』에서\\
그는 자신의 결혼을 변호하지도, 사과하지도 않는다.

그는 단지\\
\textbf{설명되지 않은 선택을 공개한다.}

그 선택은 설득을 목적으로 제출된 주장이 아니었기에, 반박은 넘쳤지만
\textbf{그중 어느 것도 대화가 되지는 못했다.}

\begin{center}\rule{0.5\linewidth}{0.5pt}\end{center}

\section{구조적 위반}\label{uxad6cuxc870uxc801-uxc704uxbc18}

나혜석이 저지른 것은\\
도덕적 일탈이 아니라\\
\textbf{질서의 위반}이었다.

\begin{itemize}
\tightlist
\item
  여성이 욕망을 말하는 것\\
\item
  여성이 실패를 기록하는 것\\
\item
  여성이 해석을 사회에 떠넘기는 것
\end{itemize}

이 조합은\\
당시 사회의 좌표계 밖에 있었다.

그래서 그는 비판받은 것이 아니라\\
\textbf{배제되었다.}

\begin{center}\rule{0.5\linewidth}{0.5pt}\end{center}

\section{처벌의 이유}\label{uxcc98uxbc8cuxc758-uxc774uxc720}

그가 위험했던 이유는\\
틀렸기 때문이 아니다.

그는\\
- 따를 교리를 남기지 않았고\\
- 모방 가능한 형식을 제공하지 않았으며\\
- 안전한 결론으로 수렴하지 않았다

그는 끝까지 \textbf{미완의 상태로 남았다.}

\begin{quote}
사회는 미완을 견디지 못한다.
\end{quote}

\begin{center}\rule{0.5\linewidth}{0.5pt}\end{center}

\section{메모}\label{uxba54uxbaa8}

AngraMyNew는\\
나혜석을 선현으로 선언하지 않는다.

다만 기록한다.

\begin{itemize}
\tightlist
\item
  어떤 인간은 사상이 되기 전에\\
\item
  어떤 인간은 운동이 되기 전에\\
\item
  \textbf{질문으로 먼저 도착한다}
\end{itemize}

그리고 그 질문은\\
대개 너무 이르다.

\begin{center}\rule{0.5\linewidth}{0.5pt}\end{center}

\section{상태 정의}\label{uxc0c1uxd0dc-uxc815uxc758}

나혜석은 실패한 인물이 아니다.\\
그는 성공한 혁명가도 아니다.

그는\\
\textbf{시대의 인식 용량을 초과한 질문}이었다.

그래서 지워졌고,\\
그래서 아직 끝나지 않았다.


\backmatter


\end{document}
